% KOMA-Script layout settings
\documentclass[
	final,
	a4paper,
	ngerman,
	mpinclude = true, % include marginpar in textwidth for headsepline
	twoside = true,
	open = right,
	cleardoublepage = plain,
	DIV = 13,
	BCOR = 1cm,
	titlepage = firstiscover,
	]{scrbook}

\usepackage[T1]{fontenc}
\usepackage[utf8]{inputenc}
\usepackage[utf8]{luainputenc}

\usepackage{xspace}
\usepackage{calc} % \widthof

% margin notes
\setlength{\marginparwidth}{1.8\marginparwidth}
\setlength{\marginparsep}{10mm} % line numbers
\newcommand{\marginnote}[1]{\marginpar{\singlespacing\raggedright\footnotesize#1}}

% custom formatting for acts and scenes
\addtokomafont{sectioning}{\rmfamily\scshape\mdseries\centering}
\newcommand{\act}{\chapter}
\renewcommand*{\raggedsection}{\centering}
\renewcommand*{\chapterformat}{}
\renewcommand*{\chaptermarkformat}{}
\newcommand{\scene}{\setcounter{subscene}{1}\section}
\RedeclareSectionCommand[style=chapter]{section}
\renewcommand*{\sectionformat}{Szene \thesection~— }
\renewcommand*{\sectionmarkformat}{Szene \thesection~— }
\newcommand{\direction}[1]{(\textit{#1})}
\newcommand{\setting}[1]{\vspace{-0.5\baselineskip}\centering\textit{#1}}
\newcommand{\hiat}{%
	\begin{center}
		\tiny
		\raisebox{0.5ex}{\rule{0.3\linewidth}{0.4pt}}
		\textit{fickstrich}
		\raisebox{0.5ex}{\rule{0.3\linewidth}{0.4pt}}
	\end{center}
}
% TODO refstepcounter
\newcounter{subscene}
\setcounter{subscene}{1}
\newcommand{\subscene}{\marginnote{Szene \arabic{chapter}.\arabic{section}.\alph{subscene}}\stepcounter{subscene}}

% "Elements of Typographic Style" table of contents
\DeclareTOCStyleEntry[
		raggedpagenumber=true,
		linefill = {},
		entrynumberformat = {\phantom},
		indent = 0cm,
	]{tocline}{chapter}
\newlength{\scenenumwidth}
\setlength{\scenenumwidth}{\widthof{Szene 8.88 }}
\DeclareTOCStyleEntry[
		raggedpagenumber = true,
		linefill = {},
		indent = 0.3\linewidth,
		entrynumberformat = {{\footnotesize{\textsc{Szene}}}\enspace},
		numwidth = \scenenumwidth,
	]{tocline}{section}

% header
\usepackage[
		automark,
		headsepline,
		headwidth=textwithmarginpar,
	]{scrlayer-scrpage}
	\pagestyle{scrheadings}
	\ihead{}
	\chead{}
	\ohead{}
	\cehead{\leftmark}
	\cohead{\rightmark}

\usepackage[utf8]{inputenc}
\usepackage{babel}

% typography
\usepackage{ebgaramond}
\usepackage{microtype}
\usepackage{setspace} % one-half spacing
\usepackage[modulo,running]{lineno} % line numbers
	%\renewcommand{\thelinenumber}{\thesection.\arabic{linenumber}}
	\renewcommand{\thelinenumber}{\arabic{section}.\arabic{linenumber}}
\usepackage{csquotes}
\usepackage{siunitx}

% special version for directors
\usepackage{substr}
\newcommand{\ifdirectorsversion}[2]{%
	\IfSubStringInString{\detokenize{regie}}{\jobname}{#1}{#2}
}

% two column layout for character names and lines
\usepackage{enumitem}
\newlist{play}{description}{1}
\newlength{\widthofchar}
\setlength{\widthofchar}{\widthof{\textsc{Schwester\quad}}}
\setlist[play]{
	labelwidth=\widthofchar,
	leftmargin=!,
	font=\rmfamily\mdseries\scshape,
	itemsep=0pt,
	before={\linenumbers*}
}

% gray out deletions
\usepackage{xcolor}
\usepackage{comment}
\ifdirectorsversion{%
	\newenvironment{deletion}{%
		\vspace{0.25\baselineskip}
		\hrule
		\vspace{0.25\baselineskip}
		\color{darkgray}
	}{
		\color{black}
		\vspace{0.25\baselineskip}
		\hrule
		\vspace{0.25\baselineskip}
	}
}{%
	\excludecomment{deletion}
}

% list of characters at the beginning of a scene
\newcommand{\characterlist}[1]{{\begin{center}\textit{Personen} #1\end{center}}}

% PDF options
\usepackage[final,hidelinks]{hyperref}
	\hypersetup{
		unicode     = true,
		linktoc     = all,
		pdftitle    = {Der Nebel von Dybern},
		pdfauthor   = {Maria Lazar},
		pdfsubject  = {Ein Drama},
		pdflang     = de-DE,
		pdfdisplaydoctitle = true,
	}
	\ifdirectorsversion{\hypersetup{pdftitle={Der Nebel von Dybern (Regie-Version)}}}{}
	\addto\extrasngerman{
		\renewcommand{\chapterautorefname}{Akt}
		\renewcommand{\sectionautorefname}{Szene}
	}
\usepackage{bookmark} % toc in PDF bookmarks

% shortcuts for characters
% within line
\newcommand{\thecharacter}[1]{\textup{\textsc{#1}}\xspace}
\newcommand{\theBarbara}{\thecharacter{Barbara}}
\newcommand{\theJosef}{\thecharacter{Josef}}
\newcommand{\theKathrine}{\thecharacter{Kathrine}}
\newcommand{\theGregor}{\thecharacter{Gregor}}
\newcommand{\theJan}{\thecharacter{Jan}}
\newcommand{\theAndreas}{\thecharacter{Andreas}}
\newcommand{\theLuise}{\thecharacter{Luise}}
\newcommand{\theAgnes}{\thecharacter{Agnes}}
\newcommand{\theGeneraldirektor}{\thecharacter{Generaldirektor}}
\newcommand{\theClarisse}{\thecharacter{Clarisse}}
\newcommand{\theAlexis}{\thecharacter{Alexis}}
\newcommand{\theThomsen}{\thecharacter{Doktor Thomsen}}
\newcommand{\theJonas}{\thecharacter{Doktor Jonas}}
\newcommand{\theSalwin}{\thecharacter{Salwin}}
\newcommand{\theOberst}{\thecharacter{Oberst Brix}}
\newcommand{\theMelchior}{\thecharacter{Melchior}}
\newcommand{\theHeilsarmeeschwester}{\thecharacter{Heilsarmeeschwester}}
\newcommand{\theErsterMann}{\thecharacter{Erster Mann}}
\newcommand{\theZweiterMann}{\thecharacter{Zweiter Mann}}
\newcommand{\theSergeant}{\thecharacter{Sergeant}}
\newcommand{\theDiener}{\thecharacter{Diener}}
\newcommand{\theKinder}{\thecharacter{Kinder}}
\newcommand{\theLeute}{\thecharacter{Leute}}
\newcommand{\theSoldaten}{\thecharacter{Soldaten}}

% speaker of line
\newcommand{\character}[1]{\item[#1]}
\newcommand{\Barbara}{\character{\theBarbara}}
\newcommand{\Josef}{\character{\theJosef}}
\newcommand{\Kathrine}{\character{\theKathrine}}
\newcommand{\Gregor}{\character{\theGregor}}
\newcommand{\Jan}{\character{\theJan}}
\newcommand{\Andreas}{\character{\theAndreas}}
\newcommand{\Luise}{\character{\theLuise}}
\newcommand{\Agnes}{\character{\theAgnes}}
\newcommand{\Generaldirektor}{\character{\Generaldirektor}}
\newcommand{\Clarisse}{\character{\theClarisse}}
\newcommand{\Alexis}{\character{\theAlexis}}
\newcommand{\Thomsen}{\character{\theThomsen}}
\newcommand{\Jonas}{\character{\theJonas}}
\newcommand{\Salwin}{\character{\theSalwin}}
\newcommand{\Oberst}{\character{\theOberst}}
\newcommand{\Melchior}{\character{\theMelchior}}
\newcommand{\Heilsarmeeschwester}{\character{\theHeilsarmeeschwester}}
\newcommand{\ErsterMann}{\character{\theErsterMann}}
\newcommand{\ZweiterMann}{\character{\theZweiterMann}}
\newcommand{\Sergeant}{\character{\theSergeant}}
\newcommand{\Diener}{\character{\theDiener}}
\newcommand{\Kinder}{\character{\theKinder}}
\newcommand{\Leute}{\character{\theLeute}}
\newcommand{\Soldaten}{\character{\theSoldaten}}

% cover
\usepackage{pdfpages}

% title page
\addtokomafont{titlehead}{\scshape\lsstyle}
\titlehead{\centering Wery Important Production Berlin}
\title{Der Nebel von Dybern}
\subtitle{Ein Drama}
\author{Maria Lazar}
\date{\ifdirectorsversion{-- Regie-Version --}{}}
\publishers{S. Fischer Verlag}
\uppertitleback{%
	\centering
	\addsec*{Dramatis Person\ae}
	\vspace{\baselineskip}
	\raggedright
    \theBarbara, \quad schwanger\\
    \theJosef, \quad ihr Mann\\
    \theKathrine, \quad blind\\
    \theGregor\\
    \theJan\\
    \theAndreas\\
    \theLuise\\
    \theAgnes\\
    Paul, der \theGeneraldirektor der Chemiefrabrik\\
    \theClarisse, \quad seine Frau\\
    \theAlexis, \quad Ingenieur\\
    \theThomsen, \quad Arzt\\
    \theJonas, \quad Arzt\\
    \theSalwin, \quad Journalist\\
    \theOberst\\
    Jakob \theMelchior\\
    \theHeilsarmeeschwester\\
    \theErsterMann\\
    \theZweiterMann\\
    \theSergeant\\
    \theDiener\\
    \theKinder,  \theLeute und \theSoldaten.
}
\lowertitleback{%
	\footnotesize
	\centering
	Version vom \today.\\
	\vspace{0.5\baselineskip}
	Erschienen 1932 bei \textsc{S. Fischer, Verlag A.G. Berlin}.\\
	\vspace{0.5\baselineskip}
	Gesetzt mit \LaTeX{} und \KOMAScript{} in EBGaramond.\\
}

\begin{document}
\pagenumbering{alph}
%\includepdf{cover/cover}
\cleardoubleoddemptypage

\pagenumbering{roman}
\maketitle

\pdfbookmark[chapter]{\contentsname}{toc}
\tableofcontents
\cleardoubleoddpage

\pagestyle{headings}
\pagenumbering{arabic}
\doublespacing

\act{Erster Akt}
\scene{Der Nebel}
\characterlist{\theBarbara, \theJosef, \theKathrine, \theGregor, \theJan, \theAndreas, \theThomsen, \theAgnes}
\setting{Eine einfache, saubere Wirtsstube. Frühe Nachmittagsdämmerung, schläfriges Licht. Auf der Fensterbank sitzt \theJosef mit einer Zeitung, an den Kachelofen gelennt, hockt die alte \theKathrine. \theJosef ist ein behäbiger, etwas dicker Mensch. \theKathrine ist blind. Sie starrt immer vor sich hin, als ob sie etwas sehen würde.}

\begin{play}

\Barbara
\direction{Eine tiefe volle Frauenstimme singt} \ldots Eia popeia, was raschelt im Stroh ---

\Josef
\direction{hebt den Kopf gegen die Decke} Barbaral

\Barbara
\direction{singend} Ja -

\Josef
Wenn du schon wach bist, dann komm dochherunter. Wir wollen Kaffee.

\Barbara
Ist jemand da?

\Josef
Mutter Katharine.

\Barbara
\direction{weiter singend) .. das sind die lieben Gänslein, die haben kein Schuh, derSchuster hat's Leder, kein Leisten dazu. (Stimme verklingt}

\Kathrine
Lass sie in Ruh. Sie kommt immernoch früh genug. Sie kommt viel zu früh,und wir brauchen keinen Kaffee.

\Josef
Ja, ja, schon gut. \direction{blättert in derZeitung}

\Kathrine
Was steht denn da in der Zeitungdrin? Du liest doch die Zeitung. SchöneGeschichten, Sonntagsgeschichten?

\Josef
Ja, ja, so was ähnliches.

\Kathrine
Ich brauch keine Zeitung, ich kannsie nicht lesen, ich hab keine Augen. Ichhab meine Ohren.

\Josef
Fang nur nicht wieder an mit den altenGeschichten.

\Kathrine
Es sind gar keine alten Geschichten.Das weisst du sehr gut, davon spricht einjeder. Erst heut nach der Kirche -

\Josef
Jetzt schweig schon still, ich willnichts weiter mehr hören. Und überhaupt,wenn Barbara herunterkommt. Frauen in ihrem Zustand -

\Kathrine
In ihrem Zustand, in ihrem Zustand.Wer hat sie denn in den Zustand gebracht.Das treibte und vögelt und denkt dabei andie Folgen nicht weiter.

\Josef
Halts Maul, man wird noch sein Kindkriegen dürfen.

\Kathrine
Sein Kind kriegen dürfen. Barmherziger Himmel! Hast denn Milch für dein Kind?Und reines Wasser? Und saubere Luft?

\Josef
Ja, ja, ja und noch viel, viel mehr.

\Kathrine
Mir ist mein Mädel chen an der Brustverhungert. Und meinen Buben haben siemir aus dem Feld gebracht, ich hab ihnnimmer erkannt. Gott sei Dank, dass ichjetzt nicht mehr sehen brauch. Ihr aber,ihr müsst Kinder kriegen.

\Josef
Verflucht nochmal! \direction{springt auf} Dasist doch das ist doch zehn, zwölf,fünfzehn Jahre her. Wir haben keinen Kriegmehr. Hast du verstanden!

\Kathrine
Das sagen alle, aber es nicht wahr.Es ist eine schlechte Luft in der Welt.Wenn es auch nicht in der Zeitung steht.

\Josef
Kein Wort weiter. Barbara kommt.

\Barbara
\direction{stösst die Tür auf. Sie ist einegrosse Frau, stark in der Hoffnung) Tag,Mutter Kathrine. Schön, dass du wiedermal zu uns gefunden hast, (räkelt sich}Ach Gott, ach Gott, bin ich faul. Wiekann man nur am Nachmittag so schlafen.

\Kathrine
Das ist gut, das ist recht, das istso am besten. Schlaf du nur. Kannst garnicht genug schlafen.BARBARA Wie? Was meinst du?

\Josef
\direction{steht auf, ungeduldig, macht ein Zeichen an der Stirn} Lass sein, Barbara.Was ist mit unserem Kaffee?

\Barbara
Gleich, gleich, das Wasser ist schonaufgestellt. Aber hier ist ein Dampf. ZumErsticken. Hast wie der einmal nichtschlecht gepafft. \direction{geht zum Fenster undstösst es auf}

\Kathrine
\direction{schnuppernd} Macht das Fenster zu, 05macht das Fenster zu.

\Barbara
Ach lass doch. Das bisschen frischeLuft.

\Kathrine
Das ist nicht Luft, das ist Nebel.

\Barbara
\direction{beugt sich hinaus} Was für ein komischer gelber Nebel. Man sieht ja nichteinmal die Linde mehr.

\Josef
\direction{schliesst das Fenster) Genug gelüftet,es kommt kalt herein. (zeigt auf einenTisch} Und nimm dort doch die Kinderwäsche weg. Es werden sicherlich bald Gästkommen.

\Barbara
\direction{legt die Wäsche zusammen} Eins, zweidrei, vier, fünf, sechs Hemdchen. Nocheinmal sechs, dann sind zwei Dutzendvoll. Die feinen Säumchen näht Agnes. Da sMädel hat wirklich unglaubliche Augen.

\Josef
Wo ist Agnos denn heute?

\Barbara
Sie ist schon früh morgens nach Dybern gegangen. Zu Annemarie. Die liegtimmer noch krank. Agnes bringt ihr Kirschenkompott.

\Josef
Nach Dybern. Sag mal, du hast sie dochnicht allein gehen lassen?

\Barbara
Warum denn nicht?

\Josef
Es ist nur - ich meine - es ist scheussliches Wetter - nass und kalt. Und dannplötzlich stock finster am hellichten Tag\direction{dreht das Licht an}

\Barbara
\direction{ist inzwischen in die Küche gegangen, wo man sie, die Tür bleibt offen,herumhantieren sieht} Sie geht den Wegja nicht zum ersten Mal.

\Josef
Wird sie denn nicht auf der Strassekommen?

\Barbara
Das glaub ich kaum. Der Weidenweg istdoch viel näher. \direction{summt} Die Gänslein gehen barfuss und haben kein Schuh.

\Josef
\direction{ist inzwischen auf und ab gegangen)Wann kommt sie denn zurück?BARBARA (von der Küche her} Wie?

\Josef
Wann soll Agnes zurück sein?BARBARA \direction{kommt mit einem Tablett herein) Ehes finster wird. (stellt den Kaffee vorKathrine) Da, Kathrine, greif zu. Im Kaffee ist viel Haut, und da hast du auchein paar feine Kuchen. (sieht plötzlicherstaunt zum Fenster hin} Ach, du meineGüte, es ist ja schon finster.

\Josef
Du hättest das Mädel doch nicht so allein hinauslassen sollen.

\Barbara
Sag mal, Josef, was hast du dennheute?

\Josef
Ach, gar nichts. Gib mir die Tasse her.

\Barbara
Da ist was los, du verschweigst mirWas.

\Josef
Ich verschweig dir nichts. Ist ja allesnur dummes Gewäsch. Gut, dass du heutenicht auf dem Kirchplatz warst..

\Barbara
Du, Josef, Jetzt will ich aber wirklich schon wissen -

\Josef
Da gibts nichts zu wissen. Lass mich inRuh. Man wird ja selber schon ganz verblödet, so einen gottsverfluchten, neblidogen Sonntag lang. Von Gerüchten darf mansich nicht ins Boxhorn jagen lassen, undüberhaupt, eine Frau wie du, in deinemZustand -

\Barbara
Josef, jetzt mach mir nicht längerwas vor. Ich merk es doch die ganzenletzten Tage. Da wird immerfort nur geflüstert und getuschelt, und keinersehaut einem mehr grad ins Gesicht.Glaubst du, ich merk so was nicht? Ichspür es schon auf der ganzen Haut.

\Kathrine
Es ist eine schlechte Luft in derWelt.

\Barbara
Wie? Was heisst das?

\Josef
Hör nicht auf sie. Sie hat wieder ihren verrückten Tag. Man hat ein paar toteRehe gefunden, zwischen den Weiden, undhinter Dybern, im Strassengraben auchnoch eine verreckte Kuh.

\Barbara
Und?

\Josef
Undden Leuten wurde schlecht, alssie das Vieh so liegen sahen. Ein alterBauer war es und sein Sohn. Müssen jarechte Helden sein, die beiden. ARU

\Barbara
Und?

\Josef
Und weiter nichts. Kannst das allesselbst in der Zeitung lesen. Dort stehtauch von den unsinnigen Gerüchten.

\Barbara
Wenn du an die Gerüchte nichtglaubst, weshalb verschweigst du sie mir?

\Josef
Weisst du, Barbara, in deinem Zustand -und seit du uns unlängst erst der Länge8nach auf der Nase lagst wegen dem Försterhund

\Barbara
Wenn einem plötzlich mitten am Tagein Hund in der Stube erschossen wird,und das Vieh liegt da und hat ganz blaueAugen da kann einem leicht bisschenschwindlig werden.-

\Josef
Lass gut sein, Barbara, reg dich nichtwieder auf.

\Barbara
Ach was, sprich nicht immer mit mir,als wär ich jetzt nicht ganz bei Trost.Und was das Kleine ist, das liegt gut undsicher in meinem Bauch. Gib die Zeitunghe. \direction{setzt sich mit der Zeitung an einen Tisch) (Josef drückt auf den Knopf desRadios, leichte Schlagermusik}

\Barbara
\direction{hebt nach ein paar Sekunden plötzlich den Kopf und sagt sehr laut}SagFörmal Josef, weiss man genau, dass dersterhund wirklich die Toll wut hatte?

\Josef
\direction{zuckt die Achseln) - Ich hab noch niemand gefragt.(die Tür wird aufgestossen. HereinkommenGregor, Jan, Andreas und Luise. Gregorist ein älterer Mann. Jan ein spindeldümrer, zappliger Kerl, Andreas ein kräftiger schöner Bursche, Luise ein schlankesMädchen, Typus der intelligenten Arbeiterin}

\Gregor
Oh, hier ist fein warm.

\Jan
Und immer Musik. Da gibts keine steifenBeine nicht. Komm, Luise. \direction{legt den Armum ihre Hüften}

\Luise
Lass los! Grüss Gott, Barbara.

\Gregor
\direction{zu Andreas, der als letzter kommt}Mach die Tür zu, Andreas.

\Luise
Mein Gott, der Nobel, das ist ja schonwie Rauch.

\Jan
Aber hier ist es gleich. Hier sind wirfidel. \direction{johlt zur Musik} Denn das Wirtshaus am Rand, das ist ja beknnt im ganzenLand.

\Gregor
\direction{zu Josef} Einen Scharfen, der einheizt!

\Josef
\direction{indem er ihm einschenkt) Der Jankriegt nichts. Der ist ja jetzt schon besoffen. (stellt das Radio ab}

\Jan
Oho.

\Josef
Woher kommt ihr denn?

\Gregor
Von dem neuen Kino. Dort gibt es jetzteinen grossen Ausschank.

\Josef
Spielt es denn schon?

\Gregor
Nein, noch nicht, aber man kann essich ansehen. Was rennst du denn so herum, Andreas? Was suchst du denn?

\Jan
Na der, der sucht doch natürlich die Agnes.

\Andreas
So schweig schon einmal.

\Jan
Frau Wirtin, wo ist denn das FräuleinSchwester? Das entzückende, das reizendeFräuleinchen Schwesterchen?

\Barbara
\direction{sieht verwirrt von der Zeitung auf}Agnes?

\Josef
Sie wird gleich kommen, sie war in Dybern, ein Krankenbesuch. Sie muss jedenAugenblick da sein. Aber erzählt dochlieber, wie ist das Kino?

\Jan
Fein, pickfein, viel zu fein für uns arme Leute. Wenn du mal erst die Marmortreppe runtersteigst -

\Josef
Ist es wahr, dass es ganz unter derErde ist?

\Jan
Ganz unter der Erde. Das ist jetzt dasNeueste, das Modernste. Das ist dasSchönste und das Gesündeste und das Billigste. Einen eigenen Architekten haben10sie sich dazu herbestellt. Dicht an derFabrik ist es auch. Man braucht sich nachder Arbeit bloss die Hände waschen -

\Gregor
\direction{schlägt auf den Tisch} Und jetzt sagmir nur, was dir schon wieder daran nichtrecht ist?

\Josef
Der Jan ist bös, wenn er nicht Grund genug zum Stänkern hat.

\Jan
Und ihr seid alle miteinand Idioten. Wenneuch die hohe Direktion mal ein Zuckerstück hinhält, dann schnappt ihr danach.Sonst kann sie getrost auf eure Köpfespucken.

\Gregor
Es spuckt keiner auf unsere Köpfe. Undwenn sie uns ein Kino hinstellen, so verdienen sie schließlich selber daran.

\Luise
\direction{nachdenklich} Das muss ungeheuer vielgekostet haben, so ein Riesenkino ganzunter der Erde.

\Jan
Jetzt sag mir nur einer, warum ist esdenn ganz unter der Erde?

\Gregor
Damit du das Gras wachsen hörst, duRotzbub. Das tust du ja ohnehin so gern.

\Andreas
Ich versteh aber auch nicht, warumsie es so hinunter bauen.

\Gregor
Jetzt fängst du auch an.

\Jan
Wir sind doch keine Maulwürfe.

\Gregor
Brauchst ja nicht runter, wenn es dirnicht passt.

\Kathrine
Mein Bub war auch in so einem Kinounter der Erde. Es war sehr nass.

\Josef
Schon gut, schon gut, Kathrine, sprichda nicht mit.

\Kathrine
Ihr werdet alle noch hinunter müssen. Damals hat auch ein jeder geglaubt,es trifft nur den andern.

\Josef
Aber es ist doch ein Kino, Kathrine,ein Kino.11 -

\Jan
Schrei nicht so, sie ist ja nicht taub,nur blind. \direction{greift nach Gregors Glas}

\Gregor
\direction{hält ihn zurück} Lass sein, bist jaohnehin schon besoffen.

\Jan
\direction{reisst ihm das Glas aus der Hand undtrinkt es aus} - Natürlich bin ich besoffen, ich bin ja immer besoffen, und wennmir mal die Luft ausgeht in unserer altenStink bude drüben, dann bin ich auch nurStinkbudebesoffen, fragt doch den Doktor Thomsen,der hat das gesagt, und wenn ein Mädelumfällt, mitten in der Arbeit, dann binich auch nur besoffen, und wenn einer diegewissen Flecken kriegt, die blauen Flekken, erst an den Händen, dann in den Augen --

\Barbara
\direction{die bisher mit der Zeitung vor sichwie teilnahmslos gesessen ist} Was fürFlecken, was für blaue Flecken?

\Luise
Halts Maul, Jan, du redst dich noch umdeinen Kopf.

\Gregor
Pack dein Bündel und geh, hat dichja keiner nicht hergebeten, bist doch sonur ein Fremder. Geh du dort hin, wo esdem Arbeiter wirklich schlecht geht, woer kein Fressen hat, kein Dach übermKopf. Dort kannst du deine Reden halten,Gefahren gibts in jeder Fabrik.

\Josef
Und in unserer Gegend sind die bestenLöhne. Das weiss ein jeder. Es liegt einSegen über dem ganzen Land. Blumen hinterallen Fensterscheiben. Wenn ich denke.wic es früher gewesen ist. Nichts aufs Brothaben die Leute gehabt.

\Barbara
\direction{steht auf} - Sagt mal, was sind denndas für blaue Flecken?

\Gregor
Nur aus Unvorsichtigkeit. Ihr könnteuch drauf verlassen, immer nur aus12Unvorsichtigkeit. Was predige ich nichttäglich den Leuten. Eine Stickstoffabrikist schliesslich kein Kinderzimmer.

\Luise
Nein, Gregor, das ist eine Gemeinheit,was du sagst. Und ausserdem bestimmtnicht wahr. Es tut nicht gut, wenn einervon uns so spricht.

\Gregor
Dir ist natürlich lieber, wenn einerhetzt, wie dein Jan. Siehst dich wohlauch schon Versammlungen halten. Und waskommt dabei raus - nichts als Not undElend. Schau du lieber, dass du einenbraven Mann kriegst und ein paar Kinderund ein Häuschen in unserer Siedlung.

\Andreas
\direction{sieht auf die Uhr} Wann soll die Agnes von Dybern zurück sein?

\Josef
\direction{sieht zum Fenster hinaus} Es iststock finster. Vielleicht sollte man ihmentgegengehen.

\Luise
Aber die Strasse ist doch gut beleuchtet.

\Barbara
Sie kommt auf dem Weidenweg.

\Andreas
Auf dem Weidenweg!

\Jan
Herr des Himmels, was schickt ihr siedenn auf den Weidenweg!

\Gregor
Auf dem Weidenwog ist der Nebel amschlimmsten. Dort steigt er auf vomFluss.

\Barbara
\direction{sehr heftig} Ja seid ihr denn alleverrückt geworden. Das Mädel hat docheine Taschenlampe bei sich. Die geht denWeg zum hundertsten Mal.

\Luise
Aber der Nebel.

\Barbara
Am Nebel ist noch keiner gestorben.

\Andreas
\direction{steht auf} Gebt mir eine Lampe. Ichgeh ihr entgegen.

\Josef
\direction{geht eine Laterne holen}-

\Andreas
Macht rasch, rasch - Agnes - meinGott Agnes es wird ihr doch nichtsgeschehen sein. \direction{reisst Josef die Laterneaus der Hand und stürzt hinaus. Barbarasteht mitten im Zimmer und zählt währenddes Sprechens an den Fingern}

\Barbara
Ein paar Rehe, eine Kuh, einem Bauern ist schlecht geworden und seinemSohn. In der Zeitung steht, das sind nurGerüchte, in der Zeitung steht, es istnicht der Nebel, in der Zeitung steht,man kann ja auch im Wald krank werden undauf offenem Feld, und Tiere sterbeneben - aber der Hund - sag mal, Gregor,hat der Hund wirklich die Tollwut gehabt?

\Gregor
Was weiss denn ich.

\Barbara
Ja, warum wisst ihr das denn allenicht?

\Luise
Es ist ein ungesunder Herbst, Barbara.Und wenn der Nebel einmal so dick wirdwie eine nasse Mauer -

\Josef
Sollst dir nicht so viel den Kopf zerbrechen, Barbara. Gesunden Lungen machtder Nebel nichts. Nur wenn einer ohnehinAsthma hat oder sonstwie schweren Atem

\Jan
Und wo doch die Hauptsache ist, dass dieBevölkerung ruhig bleibt, immer nur ruhig, wie es an unserer Kirche angeschlagen steht, und der Herr Pfarrer hat jaauch gepredigt, dass Gott es gar nichtso böse meint. Man braucht doch nicht inden Wald zu gehen und zum Fluss

\Josef
Halts Maul, Kerl, oder ich schmeissdich hinaus.

\Barbara
\direction{sinkt plötzlich auf einem Stuhl zusammen} Agnes - o Gott, warum habt ihrmir das nicht früher gesagt.

\Luise
Aber, Barbara, es ist doch nichts geschehen.

\Josef
Das kommt nur von den verfluchten Gerüchten.

\Gregor
Im Nebel kann doch keine Krankheitsein.

\Jan
\direction{steht auf} Ich geh dem Andreas nach.

\Barbara
Was ist denn aus dem Bauern gewordenund seinem Sohn?\direction{Josef und Gregor zucken die Achseln}

\Barbara
Ja, warum wisst ihr das denn allenicht?\direction{man hört den Motor eines Autos und hupen.Josef geht hinaus und kommt gleich daraufmit Doktor Thomsen zurück. Thomsen istein älterer, robuster Mann}

\Thomsen
\direction{schüttelt sich vor Nässe} \ldots GutonAbend.

\Gregor
Guten Abend.

\Luise
Gott sei Dank, der Doktor.

\Thomsen
Wieso? Was ist denn?

\Josef
Es ist keine Ruhe im Land, Herr Doktor

\Thomsen
Ja, ja, schon gut. Gebt mir rasch cinen starken Grog, und dann hilft mir ciner bei meiner Panne. \direction{setzt sich mürrisch in eine Ecke}

\Josef
\direction{aus der Küche heraus} Wollen Herr Doktor auch was essen?

\Thomsen
Nein.

\Barbara
Woher kommt denn der Doktor?

\Thomsen
Aus Dybern.

\Luise
Von einem Kranken?

\Thomsen
Nein.

\Gregor
Von sonst einem Besuch?

\Thomsen
Nein.

\Barbara
\direction{steht auf, stellt sich vor Thomsen,voll Angst} woher denn sonst?

\Thomsen
Von einem Toten.\direction{Stille}

\Thomsen
Von zwei Toten, wenn ihr es wissenwollt.

\Barbara
Der Bauer und sein Sohn. \direction{sinkt wieder in ihrem Stuhl zusammen}

\Josef
\direction{indem er den Grog bringt} \ldots Asthma,Herr Doktor, nicht wahr, ein schlimmerAtem.

\Luise
\direction{sehr klar} Ist es wahr, dass eine Seuche im Nebel steckt? Dass man krank wirdim Wald?

\Thomsen
Es scheint so zu sein.

\Luise
Warum sperrt man dann den Wald nichtab?

\Thomsen
Weil man nicht wissen kann, wie derWind sich dreht.

\Gregor
Und woher kommt die Krankheit soplötzlich?

\Thomsen
Das weiss Gott allein.

\Kathrine
Du sollst den Namen Gottes nichteitel nennen.

\Thomsen
\direction{fährt zusammen} Was ist das?

\Josef
Entschuldigen, Herr Doktor, es ist nurunsere alte Kathrine. \direction{macht wieder dasZeichen an der Stirn. Zu Barbara, dieplötzlich ein Tuch vom Haken reisst undzur Tür geht} Was ist, wohin willst du?

\Barbara
Ich geh dem Mädel entgegen. \direction{stockt,denn man hört ein paar schrille Pfiffe}

\Luise
\direction{springt auf} Das ist Jan, das ist seine Pfeife.STIMME von

\Andreas
\ldots Macht auf, macht raschauf.\direction{Josef reisst die Tür auf, Jan und Andreas tragen Agnes herein. Agnes, ein fünfzehnjähriges zartes Mädchen, gelb im Gesicht, mit entsetzlichen Augen}

\Agnes
Wasser - Wasser

\Thomsen
(wirft seinen Mantel auf eine Bank\ldots legt sie her, sofort.

\Agnes
Wasser \direction{mit den Händen an der Brust}Ich brenne.

\Luise
\direction{hält ihr ein Glas Wasser hin} Hier.

\Agnes
\direction{trinkt} Ich brenne, ich brenne.

\Josef
Herr Doktor, Herr Doktor!

\Barbara
\direction{schiebt Agnes ein Tuch unter denKopf} Agnes.

\Andreas
\direction{wirft sich neben ihr hin} Agnes.

\Agnes
Ich brenne.

\Thomsen
Nicht weiter trinken. \direction{er reisst ihrdas Glas aus der Hand}

\Agnes
Das Wasser brennt. \direction{sinkt einen Augenblick zurück}

\Kathrine
\direction{sie ist bis jetzt im Hintergrundan der anderen Seite des Zimmers gesessen. Nun geht sie, den Stock gehoben, inder Richtung von Agnes wie gezogen auf Adas Mädchen zu} \ldots Es riecht nach Senf.

\Agnes
Ich verbrenne.Vorhang

\end{play}

\scene{Kriesensitzung}
\scene{Im Flüchtlingslager}
\scene{Hungrige Mäuler}
\scene{Nachbohren}

\act{Zeiter Akt}
\scene{Irrungen}
\scene{Zusammenbruch}

\end{document}
