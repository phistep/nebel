% KOMA-Script layout settings
\documentclass[
	final,
	a4paper,
	ngerman,
	mpinclude = true, % include marginpar in textwidth for headsepline
	twoside = true,
	open = right,
	cleardoublepage = plain,
	DIV = 13,
	BCOR = 1cm,
	titlepage = firstiscover,
	]{scrbook}

\usepackage[T1]{fontenc}
\usepackage[utf8]{inputenc}
\usepackage[utf8]{luainputenc}

\usepackage{xspace}
\usepackage{calc} % \widthof

% margin notes
\setlength{\marginparwidth}{1.8\marginparwidth}
\setlength{\marginparsep}{10mm} % line numbers
\newcommand{\marginnote}[1]{\marginpar{\singlespacing\raggedright\footnotesize#1}}

% custom formatting for acts and scenes
\addtokomafont{sectioning}{\rmfamily\scshape\mdseries\centering}
\newcommand{\act}{\chapter}
\renewcommand*{\raggedsection}{\centering}
\renewcommand*{\chapterformat}{}
\renewcommand*{\chaptermarkformat}{}
\newcommand{\scene}{\setcounter{subscene}{1}\section}
\RedeclareSectionCommand[style=chapter]{section}
\renewcommand*{\sectionformat}{Szene \thesection~— }
\renewcommand*{\sectionmarkformat}{Szene \thesection~— }
\newcommand{\direction}[1]{(\textit{#1})}
\newcommand{\setting}[1]{\vspace{-0.5\baselineskip}\centering\textit{#1}}
\newcommand{\hiat}{%
	\begin{center}
		\tiny
		\raisebox{0.5ex}{\rule{0.3\linewidth}{0.4pt}}
		\textit{fickstrich}
		\raisebox{0.5ex}{\rule{0.3\linewidth}{0.4pt}}
	\end{center}
}
% TODO refstepcounter
\newcounter{subscene}
\setcounter{subscene}{1}
\newcommand{\subscene}{\marginnote{Szene \arabic{chapter}.\arabic{section}.\alph{subscene}}\stepcounter{subscene}}

% "Elements of Typographic Style" table of contents
\DeclareTOCStyleEntry[
		raggedpagenumber=true,
		linefill = {},
		entrynumberformat = {\phantom},
		indent = 0cm,
	]{tocline}{chapter}
\newlength{\scenenumwidth}
\setlength{\scenenumwidth}{\widthof{Szene 8.88 }}
\DeclareTOCStyleEntry[
		raggedpagenumber = true,
		linefill = {},
		indent = 0.3\linewidth,
		entrynumberformat = {{\footnotesize{\textsc{Szene}}}\enspace},
		numwidth = \scenenumwidth,
	]{tocline}{section}

% header
\usepackage[
		automark,
		headsepline,
		headwidth=textwithmarginpar,
	]{scrlayer-scrpage}
	\pagestyle{scrheadings}
	\ihead{}
	\chead{}
	\ohead{}
	\cehead{\leftmark}
	\cohead{\rightmark}

\usepackage[utf8]{inputenc}
\usepackage{babel}

% typography
\usepackage{ebgaramond}
\usepackage{microtype}
\usepackage{setspace} % one-half spacing
\usepackage[modulo,running]{lineno} % line numbers
	%\renewcommand{\thelinenumber}{\thesection.\arabic{linenumber}}
	\renewcommand{\thelinenumber}{\arabic{section}.\arabic{linenumber}}
\usepackage{csquotes}
\usepackage{siunitx}

% special version for directors
\usepackage{substr}
\newcommand{\ifdirectorsversion}[2]{%
	\IfSubStringInString{\detokenize{regie}}{\jobname}{#1}{#2}
}

% two column layout for character names and lines
\usepackage{enumitem}
\newlist{play}{description}{1}
\newlength{\widthofchar}
\setlength{\widthofchar}{\widthof{\textsc{Schwester\quad}}}
\setlist[play]{
	labelwidth=\widthofchar,
	leftmargin=!,
	font=\rmfamily\mdseries\scshape,
	itemsep=0pt,
	before={\linenumbers*}
}

% gray out deletions
\usepackage{xcolor}
\usepackage{comment}
\ifdirectorsversion{%
	\newenvironment{deletion}{%
		\vspace{0.25\baselineskip}
		\hrule
		\vspace{0.25\baselineskip}
		\color{darkgray}
	}{
		\color{black}
		\vspace{0.25\baselineskip}
		\hrule
		\vspace{0.25\baselineskip}
	}
}{%
	\excludecomment{deletion}
}

% list of characters at the beginning of a scene
\newcommand{\characterlist}[1]{{\begin{center}\textit{Personen} #1\end{center}}}

% PDF options
\usepackage[final,hidelinks]{hyperref}
	\hypersetup{
		unicode     = true,
		linktoc     = all,
		pdftitle    = {Der Nebel von Dybern},
		pdfauthor   = {Maria Lazar},
		pdfsubject  = {Ein Drama},
		pdflang     = de-DE,
		pdfdisplaydoctitle = true,
	}
	\ifdirectorsversion{\hypersetup{pdftitle={Der Nebel von Dybern (Regie-Version)}}}{}
	\addto\extrasngerman{
		\renewcommand{\chapterautorefname}{Akt}
		\renewcommand{\sectionautorefname}{Szene}
	}
\usepackage{bookmark} % toc in PDF bookmarks

% shortcuts for characters
% within line
\newcommand{\thecharacter}[1]{\textup{\textsc{#1}}\xspace}
\newcommand{\theBarbara}{\thecharacter{Barbara}}
\newcommand{\theJosef}{\thecharacter{Josef}}
\newcommand{\theKathrine}{\thecharacter{Kathrine}}
\newcommand{\theGregor}{\thecharacter{Gregor}}
\newcommand{\theJan}{\thecharacter{Jan}}
\newcommand{\theAndreas}{\thecharacter{Andreas}}
\newcommand{\theLuise}{\thecharacter{Luise}}
\newcommand{\theAgnes}{\thecharacter{Agnes}}
\newcommand{\theGeneraldirektor}{\thecharacter{Generaldirektor}}
\newcommand{\theClarisse}{\thecharacter{Clarisse}}
\newcommand{\theAlexis}{\thecharacter{Alexis}}
\newcommand{\theThomsen}{\thecharacter{Thomsen}}
\newcommand{\theJonas}{\thecharacter{Jonas}}
\newcommand{\theSalwin}{\thecharacter{Salwin}}
\newcommand{\theOberst}{\thecharacter{Oberst Brix}}
\newcommand{\theMelchior}{\thecharacter{Melchior}}
\newcommand{\theHeilsarmeeschwester}{\thecharacter{Heilsarmeeschwester}}
\newcommand{\theErsterMann}{\thecharacter{Erster Mann}}
\newcommand{\theZweiterMann}{\thecharacter{Zweiter Mann}}
\newcommand{\theSergeant}{\thecharacter{Sergeant}}
\newcommand{\theDiener}{\thecharacter{Diener}}
\newcommand{\theKinder}{\thecharacter{Kinder}}
\newcommand{\theLeute}{\thecharacter{Leute}}
\newcommand{\theSoldaten}{\thecharacter{Soldaten}}

% speaker of line
\newcommand{\character}[1]{\item[#1]}
\newcommand{\Barbara}{\character{\theBarbara}}
\newcommand{\Josef}{\character{\theJosef}}
\newcommand{\Kathrine}{\character{\theKathrine}}
\newcommand{\Gregor}{\character{\theGregor}}
\newcommand{\Jan}{\character{\theJan}}
\newcommand{\Andreas}{\character{\theAndreas}}
\newcommand{\Luise}{\character{\theLuise}}
\newcommand{\Agnes}{\character{\theAgnes}}
\newcommand{\Generaldirektor}{\character{\Generaldirektor}}
\newcommand{\Clarisse}{\character{\theClarisse}}
\newcommand{\Alexis}{\character{\theAlexis}}
\newcommand{\Thomsen}{\character{\theThomsen}}
\newcommand{\Jonas}{\character{\theJonas}}
\newcommand{\Salwin}{\character{\theSalwin}}
\newcommand{\Oberst}{\character{\theOberst}}
\newcommand{\Melchior}{\character{\theMelchior}}
\newcommand{\Heilsarmeeschwester}{\character{\theHeilsarmeeschwester}}
\newcommand{\ErsterMann}{\character{\theErsterMann}}
\newcommand{\ZweiterMann}{\character{\theZweiterMann}}
\newcommand{\Sergeant}{\character{\theSergeant}}
\newcommand{\Diener}{\character{\theDiener}}
\newcommand{\Kinder}{\character{\theKinder}}
\newcommand{\Leute}{\character{\theLeute}}
\newcommand{\Soldaten}{\character{\theSoldaten}}

% cover
\usepackage{pdfpages}

% title page
\addtokomafont{titlehead}{\scshape\lsstyle}
\titlehead{\centering Wery Important Production Berlin}
\title{Der Nebel von Dybern}
\subtitle{Ein Drama}
\author{Maria Lazar}
\date{\ifdirectorsversion{-- Regie-Version --}{}}
\publishers{S. Fischer Verlag}
\uppertitleback{%
	\centering
	\addsec*{Dramatis Person\ae}
	\vspace{\baselineskip}
	\raggedright
    \theBarbara, \quad schwanger\\
    \theJosef, \quad ihr Mann\\
    \theKathrine, \quad blind\\
    \theGregor\\
    \theJan\\
    \theAndreas\\
    \theLuise\\
    \theAgnes\\
    Paul, der \theGeneraldirektor der Chemiefrabrik\\
    \theClarisse, \quad seine Frau\\
    \theAlexis, \quad Ingenieur\\
    Doktor \theThomsen, \quad Arzt\\
    Doktor \theJonas, \quad Arzt\\
    \theSalwin, \quad Journalist\\
    \theOberst\\
    Jakob \theMelchior, \quad Militär\\
    \theHeilsarmeeschwester\\
    \theErsterMann\\
    \theZweiterMann\\
    \theSergeant\\
    \theDiener\\
    \theKinder,  \theLeute und \theSoldaten.
}
\lowertitleback{%
	\footnotesize
	\centering
	Version vom \today.\\
	\vspace{0.5\baselineskip}
	Erschienen 1932 bei \textsc{S. Fischer, Verlag A.G. Berlin}.\\
	\vspace{0.5\baselineskip}
	Gesetzt mit \LaTeX{} und \KOMAScript{} in EBGaramond.\\
}

\begin{document}
\pagenumbering{alph}
%\includepdf{cover/cover}
\cleardoubleoddemptypage

\pagenumbering{roman}
\maketitle

\pdfbookmark[chapter]{\contentsname}{toc}
\tableofcontents
\cleardoubleoddpage

\pagestyle{headings}
\pagenumbering{arabic}
\doublespacing

\act{Erster Akt}
\scene{Der Nebel}
\characterlist{\theBarbara, \theJosef, \theKathrine, \theGregor, \theJan, \theAndreas, \theThomsen, \theAgnes}
\setting{Eine einfache, saubere Wirtsstube. Frühe Nachmittagsdämmerung, schläfriges Licht. Auf der Fensterbank sitzt \theJosef mit einer Zeitung, an den Kachelofen gelennt, hockt die alte \theKathrine. \theJosef ist ein behäbiger, etwas dicker Mensch. \theKathrine ist blind. Sie starrt immer vor sich hin, als ob sie etwas sehen würde.}

\begin{play}

\Barbara
\direction{Eine tiefe volle Frauenstimme singt} \emph{\ldots Eia popeia, was raschelt im Stroh ---}

\Josef
\direction{hebt den Kopf gegen die Decke} Barbara

\Barbara
\direction{singend} \emph{Ja ---}

\Josef
Wenn du schon wach bist, dann komm doch herunter. Wir wollen Kaffee.

\Barbara
Ist jemand da?

\Josef
Mutter Katharine.

\Barbara
\direction{weiter singend} \emph{.. das sind die lieben Gänslein, die haben kein' Schuh, der Schuster hat's Leder, kein' Leisten dazu.} \direction{Stimme verklingt}

\Kathrine
Lass sie in Ruh. Sie kommt immernoch früh genug. Sie kommt viel zu früh,und wir brauchen keinen Kaffee.

\Josef
Ja, ja, schon gut. \direction{blättert in der Zeitung}

\Kathrine
Was steht denn da in der Zeitung drin? Du liest doch die Zeitung. Schöne Geschichten, Sonntagsgeschichten?

\Josef
Ja, ja, so was ähnliches.

\Kathrine
Ich brauch keine Zeitung, ich kann sie nicht lesen, ich hab keine Augen. Ich hab meine Ohren.

\Josef
Fang nur nicht wieder an mit den alten Geschichten.

\Kathrine
Es sind gar keine alten Geschichten. Das weißt du sehr gut, davon spricht einjeder. Erst heut' nach der Kirche ---

\Josef
Jetzt schweig schon still, ich will nichts weiter mehr hören. Und überhaupt, wenn Barbara herunterkommt. Frauen in ihrem Zustand ---

\Kathrine
In ihrem Zustand, in ihrem Zustand. Wer hat sie denn in den Zustand gebracht. Das treibt's und vögelt und denkt dabei an die Folgen nicht weiter.

\Josef
Halts Maul, man wird noch sein Kind kriegen dürfen.

\Kathrine
Sein Kind kriegen dürfen. Barmherziger Himmel! Hast' denn Milch für dein Kind? Und reines Wasser? Und saubere Luft?

\Josef
Ja, ja, ja und noch viel, viel mehr.

\Kathrine
Mir ist mein Mädelchen an der Brust verhungert. Und meinen Buben haben sie mir aus dem Feld gebracht, ich hab ihn nimmer erkannt. Gott sei Dank, dass ich jetzt nicht mehr sehen brauch. Ihr aber, ihr müsst Kinder kriegen.

\Josef
Verflucht nochmal! \direction{springt auf} Das ist doch --- das ist doch zehn, zwölf, fünfzehn Jahre her. Wir haben keinen Krieg mehr. Hast du verstanden!

\Kathrine
Das sagen alle, aber es nicht wahr. Es ist eine schlechte Luft in der Welt. Wenn es auch nicht in der Zeitung steht.

\Josef
Kein Wort weiter. Barbara kommt.

\Barbara
\marginnote{schwanger}
\direction{stößt die Tür auf. Sie ist eine große Frau, stark in der Hoffnung}

Tag, Mutter Kathrine. Schön, dass du wiedermal zu uns gefunden hast, \direction{räkelt sich} Ach Gott, ach Gott, bin ich faul. Wie kann man nur am Nachmittag so schlafen.

\Kathrine
Das ist gut, das ist recht, das ist so am besten. Schlaf du nur. Kannst garnicht genug schlafen.

\Barbara Wie? Was meinst du?

\Josef
\direction{steht auf, ungeduldig, macht ein Zeichen an der Stirn} Lass sein, Barbara. Was ist mit unserem Kaffee?

\Barbara
Gleich, gleich, das Wasser ist schon aufgestellt. Aber hier ist ein Dampf. Zum Ersticken. Hast wie der einmal nichtschlecht gepafft. \direction{geht zum Fenster und stößt es auf}

\Kathrine
\direction{schnuppernd} Macht das Fenster zu, macht das Fenster zu.

\Barbara
Ach lass doch. Das bisschen frische Luft.

\Kathrine
Das ist nicht Luft, das ist Nebel.

\Barbara
\direction{beugt sich hinaus} Was für ein komischer gelber Nebel. Man sieht ja nichteinmal die Linde mehr.

\Josef
\direction{schließt das Fenster} Genug gelüftet, es kommt kalt herein. \direction{zeigt auf einenTisch} Und nimm dort doch die Kinderwäsche weg. Es werden sicherlich bald Gäste kommen.

\Barbara
\direction{legt die Wäsche zusammen} Eins, zwei, drei, vier, fünf, sechs Hemdchen. Nocheinmal sechs, dann sind zwei Dutzend voll. Die feinen Säumchen näht Agnes. Das Mädel hat wirklich unglaubliche Augen.

\Josef
Wo ist Agnes denn heute?

\Barbara
Sie ist schon früh morgens nach Dybern gegangen. Zu Annemarie. Die liegt immer noch krank. Agnes bringt ihr Kirschenkompott.

\Josef
Nach Dybern. Sag mal, du hast sie doch nicht allein gehen lassen?

\Barbara
Warum denn nicht?

\Josef
Es ist nur---ich meine---es ist scheussliches Wetter --- nass und kalt. Und dann plötzlich stock finster am hellichten Tag. \direction{dreht das Licht an}

\Barbara
\direction{ist inzwischen in die Küche gegangen, wo man sie, die Tür bleibt offen, herumhantieren sieht} Sie geht den Weg ja nicht zum ersten Mal.

\Josef
Wird sie denn nicht auf der Straße kommen?

\Barbara
Das glaub' ich kaum. Der Weidenweg ist doch viel näher. \direction{summt} \emph{Die Gänslein gehen barfuß und haben kein Schuh.}

\Josef
\direction{ist inzwischen auf und ab gegangen} Wann kommt sie denn zurück?

\Barbara
\direction{von der Küche her} Wie?

\Josef
Wann soll Agnes zurück sein?

\Barbara
\direction{kommt mit einem Tablett herein} Eh' es finster wird.

\direction{stellt den Kaffee vor \theKathrine} Da, Kathrine, greif zu.
\marginnote{er ist stark}
Im Kaffee ist viel Haut, und da hast du auch ein paar feine Kuchen.

\direction{sieht plötzlich erstaunt zum Fenster hin}
Ach, du meine Güte, es ist ja schon finster.

\Josef
Du hättest das Mädel doch nicht so allein hinauslassen sollen.

\Barbara
Sag mal, Josef, was hast du denn heute?

\Josef
Ach, gar nichts. Gib mir die Tasse her.

\Barbara
Da ist was los, du verschweigst mir Was.

\Josef
Ich verschweig' dir nichts. Ist ja alles nur dummes Gewäsch. Gut, dass du heute nicht auf dem Kirchplatz warst\ldots

\Barbara
Du, Josef, Jetzt will ich aber wirklich schon wissen ---

\Josef
Da gibts nichts zu wissen. Lass mich in Ruh'. Man wird ja selber schon ganz verblödet, so einen gottsverfluchten, nebligen Sonntag lang. Von Gerüchten darf man sich nicht in's Boxhorn jagen lassen, und überhaupt, eine Frau wie du, in deinem Zustand ---

\Barbara
Josef, jetzt mach mir nicht länger was vor. Ich merk' es doch die ganzen letzten Tage. Da wird immerfort nur geflüstert und getuschelt, und keiner schaut einem mehr grad ins Gesicht. Glaubst du, ich merk so was nicht? Ich spür' es schon auf der ganzen Haut.

\Kathrine
Es ist eine schlechte Luft in der Welt.

\Barbara
Wie? Was heißt das?

\Josef
Hör' nicht auf sie. Sie hat wieder ihren verrückten Tag. Man hat ein paar tote Rehe gefunden, zwischen den Weiden, und hinter Dybern, im Straßengraben auch noch eine verreckte Kuh.

\Barbara
Und?

\Josef
Und den Leuten wurde schlecht, als sie das Vieh so liegen sahen. Ein alter Bauer war es und sein Sohn. Müssen ja rechte Helden sein, die beiden.

\Barbara
Und?

\Josef
Und weiter nichts. Kannst das alles selbst in der Zeitung lesen. Dort steht auch von den unsinnigen Gerüchten.

\Barbara
Wenn du an die Gerüchte nichtglaubst, weshalb verschweigst du sie mir?

\Josef
Weißt du, Barbara, in deinem Zustand --- und seit du uns unlängst erst der Länge nach auf der Nase lagst wegen dem Försterhund

\Barbara
Wenn einem plötzlich mitten am Tag ein Hund in der Stube erschossen wird, und das Vieh liegt da und hat ganz blaue Augen da kann einem leicht bisschenschwindlig werden. ---

\Josef
Lass gut sein, Barbara, reg' dich nicht wieder auf.

\Barbara
Ach was, sprich nicht immer mit mir, als wär' ich jetzt nicht ganz bei Trost. Und was das Kleine ist, das liegt gut und sicher in meinem Bauch. Gib die Zeitung her. \direction{setzt sich mit der Zeitung an einen Tisch} \direction{\theJosef drückt auf den Knopf des Radios, leichte Schlagermusik}

\Barbara
\direction{hebt nach ein paar Sekunden plötzlich den Kopf und sagt sehr laut} Sag mal Josef, weiß man genau, dass der Rörsterhund wirklich die Tollwut hatte?

\Josef
\direction{zuckt die Achseln} --- Ich hab noch niemand gefragt.

\direction{die Tür wird aufgestoßen. Herein kommen \theGregor, \theJan, \theAndreas und \theLuise. \theGregor ist ein älterer Mann. \theJan ein spindeldümrer, zappliger Kerl, \theAndreas ein kräftiger schöner Bursche, \theLuise ein schlankes Mädchen, Typus der intelligenten Arbeiterin}

\Gregor
Oh, hier ist fein warm.

\Jan
Und immer Musik. Da gibts keine steifen Beine nicht. Komm, Luise. \direction{legt den Armum ihre Hüften}

\Luise
Lass los! Grüß Gott, Barbara.

\Gregor
\direction{zu \theAndreas, der als letzter kommt} Mach die Tür zu, Andreas.

\Luise
Mein Gott, der Nebel, das ist ja schon wie Rauch.

\Jan
Aber hier ist es gleich. Hier sind wir fidel. \direction{johlt zur Musik} \emph{Denn das Wirtshaus am Rand, das ist ja bekannt im ganzen Land.}

\Gregor
\direction{zu \theJosef} Einen Scharfen, der einheizt!

\Josef
\direction{in dem er ihm einschenkt} Der Jan kriegt nichts. Der ist ja jetzt schon besoffen. \direction{stellt das Radio ab}

\Jan
Oho.

\Josef
Woher kommt ihr denn?

\Gregor
Von dem neuen Kino. Dort gibt es jetzteinen großen Ausschank.

\Josef
Spielt es denn schon?

\Gregor
Nein, noch nicht, aber man kann es sich ansehen. Was rennst du denn so herum, Andreas? Was suchst du denn?

\Jan
Na der, der sucht doch natürlich die Agnes.

\Andreas
So schweig schon einmal.

\Jan
Frau Wirtin, wo ist denn das Fräulein Schwester? Das entzückende, das reizende Fräuleinchen Schwesterchen?

\Barbara
\direction{sieht verwirrt von der Zeitung auf} Agnes?

\Josef
Sie wird gleich kommen, sie war in Dybern, ein Krankenbesuch. Sie muss jeden Augenblick da sein. Aber erzählt doch lieber, wie ist das Kino?

\Jan
Fein, pickfein, viel zu fein für uns arme Leute. Wenn du mal erst die Marmortreppe runtersteigst ---

\Josef
Ist es wahr, dass es ganz unter der Erde ist?

\Jan
Ganz unter der Erde. Das ist jetzt das Neueste, das Modernste. Das ist das Schönste und das Gesündeste und das Billigste. Einen eigenen Architekten haben sie sich dazu herbestellt. Dicht an der Fabrik ist es auch. Man braucht sich nachder Arbeit bloß die Hände waschen ---

\Gregor
\direction{schlägt auf den Tisch} Und jetzt sagmir nur, was dir schon wieder daran nicht recht ist?

\Josef
Der Jan ist bös', wenn er nicht Grund genug zum Stänkern hat.

\Jan
Und ihr seid alle miteinand' Idioten. Wenn euch die hohe Direktion mal ein Zuckerstück hinhält, dann schnappt ihr danach. Sonst kann sie getrost auf eure Köpfespucken.

\Gregor
Es spuckt keiner auf unsere Köpfe. Und wenn sie uns ein Kino hinstellen, so verdienen sie schließlich selber daran.

\Luise
\direction{nachdenklich} Das muss ungeheuer viel gekostet haben, so ein Riesenkino ganz unter der Erde.

\Jan
Jetzt sag mir nur einer, warum ist esdenn ganz unter der Erde?

\Gregor
Damit du das Gras wachsen hörst, du Rotzbub. Das tust du ja ohnehin so gern.

\Andreas
Ich versteh aber auch nicht, warum sie es so hinunter bauen.

\Gregor
Jetzt fängst du auch an.

\Jan
Wir sind doch keine Maulwürfe.

\Gregor
Brauchst ja nicht runter, wenn es dirnicht passt.

\Kathrine
Mein Bub war auch in so einem Kino unter der Erde. Es war sehr nass.

\Josef
Schon gut, schon gut, Kathrine, sprich da nicht mit.

\Kathrine
Ihr werdet alle noch hinunter müssen. Damals hat auch ein jeder geglaubt, es trifft nur den andern.

\Josef
Aber es ist doch ein Kino, Kathrine, ein Kino.

\Jan
Schrei nicht so, sie ist ja nicht taub, nur blind. \direction{greift nach \theGregor{}s Glas}

\Gregor
\direction{hält ihn zurück} Lass sein, bist jaohnehin schon besoffen.

\Jan
\direction{reißt ihm das Glas aus der Hand und trinkt es aus} --- Natürlich bin ich besoffen, ich bin ja immer besoffen, und wenn mir mal die Luft ausgeht in unserer alten Stinkbude drüben, dann bin ich auch nur besoffen, fragt doch den Doktor Thomsen, der hat das gesagt, und wenn ein Mädel umfällt, mitten in der Arbeit, dann bin ich auch nur besoffen, und wenn einer die gewissen Flecken kriegt, die blauen Flecken, erst an den Händen, dann in den Augen ---

\Barbara
\direction{die bisher mit der Zeitung vor sichwie teilnahmslos gesessen ist} Was für Flecken, was für blaue Flecken?

\Luise
Halts Maul, Jan, du redst dich noch umdeinen Kopf.

\Gregor
Pack dein Bündel und geh', hat dich ja keiner nicht hergebeten, bist doch so nur ein Fremder. Geh du dort hin, wo es dem Arbeiter wirklich schlecht geht, wo er kein Fressen hat, kein Dach überm Kopf. Dort kannst du deine Reden halten, Gefahren gibts in jeder Fabrik.

\Josef
Und in unserer Gegend sind die besten Löhne. Das weiß ein jeder. Es liegt ein Segen über dem ganzen Land. Blumen hinter allen Fensterscheiben. Wenn ich denke, wie es früher gewesen ist. Nichts auf's Brot haben die Leute gehabt.

\Barbara
\direction{steht auf} --- Sagt mal, was sind denn das für blaue Flecken?

\Gregor
Nur aus Unvorsichtigkeit. Ihr könnt euch drauf verlassen, immer nur aus Unvorsichtigkeit. Was predige ich nicht täglich den Leuten. Eine Stickstoffabrik ist schließlich kein Kinderzimmer.

\Luise
Nein, Gregor, das ist eine Gemeinheit, was du sagst. Und außerdem bestimmt nicht wahr. Es tut nicht gut, wenn einer von uns so spricht.

\Gregor
Dir ist natürlich lieber, wenn einer hetzt, wie dein Jan. Siehst' dich wohl auch schon Versammlungen halten. Und was kommt dabei raus --- nichts als Not und Elend. Schau' du lieber, dass du einen braven Mann kriegst und ein paar Kinder und ein Häuschen in unserer Siedlung.

\Andreas
\direction{sieht auf die Uhr} Wann soll die Agnes von Dybern zurück sein?

\Josef
\direction{sieht zum Fenster hinaus} Es ist stock finster. Vielleicht sollte man ihr entgegengehen.

\Luise
Aber die Straße ist doch gut beleuchtet.

\Barbara
Sie kommt auf dem Weidenweg.

\Andreas
Auf dem Weidenweg!

\Jan
Herr des Himmels, was schickt ihr sie denn auf den Weidenweg!

\Gregor
Auf dem Weidenwog ist der Nebel am schlimmsten. Dort steigt er auf vom Fluss.

\Barbara
\direction{sehr heftig} Ja seid ihr denn alle verrückt geworden. Das Mädel hat doch eine Taschenlampe bei sich. Die geht den Weg zum hundertsten Mal.

\Luise
Aber der Nebel.

\Barbara
Am Nebel ist noch keiner gestorben.

\Andreas
\direction{steht auf} Gebt mir eine Lampe. Ich geh' ihr entgegen.

\Josef
\direction{geht eine Laterne holen}

\Andreas
Macht rasch, rasch --- Agnes --- mein Gott Agnes es wird ihr doch nichts geschehen sein. \direction{reißt \theJosef die Laterne aus der Hand und stürzt hinaus.}

\Barbara
\direction{ steht mitten im Zimmer und zählt während des Sprechens an den Fingern}
Ein paar Rehe, eine Kuh, einem Bauern ist schlecht geworden und seinem Sohn. In der Zeitung steht, das sind nur Gerüchte, in der Zeitung steht, es istnicht der Nebel, in der Zeitung steht, man kann ja auch im Wald krank werden und auf offenem Feld, und Tiere sterben eben --- aber der Hund --- sag mal, Gregor, hat der Hund wirklich die Tollwut gehabt?

\Gregor
Was weiß denn ich.

\Barbara
Ja, warum wisst ihr das denn alle nicht?

\Luise
Es ist ein ungesunder Herbst, Barbara. Und wenn der Nebel einmal so dick wird wie eine nasse Mauer ---

\Josef
Sollst dir nicht so viel den Kopf zerbrechen, Barbara. Gesunden Lungen macht der Nebel nichts. Nur wenn einer ohnehin Asthma hat oder sonstwie schweren Atem

\Jan
Und wo doch die Hauptsache ist, dass die Bevölkerung ruhig bleibt, immer nur ruhig, wie es an unserer Kirche angeschlagen steht, und der Herr Pfarrer hat ja auch gepredigt, dass Gott es gar nicht so böse meint. Man braucht doch nicht in den Wald zu gehen und zum Fluss

\Josef
Halts Maul, Kerl, oder ich schmeiß' dich hinaus.

\Barbara
\direction{sinkt plötzlich auf einem Stuhl zusammen} Agnes --- o Gott, warum habt ihrmir das nicht früher gesagt.

\Luise
Aber, Barbara, es ist doch nichts geschehen.

\Josef
Das kommt nur von den verfluchten Gerüchten.

\Gregor
Im Nebel kann doch keine Krankheit sein.

\Jan
\direction{steht auf} Ich geh dem Andreas nach.

\Barbara
Was ist denn aus dem Bauern gewordenund seinem Sohn?

\direction{\theJosef und \theGregor zucken die Achseln}

\Barbara
Ja, warum wisst ihr das denn alle nicht?

\direction{man hört den Motor eines Autos und hupen. \theJosef geht hinaus und kommt gleich darauf mit Doktor \theThomsen zurück. \theThomsen ist ein älterer, robuster Mann}

\Thomsen
\direction{schüttelt sich vor Nässe} \ldots Guten Abend.

\Gregor
Guten Abend.

\Luise
Gott sei Dank, der Doktor.

\Thomsen
Wieso? Was ist denn?

\Josef
Es ist keine Ruhe im Land, Herr Doktor

\Thomsen
Ja, ja, schon gut. Gebt mir rasch cinen starken Grog, und dann hilft mir einer bei meiner Panne. \direction{setzt sich mürrisch in eine Ecke}

\Josef
\direction{aus der Küche heraus} Wollen Herr Doktor auch was essen?

\Thomsen
Nein.

\Barbara
Woher kommt denn der Doktor?

\Thomsen
Aus Dybern.

\Luise
Von einem Kranken?

\Thomsen
Nein.

\Gregor
Von sonst einem Besuch?

\Thomsen
Nein.

\Barbara
\direction{steht auf, stellt sich vor \theThomsen, voll Angst} woher denn sonst?

\Thomsen
Von einem Toten.

\direction{Stille}

\Thomsen
Von zwei Toten, wenn ihr es wissen wollt.

\Barbara
Der Bauer und sein Sohn. \direction{sinkt wieder in ihrem Stuhl zusammen}

\Josef
\direction{indem er den Grog bringt} \ldots Asthma, Herr Doktor, nicht wahr, ein schlimmer Atem.

\Luise
\direction{sehr klar} Ist es wahr, dass eine Seuche im Nebel steckt? Dass man krank wird im Wald?

\Thomsen
Es scheint so zu sein.

\Luise
Warum sperrt man dann den Wald nicht ab?

\Thomsen
Weil man nicht wissen kann, wie der Wind sich dreht.

\Gregor
Und woher kommt die Krankheit so plötzlich?

\Thomsen
Das weiß Gott allein.

\Kathrine
Du sollst den Namen Gottes nicht eitel nennen.

\Thomsen
\direction{fährt zusammen} Was ist das?

\Josef
Entschuldigen, Herr Doktor, es ist nur unsere alte Kathrine.

\direction{macht wieder das Zeichen an der Stirn. Zu \theBarbara, die plötzlich ein Tuch vom Haken reißt undzur Tür geht}

Was ist, wohin willst du?

\Barbara
Ich geh dem Mädel entgegen. \direction{stockt, denn man hört ein paar schrille Pfiffe}

\Luise
\direction{springt auf} Das ist Jan, das ist seine Pfeife.

\Andreas
\direction{Stimme}
\ldots Macht auf, macht rasch auf.

\direction{\theJosef reisst die Tür auf, \theJan und \theAndreas tragen \theAgnes herein. \theAgnes, ein fünfzehnjähriges zartes Mädchen, gelb im Gesicht, mit entsetzlichen Augen}

\Agnes
Wasser --- Wasser

\Thomsen
\direction{wirft seinen Mantel auf eine Bank} \ldots legt sie her, sofort.

\Agnes
Wasser \direction{mit den Händen an der Brust} Ich brenne.

\Luise
\direction{hält ihr ein Glas Wasser hin} Hier.

\Agnes
\direction{trinkt} Ich brenne, ich brenne.

\Josef
Herr Doktor, Herr Doktor!

\Barbara
\direction{schiebt \theAgnes ein Tuch unter denKopf} Agnes.

\Andreas
\direction{wirft sich neben ihr hin} Agnes.

\Agnes
Ich brenne.

\Thomsen
Nicht weiter trinken. \direction{er reißt ihrdas Glas aus der Hand}

\Agnes
Das Wasser brennt. \direction{sinkt einen Augenblick zurück}

\Kathrine
\direction{sie ist bis jetzt im Hintergrundan der anderen Seite des Zimmers gesessen. Nun geht sie, den Stock gehoben, in der Richtung von \theAgnes wie gezogen auf das Mädchen zu} \ldots Es riecht nach Senf.

\Agnes
Ich verbrenne.

\end{play}

\scene{Kriesensitzung}
\scene{Im Flüchtlingslager}
\scene{Hungrige Mäuler}
\scene{Nachbohren}

\act{Zeiter Akt}
\scene{Irrungen}
\scene{Zusammenbruch}

\end{document}
