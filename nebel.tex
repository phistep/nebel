% KOMA-Script layout settings
\documentclass[
	final,
	a4paper,
	ngerman,
	mpinclude = true, % include marginpar in textwidth for headsepline
	twoside = true,
	open = right,
	cleardoublepage = plain,
	DIV = 13,
	BCOR = 1cm,
	titlepage = firstiscover,
	]{scrbook}

\usepackage[T1]{fontenc}
\usepackage[utf8]{inputenc}
\usepackage[utf8]{luainputenc}

\usepackage{xspace}
\usepackage{calc} % \widthof

% margin notes
\setlength{\marginparwidth}{1.8\marginparwidth}
\setlength{\marginparsep}{10mm} % line numbers
\newcommand{\marginnote}[1]{\marginpar{\singlespacing\raggedright\footnotesize#1}}

% custom formatting for acts and scenes
\addtokomafont{sectioning}{\rmfamily\scshape\mdseries\centering}
\newcommand{\act}{\chapter}
\renewcommand*{\raggedsection}{\centering}
\renewcommand*{\chapterformat}{}
\renewcommand*{\chaptermarkformat}{}
\newcommand{\scene}{\setcounter{subscene}{1}\section}
\RedeclareSectionCommand[style=chapter]{section}
\renewcommand*{\sectionformat}{Szene \thesection~— }
\renewcommand*{\sectionmarkformat}{Szene \thesection~— }
\newcommand{\direction}[1]{(\textit{#1})}
\newcommand{\setting}[1]{\vspace{-0.5\baselineskip}\centering\textit{#1}}
\newcommand{\hiat}{%
	\begin{center}
		\tiny
		\raisebox{0.5ex}{\rule{0.3\linewidth}{0.4pt}}
		\textit{fickstrich}
		\raisebox{0.5ex}{\rule{0.3\linewidth}{0.4pt}}
	\end{center}
}
\newcommand{\appearances}[1]{\quad {\footnotesize #1}}
% TODO refstepcounter
\newcounter{subscene}
\setcounter{subscene}{1}
\newcommand{\subscene}{\marginnote{Szene \arabic{chapter}.\arabic{section}.\alph{subscene}}\stepcounter{subscene}}

% "Elements of Typographic Style" table of contents
\DeclareTOCStyleEntry[
		raggedpagenumber=true,
		linefill = {},
		entrynumberformat = {\phantom},
		indent = 0cm,
	]{tocline}{chapter}
\newlength{\scenenumwidth}
\setlength{\scenenumwidth}{\widthof{Szene 8.88 }}
\DeclareTOCStyleEntry[
		raggedpagenumber = true,
		linefill = {},
		indent = 0.3\linewidth,
		entrynumberformat = {{\footnotesize{\textsc{Szene}}}\enspace},
		numwidth = \scenenumwidth,
	]{tocline}{section}

% header
\usepackage[
		automark,
		headsepline,
		headwidth=textwithmarginpar,
	]{scrlayer-scrpage}
	\pagestyle{scrheadings}
	\ihead{}
	\chead{}
	\ohead{}
	\cehead{\leftmark}
	\cohead{\rightmark}

\usepackage[utf8]{inputenc}
\usepackage{babel}

% typography
\usepackage{ebgaramond}
\usepackage{microtype}
\usepackage{setspace} % one-half spacing
\usepackage[modulo,running]{lineno} % line numbers
	%\renewcommand{\thelinenumber}{\thesection.\arabic{linenumber}}
	\renewcommand{\thelinenumber}{\arabic{section}.\arabic{linenumber}}
\usepackage{csquotes}
\usepackage{siunitx}

% special version for directors
\usepackage{substr}
\newcommand{\ifdirectorsversion}[2]{%
	\IfSubStringInString{\detokenize{regie}}{\jobname}{#1}{#2}
}

% two column layout for character names and lines
\usepackage{enumitem}
\newlist{play}{description}{1}
\newlength{\widthofchar}
\setlength{\widthofchar}{\widthof{\textsc{Schwester\quad}}}
\setlist[play]{
	labelwidth=\widthofchar,
	leftmargin=!,
	font=\rmfamily\mdseries\scshape,
	itemsep=0pt,
	before={\linenumbers*}
}

% gray out deletions
\usepackage{xcolor}
\usepackage{comment}
\ifdirectorsversion{%
	\newenvironment{deletion}{%
		\vspace{0.25\baselineskip}
		\hrule
		\vspace{0.25\baselineskip}
		\color{darkgray}
	}{
		\color{black}
		\vspace{0.25\baselineskip}
		\hrule
		\vspace{0.25\baselineskip}
	}
}{%
	\excludecomment{deletion}
}

% list of characters at the beginning of a scene
\newcommand{\characterlist}[1]{{\begin{center}\textit{Personen:}\quad{}#1\end{center}}}

% PDF options
\usepackage[final,hidelinks]{hyperref}
	\hypersetup{
		unicode     = true,
		linktoc     = all,
		pdftitle    = {Der Nebel von Dybern},
		pdfauthor   = {Maria Lazar},
		pdfsubject  = {Ein Drama},
		pdflang     = de-DE,
		pdfdisplaydoctitle = true,
	}
	\ifdirectorsversion{\hypersetup{pdftitle={Der Nebel von Dybern (Regie-Version)}}}{}
	\addto\extrasngerman{
		\renewcommand{\chapterautorefname}{Akt}
		\renewcommand{\sectionautorefname}{Szene}
	}
\usepackage{bookmark} % toc in PDF bookmarks

% shortcuts for characters
% within line
\newcommand{\thecharacter}[1]{\textup{\textsc{#1}}\xspace}
\newcommand{\theBarbara}{\thecharacter{Barbara}}
\newcommand{\theJosef}{\thecharacter{Josef}}
\newcommand{\theKathrine}{\thecharacter{Kathrine}}
\newcommand{\theGregor}{\thecharacter{Gregor}}
\newcommand{\theJan}{\thecharacter{Jan}}
\newcommand{\theAndreas}{\thecharacter{Andreas}}
\newcommand{\theLuise}{\thecharacter{Luise}}
\newcommand{\theAgnes}{\thecharacter{Agnes}}
\newcommand{\theGeneraldirektor}{\thecharacter{Generaldirektor}}
\newcommand{\theClarisse}{\thecharacter{Clarisse}}
\newcommand{\theAlexis}{\thecharacter{Alexis}}
\newcommand{\theThomsen}{\thecharacter{Thomsen}}
\newcommand{\theJonas}{\thecharacter{Jonas}}
\newcommand{\theSalwin}{\thecharacter{Salwin}}
\newcommand{\theBrix}{\thecharacter{Oberst~Brix}}
\newcommand{\theMelchior}{\thecharacter{Melchior}}
\newcommand{\theHeilsarmeeschwester}{\thecharacter{Heilsarmeeschwester}}
\newcommand{\theErsterMann}{\thecharacter{Erster~Mann}}
\newcommand{\theZweiterMann}{\thecharacter{Zweiter~Mann}}
\newcommand{\theSergeant}{\thecharacter{Sergeant}}
\newcommand{\theDiener}{\thecharacter{Diener}}
\newcommand{\theKinder}{\thecharacter{Kinder}}
\newcommand{\theLeute}{\thecharacter{Leute}}
\newcommand{\theSoldaten}{\thecharacter{Soldaten}}

% speaker of line
\newcommand{\character}[1]{\item[#1]}
\newcommand{\Barbara}{\character{\theBarbara}}
\newcommand{\Josef}{\character{\theJosef}}
\newcommand{\Kathrine}{\character{\theKathrine}}
\newcommand{\Gregor}{\character{\theGregor}}
\newcommand{\Jan}{\character{\theJan}}
\newcommand{\Andreas}{\character{\theAndreas}}
\newcommand{\Luise}{\character{\theLuise}}
\newcommand{\Agnes}{\character{\theAgnes}}
\newcommand{\Generaldirektor}{\character{Direktor}}
\newcommand{\Clarisse}{\character{\theClarisse}}
\newcommand{\Alexis}{\character{\theAlexis}}
\newcommand{\Thomsen}{\character{\theThomsen}}
\newcommand{\Jonas}{\character{\theJonas}}
\newcommand{\Salwin}{\character{\theSalwin}}
\newcommand{\Brix}{\character{\theBrix}}
\newcommand{\Melchior}{\character{\theMelchior}}
\newcommand{\Heilsarmeeschwester}{\character{Schwester}}
\newcommand{\ErsterMann}{\character{1. Mann}}
\newcommand{\ZweiterMann}{\character{2. Mann}}
\newcommand{\Sergeant}{\character{\theSergeant}}
\newcommand{\Diener}{\character{\theDiener}}
\newcommand{\Kinder}{\character{\theKinder}}
\newcommand{\Leute}{\character{\theLeute}}
\newcommand{\Soldaten}{\character{\theSoldaten}}
\newcommand{\Stimme}{\character{\emph{Stimme}}}
\newcommand{\Junge}[1]{\character{Junge #1}}
\newcommand{\Maedchen}[1]{\character{Mädchen #1}}

% cover
\usepackage{pdfpages}

% title page
\addtokomafont{titlehead}{\scshape\lsstyle}
\titlehead{\centering Wery Important Production Berlin}
\title{Der Nebel von Dybern}
\subtitle{Ein Drama}
\author{Maria Lazar}
\date{\ifdirectorsversion{-- Regie-Version --}{}}
\publishers{S. Fischer Verlag}
\uppertitleback{%
	\centering
	\addsec*{Dramatis Person\ae}
	\vspace{\baselineskip}
	\raggedright
    \theBarbara, \quad schwanger%
		\appearances{\ref{scene:I}, \ref{scene:IV}, \ref{scene:VI}.a, \ref{scene:VII}}\\
	\theJosef, \quad ihr Mann,%
		\appearances{\ref{scene:I}, \ref{scene:IV}, \ref{scene:VI}.b, \ref{scene:VII}}\\
    \theKathrine, \quad blind%
		\appearances{\ref{scene:I}, \ref{scene:IV}, \ref{scene:VII}}\\
    \theGregor%
		\appearances{\ref{scene:I}, \ref{scene:III}, \ref{scene:VI}.b}\\
    \theJan%
		\appearances{\ref{scene:I}, \ref{scene:III}, \ref{scene:VII}}\\
    \theAndreas%
		\appearances{\ref{scene:I}, \ref{scene:IV}}\\
    \theLuise%
		\appearances{\ref{scene:I}, \ref{scene:VI}.a}\\
    \theAgnes%
		\appearances{\ref{scene:I}}\\
    Paul, der \theGeneraldirektor der Chemiefrabrik%
		\appearances{\ref{scene:II}, \ref{scene:III}, \ref{scene:IV}, \ref{scene:V}, \ref{scene:VI}.c}\\
    \theClarisse, \quad seine Frau%
		\appearances{\ref{scene:II}, \ref{scene:V}, \ref{scene:VI}.c}\\
    \theAlexis, \quad Ingenieur%
		\appearances{\ref{scene:II}, \ref{scene:III}, \ref{scene:IV}, \ref{scene:V}, \ref{scene:VI}.c}\\
    Doktor \theThomsen, \quad Arzt%
		\appearances{\ref{scene:I}, \ref{scene:II}, \ref{scene:V}}\\
	Doktor \theJonas, \quad Arzt%
		\appearances{\ref{scene:II}, \ref{scene:V}, \ref{scene:IV}, \ref{scene:VI}.d}\\
    \theSalwin, \quad Pressekorrespondent%
		\appearances{\ref{scene:II}, \ref{scene:VI}.e}\\
    \theBrix, \quad Militär, Chemiker%
		\appearances{\ref{scene:IV}, \ref{scene:VI}.f}\\
    Jakob \theMelchior, \quad ehem. Militär, sein Vertrauter%
		\appearances{\ref{scene:IV}, \ref{scene:V}, \ref{scene:VI}.f}\\
    \theHeilsarmeeschwester%
		\appearances{\ref{scene:III}, \ref{scene:IV}, \ref{scene:IV}.e, \ref{scene:VI}}\\
    \theErsterMann%
		\appearances{\ref{scene:III}, \ref{scene:VI}.d}\\
    \theZweiterMann%
		\appearances{\ref{scene:III}, \ref{scene:VI}.d}\\
    \theSergeant%
		\appearances{\ref{scene:VII}}\\
    \theDiener%
		\appearances{\ref{scene:II}, \ref{scene:V}}\\
    \theKinder%
		\appearances{\ref{scene:IV}}\\
	\theLeute%
		\appearances{\ref{scene:III}}\\
	\theSoldaten%
		\appearances{\ref{scene:IV}, \ref{scene:VII}}
}
\lowertitleback{%
	\footnotesize
	\centering
	Version vom \today.\\
	\vspace{0.5\baselineskip}
	Erschienen 1932 bei \textsc{S. Fischer, Verlag A.G. Berlin}.\\
	\vspace{0.5\baselineskip}
	Gesetzt mit \LaTeX{} und \KOMAScript{} in EBGaramond.\\
}

\begin{document}
\pagenumbering{alph}
%\includepdf{cover/cover}
\cleardoubleoddemptypage

\pagenumbering{roman}
\maketitle

\pdfbookmark[chapter]{\contentsname}{toc}
\tableofcontents
\cleardoubleoddpage

\pagestyle{headings}
\pagenumbering{arabic}
\doublespacing

\act{Erster Akt}
\scene{Der Nebel}
\label{scene:I}
\characterlist{\theBarbara, \theJosef, \theKathrine, \theGregor, \theJan, \theAndreas, \theThomsen, \theAgnes}
\setting{Eine einfache, saubere Wirtsstube. Frühe Nachmittagsdämmerung, schläfriges Licht. Auf der Fensterbank sitzt \theJosef mit einer Zeitung, an den Kachelofen gelennt, hockt die alte \theKathrine. \theJosef ist ein behäbiger, etwas dicker Mensch. \theKathrine ist blind. Sie starrt immer vor sich hin, als ob sie etwas sehen würde.}

\begin{play}

\Barbara
\direction{Eine tiefe volle Frauenstimme singt} \emph{\ldots Eia popeia, was raschelt im Stroh ---}

\Josef
\direction{hebt den Kopf gegen die Decke} Barbara

\Barbara
\direction{singend} \emph{Ja ---}

\Josef
Wenn du schon wach bist, dann komm doch herunter. Wir wollen Kaffee.

\Barbara
Ist jemand da?

\Josef
Mutter Katharine.

\Barbara
\direction{weiter singend} \emph{.. das sind die lieben Gänslein, die haben kein' Schuh, der Schuster hat's Leder, kein' Leisten dazu.} \direction{Stimme verklingt}

\Kathrine
Lass sie in Ruh. Sie kommt immernoch früh genug. Sie kommt viel zu früh,und wir brauchen keinen Kaffee.

\Josef
Ja, ja, schon gut. \direction{blättert in der Zeitung}

\Kathrine
Was steht denn da in der Zeitung drin? Du liest doch die Zeitung. Schöne Geschichten, Sonntagsgeschichten?

\Josef
Ja, ja, so was ähnliches.

\Kathrine
Ich brauch keine Zeitung, ich kann sie nicht lesen, ich hab keine Augen. Ich hab meine Ohren.

\Josef
Fang nur nicht wieder an mit den alten Geschichten.

\Kathrine
Es sind gar keine alten Geschichten. Das weißt du sehr gut, davon spricht einjeder. Erst heut' nach der Kirche ---

\Josef
Jetzt schweig schon still, ich will nichts weiter mehr hören. Und überhaupt, wenn Barbara herunterkommt. Frauen in ihrem Zustand ---

\Kathrine
In ihrem Zustand, in ihrem Zustand. Wer hat sie denn in den Zustand gebracht. Das treibt's und vögelt und denkt dabei an die Folgen nicht weiter.

\Josef
Halts Maul, man wird noch sein Kind kriegen dürfen.

\Kathrine
Sein Kind kriegen dürfen. Barmherziger Himmel! Hast' denn Milch für dein Kind? Und reines Wasser? Und saubere Luft?

\Josef
Ja, ja, ja und noch viel, viel mehr.

\Kathrine
Mir ist mein Mädelchen an der Brust verhungert. Und meinen Buben haben sie mir aus dem Feld gebracht, ich hab ihn nimmer erkannt. Gott sei Dank, dass ich jetzt nicht mehr sehen brauch. Ihr aber, ihr müsst Kinder kriegen.

\Josef
Verflucht nochmal! \direction{springt auf} Das ist doch --- das ist doch zehn, zwölf, fünfzehn Jahre her. Wir haben keinen Krieg mehr. Hast du verstanden!

\Kathrine
Das sagen alle, aber es nicht wahr. Es ist eine schlechte Luft in der Welt. Wenn es auch nicht in der Zeitung steht.

\Josef
Kein Wort weiter. Barbara kommt.

\Barbara
\marginnote{schwanger}
\direction{stößt die Tür auf. Sie ist eine große Frau, stark in der Hoffnung}

Tag, Mutter Kathrine. Schön, dass du wiedermal zu uns gefunden hast, \direction{räkelt sich} Ach Gott, ach Gott, bin ich faul. Wie kann man nur am Nachmittag so schlafen.

\Kathrine
Das ist gut, das ist recht, das ist so am besten. Schlaf du nur. Kannst garnicht genug schlafen.

\Barbara Wie? Was meinst du?

\Josef
\direction{steht auf, ungeduldig, macht ein Zeichen an der Stirn} Lass sein, Barbara. Was ist mit unserem Kaffee?

\Barbara
Gleich, gleich, das Wasser ist schon aufgestellt. Aber hier ist ein Dampf. Zum Ersticken. Hast wie der einmal nichtschlecht gepafft. \direction{geht zum Fenster und stößt es auf}

\Kathrine
\direction{schnuppernd} Macht das Fenster zu, macht das Fenster zu.

\Barbara
Ach lass doch. Das bisschen frische Luft.

\Kathrine
Das ist nicht Luft, das ist Nebel.

\Barbara
\direction{beugt sich hinaus} Was für ein komischer gelber Nebel. Man sieht ja nichteinmal die Linde mehr.

\Josef
\direction{schließt das Fenster} Genug gelüftet, es kommt kalt herein. \direction{zeigt auf einenTisch} Und nimm dort doch die Kinderwäsche weg. Es werden sicherlich bald Gäste kommen.

\Barbara
\direction{legt die Wäsche zusammen} Eins, zwei, drei, vier, fünf, sechs Hemdchen. Nocheinmal sechs, dann sind zwei Dutzend voll. Die feinen Säumchen näht Agnes. Das Mädel hat wirklich unglaubliche Augen.

\Josef
Wo ist Agnes denn heute?

\Barbara
\marginnote{Unweit der belgischen Stadt Ypern war im April 1915 von deutscher Seite erstmals Giftgas eingesetzt worden.}
Sie ist schon früh morgens nach Dybern gegangen. Zu Annemarie. Die liegt immer noch krank. Agnes bringt ihr Kirschenkompott.

\Josef
Nach Dybern. Sag mal, du hast sie doch nicht allein gehen lassen?

\Barbara
Warum denn nicht?

\Josef
Es ist nur---ich meine---es ist scheussliches Wetter --- nass und kalt. Und dann plötzlich stock finster am hellichten Tag. \direction{dreht das Licht an}

\Barbara
\direction{ist inzwischen in die Küche gegangen, wo man sie, die Tür bleibt offen, herumhantieren sieht} Sie geht den Weg ja nicht zum ersten Mal.

\Josef
Wird sie denn nicht auf der Straße kommen?

\Barbara
Das glaub' ich kaum. Der Weidenweg ist doch viel näher. \direction{summt} \emph{Die Gänslein gehen barfuß und haben kein Schuh.}

\Josef
\direction{ist inzwischen auf und ab gegangen} Wann kommt sie denn zurück?

\Barbara
\direction{von der Küche her} Wie?

\Josef
Wann soll Agnes zurück sein?

\Barbara
\direction{kommt mit einem Tablett herein} Eh' es finster wird.

\direction{stellt den Kaffee vor \theKathrine} Da, Kathrine, greif zu.
\marginnote{er ist stark}
Im Kaffee ist viel Haut, und da hast du auch ein paar feine Kuchen.

\direction{sieht plötzlich erstaunt zum Fenster hin}
Ach, du meine Güte, es ist ja schon finster.

\Josef
Du hättest das Mädel doch nicht so allein hinauslassen sollen.

\Barbara
Sag mal, Josef, was hast du denn heute?

\Josef
Ach, gar nichts. Gib mir die Tasse her.

\Barbara
Da ist was los, du verschweigst mir Was.

\Josef
Ich verschweig' dir nichts. Ist ja alles nur dummes Gewäsch. Gut, dass du heute nicht auf dem Kirchplatz warst\ldots

\Barbara
Du, Josef, Jetzt will ich aber wirklich schon wissen ---

\Josef
Da gibts nichts zu wissen. Lass mich in Ruh'. Man wird ja selber schon ganz verblödet, so einen gottsverfluchten, nebligen Sonntag lang. Von Gerüchten darf man sich nicht in's Boxhorn jagen lassen, und überhaupt, eine Frau wie du, in deinem Zustand ---

\Barbara
Josef, jetzt mach mir nicht länger was vor. Ich merk' es doch die ganzen letzten Tage. Da wird immerfort nur geflüstert und getuschelt, und keiner schaut einem mehr grad ins Gesicht. Glaubst du, ich merk so was nicht? Ich spür' es schon auf der ganzen Haut.

\Kathrine
Es ist eine schlechte Luft in der Welt.

\Barbara
Wie? Was heißt das?

\Josef
Hör' nicht auf sie. Sie hat wieder ihren verrückten Tag. Man hat ein paar tote Rehe gefunden, zwischen den Weiden, und hinter Dybern, im Straßengraben auch noch eine verreckte Kuh.

\Barbara
Und?

\Josef
Und den Leuten wurde schlecht, als sie das Vieh so liegen sahen. Ein alter Bauer war es und sein Sohn. Müssen ja rechte Helden sein, die beiden.

\Barbara
Und?

\Josef
Und weiter nichts. Kannst das alles selbst in der Zeitung lesen. Dort steht auch von den unsinnigen Gerüchten.

\Barbara
Wenn du an die Gerüchte nichtglaubst, weshalb verschweigst du sie mir?

\Josef
Weißt du, Barbara, in deinem Zustand --- und seit du uns unlängst erst der Länge nach auf der Nase lagst wegen dem Försterhund

\Barbara
Wenn einem plötzlich mitten am Tag ein Hund in der Stube erschossen wird, und das Vieh liegt da und hat ganz blaue Augen da kann einem leicht bisschenschwindlig werden. ---

\Josef
Lass gut sein, Barbara, reg' dich nicht wieder auf.

\Barbara
Ach was, sprich nicht immer mit mir, als wär' ich jetzt nicht ganz bei Trost. Und was das Kleine ist, das liegt gut und sicher in meinem Bauch. Gib die Zeitung her. \direction{setzt sich mit der Zeitung an einen Tisch} \direction{\theJosef drückt auf den Knopf des Radios, leichte Schlagermusik}

\Barbara
\direction{hebt nach ein paar Sekunden plötzlich den Kopf und sagt sehr laut} Sag mal Josef, weiß man genau, dass der Rörsterhund wirklich die Tollwut hatte?

\Josef
\direction{zuckt die Achseln} --- Ich hab noch niemand gefragt.

\direction{die Tür wird aufgestoßen. Herein kommen \theGregor, \theJan, \theAndreas und \theLuise. \theGregor ist ein älterer Mann. \theJan ein spindeldümrer, zappliger Kerl, \theAndreas ein kräftiger schöner Bursche, \theLuise ein schlankes Mädchen, Typus der intelligenten Arbeiterin}

\Gregor
Oh, hier ist fein warm.

\Jan
Und immer Musik. Da gibts keine steifen Beine nicht. Komm, Luise. \direction{legt den Armum ihre Hüften}

\Luise
Lass los! Grüß Gott, Barbara.

\Gregor
\direction{zu \theAndreas, der als letzter kommt} Mach die Tür zu, Andreas.

\Luise
Mein Gott, der Nebel, das ist ja schon wie Rauch.

\Jan
Aber hier ist es gleich. Hier sind wir fidel. \direction{johlt zur Musik} \emph{Denn das Wirtshaus am Rand, das ist ja bekannt im ganzen Land.}

\Gregor
\direction{zu \theJosef} Einen Scharfen, der einheizt!

\Josef
\direction{in dem er ihm einschenkt} Der Jan kriegt nichts. Der ist ja jetzt schon besoffen. \direction{stellt das Radio ab}

\Jan
Oho.

\Josef
Woher kommt ihr denn?

\Gregor
Von dem neuen Kino. Dort gibt es jetzteinen großen Ausschank.

\Josef
Spielt es denn schon?

\Gregor
Nein, noch nicht, aber man kann es sich ansehen. Was rennst du denn so herum, Andreas? Was suchst du denn?

\Jan
Na der, der sucht doch natürlich die Agnes.

\Andreas
So schweig schon einmal.

\Jan
Frau Wirtin, wo ist denn das Fräulein Schwester? Das entzückende, das reizende Fräuleinchen Schwesterchen?

\Barbara
\direction{sieht verwirrt von der Zeitung auf} Agnes?

\Josef
Sie wird gleich kommen, sie war in Dybern, ein Krankenbesuch. Sie muss jeden Augenblick da sein. Aber erzählt doch lieber, wie ist das Kino?

\Jan
Fein, pickfein, viel zu fein für uns arme Leute. Wenn du mal erst die Marmortreppe runtersteigst ---

\Josef
Ist es wahr, dass es ganz unter der Erde ist?

\Jan
Ganz unter der Erde. Das ist jetzt das Neueste, das Modernste. Das ist das Schönste und das Gesündeste und das Billigste. Einen eigenen Architekten haben sie sich dazu herbestellt. Dicht an der Fabrik ist es auch. Man braucht sich nachder Arbeit bloß die Hände waschen ---

\Gregor
\direction{schlägt auf den Tisch} Und jetzt sagmir nur, was dir schon wieder daran nicht recht ist?

\Josef
Der Jan ist bös', wenn er nicht Grund genug zum Stänkern hat.

\Jan
Und ihr seid alle miteinand' Idioten. Wenn euch die hohe Direktion mal ein Zuckerstück hinhält, dann schnappt ihr danach. Sonst kann sie getrost auf eure Köpfespucken.

\Gregor
Es spuckt keiner auf unsere Köpfe. Und wenn sie uns ein Kino hinstellen, so verdienen sie schließlich selber daran.

\Luise
\direction{nachdenklich} Das muss ungeheuer viel gekostet haben, so ein Riesenkino ganz unter der Erde.

\Jan
Jetzt sag mir nur einer, warum ist esdenn ganz unter der Erde?

\Gregor
Damit du das Gras wachsen hörst, du Rotzbub. Das tust du ja ohnehin so gern.

\Andreas
Ich versteh aber auch nicht, warum sie es so hinunter bauen.

\Gregor
Jetzt fängst du auch an.

\Jan
Wir sind doch keine Maulwürfe.

\Gregor
Brauchst ja nicht runter, wenn es dirnicht passt.

\Kathrine
Mein Bub war auch in so einem Kino unter der Erde. Es war sehr nass.

\Josef
Schon gut, schon gut, Kathrine, sprich da nicht mit.

\Kathrine
Ihr werdet alle noch hinunter müssen. Damals hat auch ein jeder geglaubt, es trifft nur den andern.

\Josef
Aber es ist doch ein Kino, Kathrine, ein Kino.

\Jan
Schrei nicht so, sie ist ja nicht taub, nur blind. \direction{greift nach \theGregor{}s Glas}

\Gregor
\direction{hält ihn zurück} Lass sein, bist jaohnehin schon besoffen.

\Jan
\direction{reißt ihm das Glas aus der Hand und trinkt es aus} --- Natürlich bin ich besoffen, ich bin ja immer besoffen, und wenn mir mal die Luft ausgeht in unserer alten Stinkbude drüben, dann bin ich auch nur besoffen, fragt doch den Doktor Thomsen, der hat das gesagt, und wenn ein Mädel umfällt, mitten in der Arbeit, dann bin ich auch nur besoffen, und wenn einer die gewissen Flecken kriegt, die blauen Flecken, erst an den Händen, dann in den Augen ---

\Barbara
\direction{die bisher mit der Zeitung vor sichwie teilnahmslos gesessen ist} Was für Flecken, was für blaue Flecken?

\Luise
Halts Maul, Jan, du redst dich noch umdeinen Kopf.

\Gregor
Pack dein Bündel und geh', hat dich ja keiner nicht hergebeten, bist doch so nur ein Fremder. Geh du dort hin, wo es dem Arbeiter wirklich schlecht geht, wo er kein Fressen hat, kein Dach überm Kopf. Dort kannst du deine Reden halten, Gefahren gibts in jeder Fabrik.

\Josef
Und in unserer Gegend sind die besten Löhne. Das weiß ein jeder. Es liegt ein Segen über dem ganzen Land. Blumen hinter allen Fensterscheiben. Wenn ich denke, wie es früher gewesen ist. Nichts auf's Brot haben die Leute gehabt.

\Barbara
\direction{steht auf} --- Sagt mal, was sind denn das für blaue Flecken?

\Gregor
Nur aus Unvorsichtigkeit. Ihr könnt euch drauf verlassen, immer nur aus Unvorsichtigkeit. Was predige ich nicht täglich den Leuten. Eine Stickstoffabrik ist schließlich kein Kinderzimmer.

\Luise
Nein, Gregor, das ist eine Gemeinheit, was du sagst. Und außerdem bestimmt nicht wahr. Es tut nicht gut, wenn einer von uns so spricht.

\Gregor
Dir ist natürlich lieber, wenn einer hetzt, wie dein Jan. Siehst' dich wohl auch schon Versammlungen halten. Und was kommt dabei raus --- nichts als Not und Elend. Schau' du lieber, dass du einen braven Mann kriegst und ein paar Kinder und ein Häuschen in unserer Siedlung.

\Andreas
\direction{sieht auf die Uhr} Wann soll die Agnes von Dybern zurück sein?

\Josef
\direction{sieht zum Fenster hinaus} Es ist stock finster. Vielleicht sollte man ihr entgegengehen.

\Luise
Aber die Straße ist doch gut beleuchtet.

\Barbara
Sie kommt auf dem Weidenweg.

\Andreas
Auf dem Weidenweg!

\Jan
Herr des Himmels, was schickt ihr sie denn auf den Weidenweg!

\Gregor
Auf dem Weidenwog ist der Nebel am schlimmsten. Dort steigt er auf vom Fluss.

\Barbara
\direction{sehr heftig} Ja seid ihr denn alle verrückt geworden. Das Mädel hat doch eine Taschenlampe bei sich. Die geht den Weg zum hundertsten Mal.

\Luise
Aber der Nebel.

\Barbara
Am Nebel ist noch keiner gestorben.

\Andreas
\direction{steht auf} Gebt mir eine Lampe. Ich geh' ihr entgegen.

\Josef
\direction{geht eine Laterne holen}

\Andreas
Macht rasch, rasch --- Agnes --- mein Gott Agnes es wird ihr doch nichts geschehen sein. \direction{reißt \theJosef die Laterne aus der Hand und stürzt hinaus.}

\Barbara
\direction{ steht mitten im Zimmer und zählt während des Sprechens an den Fingern}
Ein paar Rehe, eine Kuh, einem Bauern ist schlecht geworden und seinem Sohn. In der Zeitung steht, das sind nur Gerüchte, in der Zeitung steht, es istnicht der Nebel, in der Zeitung steht, man kann ja auch im Wald krank werden und auf offenem Feld, und Tiere sterben eben --- aber der Hund --- sag mal, Gregor, hat der Hund wirklich die Tollwut gehabt?

\Gregor
Was weiß denn ich.

\Barbara
Ja, warum wisst ihr das denn alle nicht?

\Luise
Es ist ein ungesunder Herbst, Barbara. Und wenn der Nebel einmal so dick wird wie eine nasse Mauer ---

\Josef
Sollst dir nicht so viel den Kopf zerbrechen, Barbara. Gesunden Lungen macht der Nebel nichts. Nur wenn einer ohnehin Asthma hat oder sonstwie schweren Atem

\Jan
Und wo doch die Hauptsache ist, dass die Bevölkerung ruhig bleibt, immer nur ruhig, wie es an unserer Kirche angeschlagen steht, und der Herr Pfarrer hat ja auch gepredigt, dass Gott es gar nicht so böse meint. Man braucht doch nicht in den Wald zu gehen und zum Fluss

\Josef
Halts Maul, Kerl, oder ich schmeiß' dich hinaus.

\Barbara
\direction{sinkt plötzlich auf einem Stuhl zusammen} Agnes --- o Gott, warum habt ihrmir das nicht früher gesagt.

\Luise
Aber, Barbara, es ist doch nichts geschehen.

\Josef
Das kommt nur von den verfluchten Gerüchten.

\Gregor
Im Nebel kann doch keine Krankheit sein.

\Jan
\direction{steht auf} Ich geh dem Andreas nach.

\Barbara
Was ist denn aus dem Bauern gewordenund seinem Sohn?

\direction{\theJosef und \theGregor zucken die Achseln}

\Barbara
Ja, warum wisst ihr das denn alle nicht?

\direction{man hört den Motor eines Autos und hupen. \theJosef geht hinaus und kommt gleich darauf mit Doktor \theThomsen zurück. \theThomsen ist ein älterer, robuster Mann}

\Thomsen
\direction{schüttelt sich vor Nässe} \ldots Guten Abend.

\Gregor
Guten Abend.

\Luise
Gott sei Dank, der Doktor.

\Thomsen
Wieso? Was ist denn?

\Josef
Es ist keine Ruhe im Land, Herr Doktor

\Thomsen
Ja, ja, schon gut. Gebt mir rasch cinen starken Grog, und dann hilft mir einer bei meiner Panne. \direction{setzt sich mürrisch in eine Ecke}

\Josef
\direction{aus der Küche heraus} Wollen Herr Doktor auch was essen?

\Thomsen
Nein.

\Barbara
Woher kommt denn der Doktor?

\Thomsen
Aus Dybern.

\Luise
Von einem Kranken?

\Thomsen
Nein.

\Gregor
Von sonst einem Besuch?

\Thomsen
Nein.

\Barbara
\direction{steht auf, stellt sich vor \theThomsen, voll Angst} woher denn sonst?

\Thomsen
Von einem Toten.

\direction{Stille}

\Thomsen
Von zwei Toten, wenn ihr es wissen wollt.

\Barbara
Der Bauer und sein Sohn. \direction{sinkt wieder in ihrem Stuhl zusammen}

\Josef
\direction{indem er den Grog bringt} \ldots Asthma, Herr Doktor, nicht wahr, ein schlimmer Atem.

\Luise
\direction{sehr klar} Ist es wahr, dass eine Seuche im Nebel steckt? Dass man krank wird im Wald?

\Thomsen
Es scheint so zu sein.

\Luise
Warum sperrt man dann den Wald nicht ab?

\Thomsen
Weil man nicht wissen kann, wie der Wind sich dreht.

\Gregor
Und woher kommt die Krankheit so plötzlich?

\Thomsen
Das weiß Gott allein.

\Kathrine
Du sollst den Namen Gottes nicht eitel nennen.

\Thomsen
\direction{fährt zusammen} Was ist das?

\Josef
Entschuldigen, Herr Doktor, es ist nur unsere alte Kathrine.

\direction{macht wieder das Zeichen an der Stirn. Zu \theBarbara, die plötzlich ein Tuch vom Haken reißt undzur Tür geht}

Was ist, wohin willst du?

\Barbara
Ich geh dem Mädel entgegen. \direction{stockt, denn man hört ein paar schrille Pfiffe}

\Luise
\direction{springt auf} Das ist Jan, das ist seine Pfeife.

\Andreas
\direction{Stimme}
\ldots Macht auf, macht rasch auf.

\direction{\theJosef reisst die Tür auf, \theJan und \theAndreas tragen \theAgnes herein. \theAgnes, ein fünfzehnjähriges zartes Mädchen, gelb im Gesicht, mit entsetzlichen Augen}

\Agnes
Wasser --- Wasser

\Thomsen
\direction{wirft seinen Mantel auf eine Bank} \ldots legt sie her, sofort.

\Agnes
Wasser \direction{mit den Händen an der Brust} Ich brenne.

\Luise
\direction{hält ihr ein Glas Wasser hin} Hier.

\Agnes
\direction{trinkt} Ich brenne, ich brenne.

\Josef
Herr Doktor, Herr Doktor!

\Barbara
\direction{schiebt \theAgnes ein Tuch unter denKopf} Agnes.

\Andreas
\direction{wirft sich neben ihr hin} Agnes.

\Agnes
Ich brenne.

\Thomsen
Nicht weiter trinken. \direction{er reißt ihrdas Glas aus der Hand}

\Agnes
Das Wasser brennt. \direction{sinkt einen Augenblick zurück}

\Kathrine
\direction{sie ist bis jetzt im Hintergrundan der anderen Seite des Zimmers gesessen. Nun geht sie, den Stock gehoben, in der Richtung von \theAgnes wie gezogen auf das Mädchen zu} \ldots Es riecht nach Senf.

\Agnes
Ich verbrenne.

\end{play}

% ----------


\scene{Kriesensitzung}
\label{scene:II}
\characterlist{\theGeneraldirektor, \theClarisse, \theAlexis, \theThomsen, \theJonas, \theDiener, \theSalwin}
\marginnote{Ledersessel}
\setting{Ein elegantes, diskretes Herrenzimmer, Bücherregale, Tisch mit Klubfauteuils. Schreibtisch, Telephon. Es ist Abend, An dem Schreibtisch, der mit Zeitungen übersät ist, sitzt der \theGeneraldirektor, ein angenehmer, etwas dicker Mensch, mit gutmütigem Gesicht. Er liest in einer Zeitung. Wie der Vorhang aufgeht, tritt eben seine Frau, \theClarisse, in die Tür hinter seinem Rücken. Sie ist hübsch und gepflegt.}

\begin{play}

\Clarisse
Paul!

\Generaldirektor
Ja, mein Kind?

\Clarisse
Sag mal, Paul, ist es wahr, kommst du wieder nicht zum Abendessen?

\Generaldirektor
\direction{mit einer Handbewegung gegen den Schreibtisch} Geschäfte. Dusichst doch, ich weiß nicht, wo mir derKopf steht.

\Clarisse
\direction{setzt sich auf die Armlehne von seinem Sessel} Brrr, die vielen Zeitungen. Kann das nicht dein Sekretär machen? Du musst das Zeug doch nicht selber lesen.

\Generaldirektor
Lass sein, Kind, das verstehst du nicht.

\Clarisse
Aber, dass du was essen musst, das verstehe ich. Ich werde dir ein paar Brötchen bringen lassen und Tee. Und ich werde dir dabei Gesellschaft leisten.

\Generaldirektor
Du bist lieb.

\Clarisse
Immer allein im Speisezimmer unten, das halte ich nicht aus. Zu Mittag sind noch wenigstens die Kinder bei mir.

\Generaldirektor
Es dauert jetzt wirklich nurmehr ein paar Tage.

\Clarisse
Und dann?

\Generaldirektor
Dann --- fahren wir vielleicht nach Paris.

\Clarisse
Warum nur vielleicht?

\Generaldirektor
Schau, Clarisse, du darfst mich nicht quälen, es gehen wichtige Dinge vor im Werk. Vielleicht fährst du allein voraus mit den Kindern.

\Clarisse
Mit den Kindern? Du bist wohl nicht recht gescheit. Was soll ich mit den Kindern in Paris? Die brauchen doch keine Abwechslung.

\Generaldirektor
Wir werden uns das alles nochüberlegen. Am besten ist, \direction{das Telephonläutet, er nimmt don Hörer} Entschuldige---

jawohl, Doktor Thomsen, jawohl, ich habe angerufen--- .

Hören Sie mal---ich habe da ich habe da eine kleine Konferenz bei mir zu Hause---

Sie wissen schon---

einige Herren unseres Betriebes---

ich möchte Sie sehr bitten, kommen Sie doch auch herüber---

ja, ja, sofort---

Wie? schonwieder? Und wieder bei Dybern---

ja, dasverstehe ich, ich begreife vollkommen, eben deshalb wäre es wichtig, dass Sie jetzt kämen---

natürlich können Sie den Herrn auch mitbringen --- Doktor Jonas, nicht wahr---

er ist doch verlässlich---

ja,ja Also Sie kommen sofort.

\direction{legt den Hörer auf}

\Clarisse
Was ist denn los, Paul? Was ist denndas für eine Konferenz hier bei uns zu Hause um neun Uhr abends? Und wozu brauchst du dazu Doktor Thomsen?

\Generaldirektor
Ich kann dir das jetzt wirklich nicht in der Eile erklären. Die Herren müssen jeden Augenblick kommen. Ich bitte dich, du darfst mir nicht böse sein, aber ich bin sehr nervös, überreizt, überarbeitet --- geh lieber schlafen.

\Clarisse
Da will ich dich nicht länger stören, gute Nacht.

\direction{geht beleidigt hinaus. In der Tür prallt sie zusammen mit Ingenieur \theAlexis, der sich flüchtig vor ihrverneigt. \theAlexis ist ein etwas geschniegelter junger Mensch}

\Generaldirektor
\direction{schüttelt ihm die Hand} Gut, dass Sie da sind, Alexis. Sie sind der Erste. Nehmen Sie Platz. Hier, eine Ziegarette. Haben Sie den Oberst erreicht?

\Alexis
Er war natürlich unerreichbar wie immer. Im Laboratorium, darf nicht gestört werden.

\Generaldirektor
Er muss aber kommen. Es ist ganz unmöglich.

\Alexis
Herr Melchior, oder wie der Kerl heisst, den er sich da hält, versprach, ihm alles auszurichten. Eine scheußliche Visage hat dieser Bursche.

\Generaldirektor
Halten Sie es für sicher, dass der Oberst kommt?

\Alexis
Wir wollen hoffen.

\Generaldirektor
Ich habe Brix seit--- seitden unheimlichen Ereignissen überhauptnicht mehr gesehen. Es ist nicht angenehm, mit einem Menschen zu arbeiten, mit dem man so gar keinen Kontakt halten kann. In einem Fall wie jetzt, wenn die fürchterlichsten Gerüchte plötzlich entstehen, Verdächtigungen, die wir allen icht auf uns sitzen lassen können.

\Alexis
Wieso? Wer?

\Generaldirektor
Hier noch niemand. Bei uns in der Gegend noch niemand. Aber das Ausland, die Zeitungen --- noch wagt kein Mensch auszusprechen, was zwischen den Zeilen steht, noch schreibt man überall nur von dem rätselhaften Nebel --- es ist übrigens schon wieder ein Todesfall. Ein achtjähriger Junge in Dybern.

\direction{in diesem Augenblick steht \theClarisse wieder in der Tür}

\Clarisse
Ich bitte vielmals um Entschuldigung,ich ich muss dich rasch noch etwas fragen.

\Generaldirektor
Bitte?

\Clarisse
Im Kinderzimmer sind alle Fenster geschlossen. Die Kleinen haben doch noch keine Nacht ohne frische Luft geschlafen. Die Nurse behauptet, du hättest Auftraggegeben?

\Generaldirektor
Stimmt. Es ist ein hässlicher Nebel draussen, der sich auf die Lungen legt. Wir wollen für ein paar Tage ein Ausnahme machen.

\Clarisse
Aber Nebel allein kann doch nicht schaden

\direction{in diesem Augenblick führt ein \theDiener Doktor \theThomsen und Doktor \theJonas herein. \theJonas noch jung, hager und eckig}

\Clarisse
\direction{zu \theThomsen} Sagen Sie selbst, Herr Doktor, ob Nebel allein Kindern etwas schaden kann.

\Thomsen
\direction{sehr förmlich} Darüber kann ich keine Auskunft geben, gnädige Frau.

\direction{zum \Generaldirektor} Erlauben Sie, Herr Generaldirektor, dass ich Sie mit meinem Kollegen bekannt mache: Doktor Jonas. \direction{Händeschütteln}

\Generaldirektor
Doktor Jonas --- Ingenieur Alexis \direction{auf \theThomsen zeigend} Die Herren kennen sich wohl schon.

\Clarisse
\direction{im Hintergrund, betroffen und ängstlich} Paul.

\Generaldirektor
\direction{legt den Arm um sie undführt sie hinaus} \ldots Du solltest wirklich

\direction{kommt gleich darauf wieder zurück, etwas feierlich} Ich bitte, die Herren Platz zu nehmen. Leider sind wir noch immer nicht ganz vollzählig. Oberst Brix fehlt. Er muss aber jeden Augenblick kommen. Ich hoffe wenigstens. Wünschen die Herren Zigarren?

Bitte, \direction{läutet dem \theDiener}

iberst Brix ist leider nicht telefonisch erreichbar. Es gehört zu seinen Schrullen.

\direction{zum Diener} Whisky! ---

Es gehört zu seinen Schrullen, dass er um nichts in der Welt ein Telephon haben will. \direction{der \theDiener kommt mit Whisky} Er ist ein Sonderling, aber ein großer Gelehrter.

\Alexis
Ein Genie.

\Generaldirektor
Er ist die Seele unseres Werkes. Seine Erfindungen sind unabsehbar. Nur von praktischen Dingen will er nichts hören. Diesmal jedoch werden wir ihm das nicht ersparen können. Die Situation ist unerträglich.

\direction{einen Augenblick Schweigen, plötzlich wendet er sich unvormittelt an \theThomsen} Der Junge ist also wirklich tot? Und unter denselben Symptomen?

\Thomsen
Unter denselben Symptomen wie die andern. Sie können meinen Kollegen fragen. \direction{Hand bewegung gegen \theJonas}

\Jonas
Es ist immer dasselbe Bild.

\Generaldirektor
Und zwar?

\Thomsen
Erstickung.

\Jonas
Vergiftung.

\Generaldirektor
Herr Doktor, was meinen Sie damit?

\Alexis
\direction{fast gleichzeitig auffahrend} Wie können Sie so was behaupten?

\Jonas
Ich behaupte gar nichts. Es sind Vergiftungssymptome.

\Generaldirektor
Ich muss Sie bitten, sich deutlicher auszudrücken. An welche Art von Gift denken Sie?

\Jonas
An keines, das ich kenne.

\Generaldirektor
Also?

\Thomsen
Der Nebel ist vergiftet.

\Alexis
Dann ist es ja doch der Nebel.

\Thomsen
Es steckt eine neue, eine gefährliche Krankheit im Nebel. Eine Krankheit, die wir Ärzte noch nicht kennen.

\Alexis
Also eine neue Seuche, eine Epidemie?

\Thomsen
Es sieht so aus.

\Generaldirektor
Meine Herren, wenn wir mit unserer eigentlichen Konferenz auch auf den Oberst warten müssen \direction{Blick auf die Uhr} so dürfen wir doch auch jetzt nicht aneinander vorbei reden. Wir müssen den Tatsachen ins Auge sehen, wir müssen die Dinge beim Namen nennen.

\direction{in diesem Augenblick bringt ihm der \theDiener eine Visitkarte, die er erregt auf den Tisch wirft}

Da haben wir es. Ich wusste es ja. Ein Herr von der Pressekorrespondenz. Und um diesen Zeitpunkt. Eben angekommen. Nein, ausgeschlossen, unmöglich. Ich bin jetzt nicht zu sprechen. Der Herr soll morgen wiederkommen. Das heißt, nein, wenn er will, kann er warten. Er kann auch heute noch Bescheid erhalten. Führen Sie ihn in das Billardzimmer.

\direction{\theDiener ab}

\Alexis
\direction{nimmt die Karte} Salwin. Den Menschen kenne ich. Mit dem ist nicht zu spaßen.

\Generaldirektor
Am besten ist, wir lehnen die Verantwortung ab. Wir verlangen eine Kommission, ehe es zu spät wird. Chemische Werke sind keine Spielwarenfabrik. Was immer in der Gegend geschieht, es fällt auf uns. Dringt der Verdacht einmal in die Bevölkerung, so ist unabsehbar, wasnoch geschehen kann.

\Alexis
Welcher Verdacht?

\Generaldirektor
Sie wissen sehr gut, was ich meine.

\Alexis
Nein, ich weiß es nicht. Bei Gott, ich weiß es nicht. Und wer denn sonst, als ich, sollte es wissen. Es ist ausgeschlossen, ich schwöre Ihnen, meine Herren, es ist ausgeschlossen, dass dass unsere Industrie die Schuld daran trägt.

\Jonas
Sie glauben also an eine Nebelkrankheit?

\Alexis
Herr Doktor, was soll diese spöttische Frage?

\Thomsen
Mein Kollege meint ja selbst, dass es der Nebel ist. Er --- wir verstehen nur nicht, wieso.

\Alexis
Es muss doch jedom Kind begreiflich sein, dass die --- die gefährlichen Stoffe aus unserer Fabrik nicht plötzlich in den Wald von Dybern kommen können, ohne dass ein Mensch in der Fabrik selbst oder in der Stadt erkrankt. Wir sitzen doch hierdern schließlich Wand an Wand mit den großen Laboratorien --- aber in Dybern stirbt das Vieh am Fluss.

\Jonas
Und auch die Menschen, Herr Ingenieur. Denken Sie an das Wirtshaus am Rand, an das junge Mädchen, das Sonntag dort verbrannte, bei lebendige Leib inwendig verbrannte.

\Thomsen
Der Fall hat ungeheures Aufsehen erregt, vielleicht noch mehr als alle anden. Das Mädchen starb in der Wirtsstube vor den Augen der Gäste. Seither traut kein Mensch sich mehr auf den Weidenweg.

\Alexis
Nehmen Sie an, wir fabrizieren Gift, entsetzliches, gefährliches Gift. Nehmen Sie an. Aber dann bedenken Sie, wie käme dieses Gift gerade auf den Weidenweg. Herr Generaldirektor, helfen Sie mir doch.

\Generaldirektor
Es ist unbegreiflich, es ist nicht auszudenken, aber jeder Verdacht setzt sich durch, und wenn er noch so unsinnig ist. Wir brauchen eine Kommission.

\Alexis
Wir brauchen keine Kommission. Horrgott im Himmel, sind wir denn Verbrecher? Wir arbeiten im Dienste der Wissenschaft, des Friedens und des Staates. Wir sind nicht verpflichtet, Spione zu uns hereinzulassen. Unsere Forschungen sind noch nicht beendet. Oberst Brix wird das nicht gestatten.

\Generaldirektor
Oberst Brix scheint schon wieder einmal unerreichbar. Und ohne ihn sind mir die Hände gebunden.

\Alexis
Dann verlassen Sie sich auf mich. Wirmüssen die schändlichen Gerüchte bloss im Keim ersticken. Ich verbürge mich, ich, der Leiter der Abteilung A.

\Generaldirektor
Und wenn der Nebel bis nachDybern selber dringt? Es braucht sich derWind bloss etwas weiter westlich zu drehen.

\Alexis
Wir sind für den Wind nicht verantwortlich. Dann wird Dybern eben evakuiert.

\Jonas
Und wenn der Wind sich noch weiter westlich oder gar südlich dreht?

\Alexis
Dann schliessen wir unsere Häuser abund versuchen unsere neuen Sauerstoffpumpen.

\Jonas
Sie scheinen ja schon auf alles gerüstet?

\Alexis
Sind wir auch. Wir fürchten uns vorkeinem Nebel.

\Generaldirektor
Alexis, wir sind jetzt ganz unter uns --- sind Sie so sicher, dass es der Nebel ist?

\Alexis
\direction{aufspringend} Meine Herren, wenn esnicht der Nebel wäre, ich bin verantwortlich. Ich bin der Leiter der Abteilung A --- meine Herren, hier vor Ihren Augen würde ich mir eine Kugel durch die Schläfen jagen.

\Thomsen
Unvorsichtigkeit oder dergleichen ausgeschlossen?

\Alexis
Ausgeschlossen.

\Generaldirektor
Ausgeschlossen!

\Jonas
Und Verbrechen?

\Alexis
Sie sind ja wahnsinnig. \direction{der \theDiener kommt herein}

\Diener
Der Herr kann nicht länger warten. Der Herr muss heute Nacht noch weiterfahren.

\Generaldirektor
\direction{steht auf} Dann führen Sieden Herrn herein. \direction{gleich darauf wird vom Diener hereingeführt \theSalwin. Klein, geschmeidig, unterwürfig und unverschämt}

\Salwin
\direction{sich umsehend} Herr Generaldirektor?

\Generaldirektor
\direction{auf ihn zutretend} Das binich. Ich begrüße Sie, Herr Salwin. Sie kommen eben zu einer kleinen Abendgesellschaft. Darf ich Sie mit meinen Gästen bekanntmachen? Herr Salwin von der Pressekorrespondenz. Doktor Thomsen, Ingenieur Alexis, Doktor Jonas. \direction{kurze Verbeugungen}

\Salwin
Ich bitte tausendmal um Entschuldigung, wenn ich so plötzlich hier eindringe. Es handelt sich nur um eine ganz kurze Information.

\Generaldirektor
Wünschen Sie mich allein zusprechen?

\Salwin
Ich weiß nicht aber die Fragen, die ich zu stellen habe, sind doch vielleichtmehr diskreter Natur. Es handelt sich --- es handelt sich um den Nebel von Dybern.

\Generaldirektor
Nein wirklich, davon war ja eben unter uns die Rede. Eine unbegreifliche Geschichte. Aber bitte, nehmen Sie doch Platz. Ich selber werde Ihnen nicht viel sagen können, ich komme den ganzen Tag aus meinem Büro nicht heraus. Aber Herr Doktor Thomsen ist Arzt, er kennt die einzelnen Fälle.

\Salwin
Sehr angenehm, Herr Doktor, sehr interessant. Sie begreifen ja, die ganze Öffentlichkeit harrt gespannt auf eine Aufklärung. \direction{zieht ein Notizbuch hervor} Peinliche Gerüchte kursieren im Ausland. Es wird ja heutzutage alles politisch ausgebeutet. Woran sterben die Leute in Dybern? Was ist Ihre Meinung?

\Thomsen
Ich habe keine bestimmte Meinung. Eine Kommission von Ärzten soll darüberentscheiden.

\Salwin
Aber dieser Kommission werden Sie beitreten?

\Thomsen
Selbstverständlich.

\Salwin
Und Ihre Meinung wird sein? Gift?

\Thomsen
\direction{zurückfahrend} Gift? Was für Gift?

\Alexis
Das ist doch eine gottverfluchte Gemeinheit.

\Salwin
Wie meinen Sie, Herr Ingenieur Alexis? Leiter der Abteilung A, nicht wahr?

\Alexis
\marginnote{\emph{Feigling}.\par(Das äußere Geschlechts\-teil einer Hündin.)}
Ich meine, dass jeder ein Hundsfott ist und ein Vaterlands verräter, der solche Gerüchte verstreut.

\Salwin
Ganz Ihrer Ansicht, Herr Ingenieur. Solche Gerüchte sind sogar Hochverrat, bedeuten Krieg und Schlimmeres vielleicht. Unsere Grenzen können morgen in Flammen stehen.

\Alexis
Ja, zum Teufel, weshalb fragen Sie dann noch?

\Salwin
Wie bitte? Ich? Meine Aufgabe ist es, die Wahrheit ans Licht zu bringen, meine Aufgabe ist es, die Öffentlichkeit zu beruhigen. Herr Doktor \direction{notiert} Thomsen, nicht wahr, woran sterben die Leute von Dybern?

\Thomsen
\direction{sehr kurz} Am Nebel.

\Salwin
Es gibt also eine neue und geheimnisvolle Nebelkrankheit?

\Thomsen
Ja.

\Salwin
\direction{notiert} Nur bei Dybern?

\Thomsen
Ja.

\Salwin
\direction{in seinem Notizbuch lesend} Und das Mädchen vom Wirtshaus am Rand?

\Generaldirektor
War aus der Gegend von Dybern gekommen.

\Salwin
Keine Ursache zur Beunruhigung?

\Jonas
Solange der Wind sich nicht weiter nach Westen dreht.

\Salwin
Wie bitte?

\Jonas
Eine Witterungskrankheit muss wohl vom Winde abhängig sein. Aber das werden Sie kaum brauchen können.

\Salwin
Ich verstehe Sie nicht, Herr --- Herr Doktor Jonas, nicht wahr?

\Jonas
Ich meine, das gehört nicht auch zur Beruhigung der Öffentlichkeit. Und darauf kommt es Ihnen ja an.

\Salwin
Natürlich kommt es mir darauf an. Die Öffentlichkeit hat ein Recht darauf, die Wahrheit zu erfahren. \direction{zum \theGeneraldirektor} Ich möchte nur noch einige unbeträchtliche Informationen über Ihre chemischen Werke \direction{das Telefon läutet Sturm}

\Generaldirektor
\direction{stürzt hin, hebt den Hörerab, hört lange zu, tiefe Stille} Ich bines selbst --- ja, wo --- gewiss --- ich danke Ihnen wir werden dafür Sorge tragen \direction{legt den Hörer zurück}

Meine Herren,eine schlimme Nachricht Dybern liegt im tiefsten Nebel. Erstickungsanfälle. Der Wind hat sich gedreht. Es herrscht eine Panik. Man flieht zu uns. Die ersten Leute sind schon im Wirtshaus am Rand.

\Salwin
\direction{stürzt an das Telefon, schreit hinein} Überland!

\Thomsen
\direction{schon an der Tür} Ich fahre gleich in mein Krankenhaus. \direction{ab}

\Jonas
\direction{zum \theGeneraldirektor} Ich muss nach Dybern. Verschaffen Sie mir sofort ein paar Gasmasken.

\Generaldirektor
Gasmasken? Ich --- ich habe keine Gasmasken.

\Salwin
\direction{wie vorher ins Telefon} Überland!

\Alexis
\direction{zum \theGeneraldirektor} Im Kino haben siebenhundert Leute Platz. Lassen Sies ofort Auftrag erteilen. Unsere Notbetten ---

\Jonas
\direction{zum \theGeneraldirektor, der wie versteinert dasteht} Gasmasken! Ich brauche Gasmasken.
\end{play}

% ---------


\scene{Im Flüchtlingslager}
\label{scene:III}
\characterlist{\theHeilsarmeeschwester, \theJan, \theLuise, \theGregor, \theErsterMann, \theZweiterMann, \theGeneraldirektor, \theAlexis, \emph{Gestalten}}
\setting{Vorraum des Kinos. Photos an den Wänden. Im Hintergrund eine breite Treppe, die nach abwärts führt. Links ein großes, leeres Buffet, rechts eine Kasse, in der eine junge \theHeilsarmeeschwester sitzt, Herein kommen \theJan, \theLuise und \theGregor. \theLuise und \theGregor bleiben etwas betroffen in der Türe stehen, \theJan geht mitstark posierter Unbefangenheit auf die Kassezu.}
\begin{play}

\Jan
Tag, schönes Kind. Wir brauchen drei Billets für die heutige Abendvorstellung.

\Heilsarmeeschwester
Pst, sprechen Sie doch nicht so laut. Die armen Leute untens chlafen schon.

\Jan
Was Sie nicht sagen. Die Leute schlafen. Das ist mir ja ein nettes Kino.

\Luise
Quatsch nicht, Jan. Was sollen die dummen Witze. Frag lieber die Schwester ---

\Jan
Wieso Schwester. Siehst du denn nicht, dass das ein Leutnant ist. Da muss unsereiner wohl salutieren. \direction{salutiert}

\Heilsarmeeschwester
Der junge Mann ist abergut gelaunt.

\Gregor
Hören Sie nicht auf ihn, liebe Schwester. Und seien Sie uns nicht böse, wenn wir stören. Wir kamen nur eben vorbei, und es ist hier so merkwürdig still.

\Heilsarmeeschwester
Die armen Leute sind müde von alle den Aufregungen. Sie liegen schon seit einer halben Stunde in ihrenBetten. Vorher hatten wir noch einen wunderbaren Gottesdienst. Wollen Sie vielleicht auch morgen einem solchen beiwohnen?

\Jan
\direction{indem er am Buffet herumschnüffelt} Und was zu trinken da war, habt ihr ausgesoffen, Pfui Teufel, ist ja alles leer.

\Heilsarmeeschwester
Gott bewahre. Wir haben Alkoholverbot.

\Luise
Finden da unten wirklich siebenhundert Menschen Platz?

\Heilsarmeeschwester
Die armen Leute liegen zu Dritt und zu Viert in den Betten. Einige sogar auf dem nackten Fussboden. Aber sie ertragen ihr hartes Schicksal mit Geduld.

\Gregor
Und wie lange soll das dauern?

\Heilsarmeeschwester
Das weiss Gott.

\Jan \direction{auf dem Buffet sitzend, mit den Beinen baumelnd} Du sollst den Namen Gottesnicht eitel nennen.

\Heilsarmeeschwester
Wie? Was meint der junge Mann?

\Gregor
Sie dürfen nicht auf ihn hören, Schwester. Er ist ein Hanswurst. Geh, schäm dich, Jan.

\Jan
Warum denn? Ich versteh euch nicht. Das ist doch ein frommer Satz, nicht wahr,Schwester Leutnant. Sogar einer von den zehn Geboten. Wo hab ich das nur unlängst gehört? Wo war das Luise?

\Luise
Ach, Jan, erinnere mich nicht daran.Es war zu grässlich. \direction{setzt sich auf einen kleinen Hocker neben dem Buffet}

\Jan
Du darfst nicht böse sein, Luise, aber es fällt mir wirklich erst jetzt wieder ein.Das war doch die alte Kathrine.

\Luise
Ich begreife überhaupt nicht, dass Barbara sie so auf die Dauer erträgt. Sie geht ihr ja nicht von der Seite seit -seit damals. Und dazu noch das Haus voll von Flüchtlingskindern.

\Heilsarmeeschwester
Sie reden wohl von der Frau vom Wirtshaus am Rand. Ja, das muss eine gute Frau sein. Die gehört nicht zu denen, die den Kopf hängen lassen und an Gott verzweifeln, obwohl er sie doch selbst so hart geschlagen hat.

\Gregor
\direction{neben dem Billetschalter} Woher wissen Sie denn von ihr?

\Heilsarmeeschwester
Wir haben mehrere Frauen dort unten, die ihre Kinder bei ihr zurücklassen mussten. Hallo, was ist das, was wollt ihr denn? \direction{zwei Männer in Hemdärmeln, nur dürftig bekleidet, sind eben die Treppe heraufgekommen}

\ErsterMann
Ich halt es nicht aus, wir sind zu Dritt in einem Bett.

\ZweiterMann
Und ich ersticke.

\Heilsarmeeschwester
Aber, aber, wer wird sich denn so gehen lassen. \direction{die beiden Männer setzen sich mutlos auf die oberste Treppenstufe}

\ErsterMann
Die Frau neben mir stöhnt die ganze Zeit.

\ZweiterMann
Und ich geh auch nicht mehr runter in diese Holle Dann lieber noch in den Nebel hinaus. Da wird einer wenigstens gleich kaputt.

\Heilsarmeeschwester
Wollt ihr die ersten sein, die verzagen? Habt ihr nicht mitgesungen heute Abend bei unserem tröstenden Gottesdienst?

\Jan
\direction{Ist vom Buffet gesprungen und geht suchend an den Wänden hin und her.} Sagen Sie mal, fromme Schwester, gibt es denn in diesem großartigen und herrlichen Kino keine anständige Ventilation

\Heilsarmeeschwester
Ich weiß von nichts. Aber ich bitte euch, liebe Leute \direction{tritt aus dem Schalter heraus und auf die beiden Männer zu} wollet jetzt keine Ausnahme machen. Geht hinunter zu euren Brüdern und ertragt geduldig das gemeinsame Los.

\ErsterMann
Ich pfeif' auf dieses gemeinsame Los. Da steckt man uns in so ein Hundeloch.

\ZweiterMann
Und ich hol meine paar Sachen und geh. Und wenn ich an den Türen betteln muss.

\Heilsarmeeschwester
\direction{mit aufgehobenen Händen} Bitte, bitte, tut das nicht. Niemand weiß, was der Himmel noch über uns vorhängt. Nur so lange wir wahrhaft einig bleiben, kommt zu dem Zorn des Herrn nicht auch noch die rohe Gewalt der Menschen. Werdet nur jetzt nicht zu Landstreichern. Gehorcht der Ordnung und Disziplin.

\ErsterMann
Das Mensch ist verrückt.

\ZweiterMann
Kann ihr keiner das Maul stopfen.

\Heilsarmeeschwester
Ich bete für euch \direction{steht abgewendet mit gefalteten Händen}

\Gregor
\direction{tritt auf die beiden Männer zu} Ihr seid aus Dybern?

\ErsterMann
Ja.

\Gregor
Evakuiert?

\ErsterMann
Was fragst du noch?

\Gregor
Einer von euch kann bei mir schlafen,

\ErsterMann
Oh.

\Gregor
Für zwei ist meine Kammer zu klein,

\ErsterMann
Dann soll der da mit. Er ist noch elender.

\ZweiterMann
Geh du nur selbst, ich bring mich auch durch auf der Straße.

\Heilsarmeeschwester
Halleluja, gepriesen sei Gott und unser Heiland, der Erlöser.

\Jan
\direction{der immer noch an den Wänden herumschleicht} Der andere kann zu uns, nicht wahr, Luise, du hast nichts dagegen?

\Luise
Wir werden schon noch einen Strohsack auftreiben.

\Jan
Aber du bleibst im Bett, und schnarchen darf er nicht.

\Heilsarmeeschwester
Halleluja, halleluja.

\Jan
Hören Sie doch auf mit dem Geplärr, Schwester Leutnant, Sie wecken ja die armen Leute unten.

\Heilsarmeeschwester
Gesegnet sei der Herr, der das menschliche Herz erweicht und den menschlichen Sinn.

\direction{\theErsterMann und \theZweiterMann stehen auf, sehr verlegen}

\ErsterMann
Ich dank auch schön

\ZweiterMann
--- ich dank auch schön.

\Jan
\direction{klopft plötzlich an die Wand} Hallo, Schwester, was ist denn das? Das klingt ja hohl.

\Heilsarmeeschwester
Ich weiss es nicht, ich weiß es nicht.

\Jan
Kommt mal alle her und hört euch das an. \direction{klopft weiter, die andern stellen sich um ihn herum}

\Heilsarmeeschwester
Lassen Sie doch unsere Wände in Ruh.

\Jan
Passt auf, die Wand muss sich sicher wo öffnen lassen. Weg, Schwester. \direction{schiebt sie zur Seite} Sie haben hier nichts zu befehlen.

\Heilsarmeeschwester
Ich habe hier alles zusagen, ich, wer denn sonst. Mir ist das Flüchtlingslager überantwortet, und ichverbiete euch

\Jan
Seht ihr den feinen Strich dort an derWand. Dünn wie Spinn web. Da muss doch irgendwo ein Griff oder ein Hebel sein.

\Heilsarmeeschwester
Hier werden keine Untersuchungen angestellt. Und ich befehle euch im Namen des Herrn

\Jan
Welches Herrn?

\Heilsarmeeschwester
Im Namen Gottes, der denGehorsam verlangt: Geht nach Hause, meine Lieben, ich bitte euch. Ihr werdet es sicherlich nicht bereuen. Und nehmt diebeiden Gäste mit, die ihr in eurer Herzensgüte \direction{die andern sprechen jetzt fast gleichzeitig}

\Luise
Sich doch mal nach hinter dem Schalter,Jan.GREGOR Jetzt möchte ich aber schon selberwissen

\ErsterMann \direction{gemeinsam}
\ZweiterMann
\direction{klopfen an die Wand} Esklingt hohl, ja es klingt wirklich hohl.

\Jan
\direction{hinter dem Schalter, zur Schwester, die ihn zurückhalten will}

Gehen Sie zum Teufel, Sie Leutnant.

\direction{in diesem Augenblick wird die Türe aufgerissen und herein kommen der \theGeneraldirektor und \theAlexis. Beide fahren zuerst zurück}

\Alexis
Schwester, was ist das für eine Volksversammlung?

\Generaldirektor
Wir haben Sie doch ausdrücklich ersucht, für unbedingte Ruhe des Nachts zu sorgen.

\Heilsarmeeschwester
Verzeihung, liebe Herren,aber es ist nicht meine Schuld.

\Alexis
Wer sind die Leute hier? Flüchtlinge?

\Heilsarmeeschwester
\direction{auf die beiden Männer deutend} Nur diese beiden, lieber Herr.

\Generaldirektor
Und die andern?

\Heilsarmeeschwester
Ich weiß es nicht. Vorübergehende Spaziergänger.

\Jan
Wollen eben auch einmal ins Kino gehen.

\Generaldirektor
Hier gibt es weiß Gott keine Abendbelustigung.

\Alexis
Vorübergehende haben hier nichts zusuchen.

\Luise
Ja ist denn hier ein Gefangenenlager?

\Alexis
Sie haben keine Fragen zu stellen.

\Gregor
Komm, Luise, komm, wir wollen gehen.

\Luise
Und wenn die Leute da unten ersticken, wird man sie wohl zu sich einladen dürfen.

\Generaldirektor
Wie? Was reden Sie da? Was soll das heißen?

\Luise
Der Mann dort ist unser Gast. Er kommt mit uns.

\Alexis
Halt, das gibts nicht. Wie konnten Sie nur erlauben, Schwester

\Heilsarmeeschwester
Die guten Menschen warenvoll Erbarmen, als sie die beiden Männer heraufkommen sahen.

\ErsterMann
Und was mich betrifft, die Herren müssen entschuldigen. Ich halt es indem Gestank nicht aus.

\ZweiterMann
Und mich bringt man nicht mehrin diese Hölle.

\Generaldirektor
Aber das geht doch nicht, ihrgehört doch zum Lager, da kann man jetzt keine Ausnahme machen.

\Alexis
Es ist völlig unmöglich,

\direction{in dem Augenblick tauchen sechs, acht Gestalten, alle eben aus dem Bett aufgestanden, hinten auf der Treppe auf}

Also sehen Sie, da haben wir schon die Bescherung.

\Generaldirektor
\direction{auf die eben Erschienenen zutretend und sehr höflich} Ich bittetausendmal um Entschuldigung, falls wirSie stören. Wir kamen nur --- um die Ventilation zu untersuchen.

\Heilsarmeeschwester
\direction{mit ausgebreiteten Armenauf sie zutretend} Geliebte Brüder, vertraut doch auf uns. Die Herren sind nur gekommen, um euch Gutes zu erweisen, dennin Zeiten der Not da blüht erst die Nächstenhilfe.

\Leute
Wir können aber nicht schlafenEs ist zu stickig\ldots Das ganze Lager ist ja auf \ldots Wenn wir der Schwester nichtso fest versprochen hätten

\Heilsarmeeschwester
Seht ihr, ihr habt es mir versprochen. Und wollt ihr nun euer Wort nicht halten \ldots Ich aber komme zu euch\ldots{} ich will mit euch beten \ldots --- Wir wollensingen --- einen Choral --- kommt.

\direction{die \theLeute lassen sich von ihr zurücktreiben, sie folgt ihnen die Treppe hinunter}

\direction{die beiden Männer haben inzwischen \theJan, \theGregor und \theLuise zur Tür hingezogen.}

\ErsterMann
Na denn fix, verduften wir.

\ZweiterMann
Jetzt lasst uns nur nicht auchnoch im Stich.

\Gregor
\direction{zu \theJan} So komm doch schon. Waswillst du denn noch?

\direction{\theErsterMann, \theZweiterMann, \theGregor, \theLuise und \theJan schlüpfen zur Tür hinaus}

\Alexis
\direction{merkt es im letzten Augenblick, will ihnen nach} Halt, halt, das geht nicht.

\Generaldirektor
\direction{hält ihn zurück} So lassen Sie doch. Sie können die Leute ja auch nicht anbinden.

\Alexis
Ich fürchte Unruhen, Aufstand und Gewalt mehr als jeden Nebel.

\Generaldirektor
Jetzt ist nicht Zeit, darübernachzudenken. Jetzt haben wir Wichtigereszu tun. Wo ist der Schlüssel? \direction{sperrt dieTür ab} Und die Tür muss abgesperrtbleiben. Wenn nur die Schwester nichtkommt.

\Alexis
Sie singt doch unten einen Choral. Hören Sie nicht? \direction{man hört auch wirklich nicht zu laut einen Choral}

\Generaldirektor
Dass der Oberst nicht hierist, habe ich erwartet. Es nützt nichts, ihn zu bestellen. Ich glaube, wir müssen auf ihn verzichten.

\Alexis
Und dabei ist doch eben hier alles sein Werk. Es ist und bleibt mir unverständlich---

\Generaldirektor
Es ist nicht Zeit, sich jetztüber ihn den Kopf zu zerbrechen. Rasch, rasch. Die Schwester kann jeden Augenblick kommen.

\Alexis
Solange sie singt, sind wir ungestört. Und auch nachher hält sie die Leute in Zucht. Wenn der Oberst mit uns die Pumpeuntersuchen wollte.

\Generaldirektor
Sie sind ein Narr, Alexis. Wie können Sie immer noch auf ihn warten. Gott weiß, an welchem Problem er brütet, für ihn existiert ja die Wirklichkeit nicht. Er ist der große Theoretiker. Sie aber, Alexis, sind ein Mann der Tat. Zeigen Sie sich als sein Stellvertreter.

\Alexis
Soll ich wirklich ohne sein Beisein ---

\Generaldirektor
Mir scheint gar, Sie fürchten sich.

\Alexis
Herr Generaldirektor

\Generaldirektor
Nehmen Sie sich zusammen, Herr Ingenieur. Wir müssen die Pumpe jetzt ausprobieren. Wenn der Wind sich weiter nach Süden dreht ---

\Alexis
Nein, nein, nein.

\Generaldirektor
Wo ist der geheime Hebel?

\Alexis
\direction{geht zaudernd auf den Billetschalterzu} Hier. Halt. Die Leute hören zu singen auf.

\Generaldirektor
\direction{lauschend} Sie beten jetzt. \direction{man hört fernes Gemurmel} Rasch! \direction{\theAlexis drückt auf einen geheimen Knopf. Die Wand öffnet sich, eine riesengroße Luftpumpenanlage wird sichtbar. Der \theGeneraldirektor stellt sich beobachtend vor die Treppe} Untersuchen Sie rasch, ob alles in Ordnung ist.

\Alexis
\direction{prüft einige Griffe} Es klappt.

\Generaldirektor
Dann los!

\Alexis
Wollen wir wirklich?

\Generaldirektor
Um Gotteswillen, versuchen Sie es doch. Sofort, sofort, man singt noch einen Choral. \direction{man hört wieder gedämpft einen Choral. Alexis manipuliert an der Pumpe herum. Zischen. Ganz kurz. Dann springt \theAlexis wieder an den Billetschalter zurück und schließt durch den Hebel die Wand}

\Generaldirektor
Ausgezeichnet, es wirkt wie ein kühler Wind. Warum nicht mehr?

\Alexis
Die Leute dürfen nicht Verdacht schöpfen.

\Generaldirektor
Wollen Sie sie lieber unten verschmachten lassen?

\Alexis
Es ist unabsehbar, was geschieht, wonnman nur ahnt, worauf wir gerüstet sind.Wir dürfen uns im Frieden doch nicht wieim Kriegsfall benehmen.

\Generaldirektor
Ich glaube, man würde dankbar sein für etwas frische Luft.

\Alexis
Man würde glauben, dass wir Schuldtragen am Nebel. Was jetzt als Unglück scheint und als Schicksalsschlag, erschiene dann als ---

\Generaldirektor
Als Verbrechen.

\Alexis
Herr General direktor!

\Generaldirektor
Fürchten wir uns nicht vor Worten, Alexis. Um so mehr als wir unschuldig sind.

\Alexis
Das sind wir bei Gott.

\Generaldirektor
\direction{Auf die Wand zeigend} Wieviel Sauerstoff ist dort drin?

\Alexis
Für vierzehn bis sechszehn Tage. \direction{in diesem Augenblick erscheint die Heilsarmeeschwester wieder. Sie ist ganz in Ekstase}

\Heilsarmeeschwester
Gelobt sei der Herr! Halleluja, er sei gepriesen. Er hat ein Wunder getan. Beten Sie mit mir. Danken Sie ihm mit mir.

\Generaldirektor
Was ist denn los, was ist geschehen, Schwester?

\Heilsarmeeschwester
Als ich unten mit der Schar der Verzweifelten auf den Knieenlag, als wir beteten, und als wir sangen,da --- es war wie ein kühler Wind. Die armen Leute atmeten auf.

\Generaldirektor
Schon gut, Schwester.

\Alexis
Ist jetzt Ruhe unten?

\Heilsarmeeschwester
Ruhe und Gottvertrauen.

\Alexis
Sie dürfen aber nicht mehr jeden zu Tür hereinlassen. Weisen Sie alle Neugierigen ab. Hier ist doch schließlich kein Theater.

\Generaldirektor
Wissen Sie überhaupt, wer die Menschen waren?

\Heilsarmeeschwester
Nein, ich weiß es nicht,---Aber es waren gute und ehrliche Leute. Nur der eine junge Mann war so sonderbar und so übermütig. Vielleicht betrunken. Er klopfte immer zu auf die Wand.

\Alexis
Wie, was? An die Wand?

\Heilsarmeeschwester
Er sagte, es klänge hohl.

\Alexis
Schwester, Schwester, um Himmel swillen! Und Sie wissen nicht einmal seinen Namen. \direction{dringt auf sie ein}

\Heilsarmeeschwester
\direction{zurück weichend} Ich --- ich weiss doch nichts.

\Alexis
Und zwei Mann vom Lager sind mitgekommen. Nicht auszudenken, was das für ein Gerede geben wird. Vermutungen tauchen auf, Verdächtigungen --- Herr Generaldirektor, wir brauchen Militär.

\Generaldirektor
Was fällt Ihnen ein.

\Alexis
\direction{packt ihn am Ärmel} Kommen Sie sofort. Die Leute müssen verhaftet werden und mit ihnen dieser Doktor Jonas, derüberall nach Gasmasken sucht. Begründungeinerlei.

\Generaldirektor
Sie sind verrückt, Alexis. Wie stellen Sie sich das eigentlich vor?

\Alexis
Die Heilsarmee ist zu schwach. Ich habe es gleich gesagt. Wir brauchen Militär, echtes Militär.

\Heilsarmeeschwester
Die armen Leute sind doch ohnehin so geduldig.

\Alexis
\direction{zieht den Generaldirektor weiter am Ärmel} Rasch, rasch, im Wagen will ich alles weiter erklären. Schwester, Sielassen keinen Menschen mehr herein.

\direction{es klopft sehr heftig an die Tür}

\Stimme
Aufmachen! Aufmachen! Zum Teufelnoch mal! Sofort aufmachen!

\Alexis
Wer ist da?

\Stimme
Machen Sie jedenfalls sofort auf. \direction{wütendes Klopfen} Aufmachen! Aufmachen!

\Alexis
Wer ist draussen?

\Stimme
\direction{Durch das Klopfen hindurch} Der Generaldirektor --- Nebel --- Scharen --- Auto ---Aufmachen!

\direction{\theAlexis sperrt die Tür auf, hereinstürzt \theSalwin}

\Salwin
Herr Generaldirektor, ich wusste es ja,ich sah ja draußen Ihren Wagen. Ich komme eben von einer Rekognoszierungsfahrt. Aufdem Motorrad \direction{kann nicht weiter vor Atemlosigkeit und Aufregung}

\Generaldirektor
Und, und? Was ist?

\Salwin
Der Wind --- der Nebel --- der Wind hat sich gewendet. Die Leute fliehen in Scharen --- der Wind hat sich gewendet --- nach Süden.

\Alexis
Wo flieht man, wo?

\Salwin
Sagen Sie mal, gibt es denn hier kein Telefon?

\Generaldirektor
Zum Teufel, woher kommen Sie denn?

\Salwin
Ich war allein, auf dem Motorrad ungefähr eine Stunde südlich von Dybern.

\Heilsarmeeschwester
\direction{ist in einer Ecke niedergekniet und betet}

\Generaldirektor
Eine Stunde von Dybern! Es istnicht abzusehen. Kommen Sie sofort Alexis. Wir müssen auf der Stelle --- \direction{ab mit Alexis}

\Salwin
Beten Sie nicht, so beten Sie dochnicht Schwester. Sagen Sie lieber, wo ist hier ein Telefon?

\Heilsarmeeschwester
Unten in der Kanzlei. Dort können Sie jetzt nicht hin. \direction{betet weiter}

\Salwin
Ja warum denn nicht?

\Heilsarmeeschwester
\direction{immer noch zwischen Beten} Dort schlafen ein Dutzend Leute.

\Salwin
\direction{will hinunter gehen} Die werden ohnehin bald aufwachen.

\Heilsarmeeschwester
\direction{hält ihn zurück} Nein, das dürfen Sie nicht, man hat mir verbotenmein Gott, wer sind Sie denn?

\Salwin
Salwin, erster Redakteur der Pressekorrespondenz.

\Heilsarmeeschwester
Sie können jetzt unmöglich hinunter. Ich beschwöre Sie im Namen Gottes und unseres Erlösers: bleiben Sie hier! Es war mir ohnehin kaum mehr möglich, die Leute in Ruhe zu halten. Erst ein Wunder hat mir geholfen.

\Salwin
Interessant. Wie war denn dieses Wunder?

\Heilsarmeeschwester
Während sie sangen und beteten, kam ein kühlender, sanfter Wind.

\Salwin
Interessant. Und Sie sind allein hier, Schwester --- wie ist Ihr Name Schwester?

\Heilsarmeeschwester
Mein Name bleibt ungenannt.

\Salwin
Und Sie haben allein hier eine ganze Horde zu bewachen?

\Heilsarmeeschwester
Die armen Leute sind keineHorde.

\Salwin
Aber sie versuchen aus zubrechen.

\Heilsarmeeschwester
Sie sind eben auch nur irrend und schwach.

\Salwin
Und wenn sie erfahren, dass der Nebel näher rückt?

\Heilsarmeeschwester
Dann gebe Gott mir die Kraft, sie zu stützen.

\Salwin
Sie sind sehr mutig. Aber ich muss jetztgehen, Schwester, wollen Sie mir nicht doch Ihren Namen sagen?

\Heilsarmeeschwester
\direction{schüttelt den Kopf}

\Salwin
Na, denn nicht. Aber noch eins, Schwester \direction{schon in der Tür} Können Sie mir eines noch sagen: Sind Sie ganz sicher, dass dieser sonderbare und rätselhafte Nebel wirklich von Gott kommt?

\Heilsarmeeschwester
Es kommt doch alles von Gott.

\Salwin
Oder vom Teufel. \direction{da sie entsetzt zurückweicht} Entschuldigen Sie, ich habe es nicht ganz so gemeint. Beten Sie nurungestört weiter. \direction{ab}

\Heilsarmeeschwester
\direction{sinkt betend in die Knie}

\end{play}

% ---------


\scene{Hungrige Mäuler}
\label{scene:IV}
\characterlist{
	\theKinder,
	\theJosef,
	\theKathrine,
	\theBrix,
	\theMelchior,
	\theBarbara,
	\theAndreas,
	\theGeneraldirektor,
	\theAlexis,
	\theSoldaten
}
\setting{Wirtsstube wie im ersten Bild. Hinten auf der Ofenbank die alte \theKathrine. Vor ihr spielen vier kleine \theKinder Haschen. Gleich darauf kommt \theJosef auch schon zur Tür herein und treibt einen Haufen \theKinder, ungefähr sieben oder acht, vor sich her. Die \theKinder, Knaben und Mädchen sind fünf bis dreizehn Jahre alt. Etwas trübes, dämmeriges Licht.}

\begin{play}

\Josef
Herein mit euch! Und dass mir keiner von der Bande sich noch vor die Tür wagt.

\Junge{1}
\direction{groß}
Aber was sollen wir hier denn tun?

\Maedchen{1}
\direction{klein, mit sehr heller Stimme} Es ist so langweilig.

\Maedchen{2}
\direction{größer}
Wir können doch nicht den ganzen Tag auf der Ofenbank hocken.

\Josef
Es sind nun mal keine vergnüglichen Zeiten.

\Junge{2}
\direction{ganz klein, weinerlich} Ich will  zu meiner Mutter.

\Josef
\direction{nimmt ihn auf den Schoss} Na wart nur,das wird nicht mehr lange dauern.

\Junge{1}
Ja, das heißt es jetzt alle Tage und dann darf man nichtmal mehr vor die Tür.

\Josef
Jetzt schweigt schon still. In einer halben Stunde gibt's was zu essen. Und einstweilen kann Mutter Kathrine euch eine Geschichte erzählen. \direction{fängt an seine Pfeife zu putzen}

\Kinder
\direction{durcheinander} Fein, fein Kathrine soll uns was erzählen--- nicht wahr, Kathrine, du erzählst uns was.

\Kathrine
Ich weiß nicht viel Lustiges zu erzählen.

\Maedchen{1}
\direction{groß}
Dann erzähl eben was Trauriges.

\Maedchen{2}
\direction{groß}
Ja, ja, das ist uns gerade recht. So was zum weinen.

\Junge{3}
Aber sieh zu, dass auch ordentlich was passiert.

\Kathrine
Ich weiß wirklich nichts zu erzählen, Kinder.

\Kinder
Oh bitte, bitte!

\Maedchen{2}
Vom Weihnachtsmann.

\Junge{3}
Ach was, den gibt's doch heuer nicht.

\Maedchen{3}
Von was denn sonst?

\Maedchen{1}
\direction{mit der auffallend hellen Stimme} Von Feen und Elfen.

\Junge{1}
Die gibts doch erstrecht nicht. Die sind längst schon kaputtgegangen im Wald.

\Maedchen{1}
Können die auch den Nebel nicht vertragen?

\Junge{3}
Ich glaube überhaupt nicht an Feen und Elfen.

\Kinder
Also was anderes, Kathrine.

\Maedchen{1}
Weißt du was. Erzähl' uns vom Tod.

\Junge{1}
Bist wohl verrückt.

\Maedchen{3}
Was hat die für Ideen!

\Maedchen{1}
Aber den Tod, den gibt's. Der ist wirklich im Wald.

\Junge{3}
Mein Vater hat gesagt, es gibt keinen Tod.

\Maedchen{4}
Natürlich gibt es keinen Tod, man stirbt einfach.

\Maedchen{1}
Aber warum denn gerade am Weidenweg?

\Junge{1}
Weil dort der dickste Nebel ist, dummes Ding.

\Maedchen{1}
Ich bin gar kein dummes Ding. Und wenn man im Nebel stirbt, dann ist dort der Tod.

\Josef
\direction{schlägt mit der Hand auf den Tisch} Jetzt hört schon einmal auf mit dem Quatsch.

\Maedchen{3}
Zu meiner Großmutter ist aber doch der Tod gekommen. Er war lang und hager und aus lauter Knochen, mit einer Sense.

\Junge{1}
Red keinen Unsinn.

\Junge{4}
Das ist doch nur so ein Gespenst, das gibt es doch nicht.

\Maedchen{4}
Nicht wahr, Kathrine, das gibt es nicht?

\Junge{1}
So sag doch Kathrine. Zweites kleines Mädchen. Nicht wahr, Kathrine, es gibt einen Tod.

\Maedchen{3}
Geht er wirklich herum und klappert mit allen Knochen?

\Maedchen{2}
Und hat er wirklich einen Mund, der immer nur lacht?

\Maedchen{3}
Und die Augen wie Löcher tief drinnen im Kopf?

\Josef
\direction{indem er wieder auf den Tisch schlägt} Wollt Ihr nicht endlich ein Ende machen!

\direction{Der kleine Junge, der auf seinen Schoß gesessen ist, springt erschrocken herunter und läuft zu den \theKinder{}n. Einen Augenblick Stille}

\Maedchen{1}
So sag doch Kathrine.

\Kathrine
Ihr seid wirklich ganz, ganz dumme Kinder. Und überhaupt soll man vom Tod nicht sprechen. Habt ihr gehört, niemals. Weil er sonst wirklich kommt.

\Maedchen{3}
Also, da seht ihr, es gibt ihn doch!

\Kathrine
Schweig still, Mädel, was weißt denn du. So einen Tod wie du meinst, den gibts schon lange nicht. Der war einmal, als die Menschen noch besser waren, als sie halbwegs Frieden hielten auf Erden. Da hatte auch der Tod noch ein Gesicht. Wenn auch kein schönes, aber doch wie ein Mensch, ein gestorbener Mensch.

\Junge{1}
Und jetzt?

\Maedchen{4}
Wie schaut er jetzt denn aus?

\Kathrine
\direction{vorgebeugt und heiser} Jetzt hat der Tod einen Rüssel. Und glatte Glotzaugen rechts und links. Jetzt hat der Tod gar kein Gesicht, Sieht aus wie ein böses und dummes Tier.

\direction{Die Tür wird nach kurzen Klopfen aufgestoßen und herein kommt ein großer schlanker Mann in feldgrauem Mantel und mit Gasmaske vor dem Gesicht von \theBrix. Ihm folgt ein kleiner dicker Mann mit plattem und gemeinem Gesicht, ebenfalls feldgrau, aber mit der Gasmaske in der Hand. Die Kinder stürzen kreischend vor Angst zur Tür hinaus, die in die Küche führt. \theJosef springt auf, lehnt entsetzt an der Wand}

\Melchior
Guten Tag.

\Brix
Guten Tag.

\Kathrine
Herr Jesus, steh uns bei.

\Josef
\direction{etwas mühsam} Guten Tag.

\Brix
Der Generaldirektor schon hier?

\Melchior
Könnt Ihr denn keine Antwort geben?

\Brix
Die Sauerstoffpumpe schon angekommen?---

\Melchior
\direction{auf \theJosef zutretend} Mensch, dir hat es wohl die Red' verschlagen.

\Brix
Genügend Gasmasken im Haus?

\Josef
Ich --- ich weiß --- von nichts.

\Melchior
Was starrst du denn so? Der Herr ist kein Gespenst.

\Kathrine
Vater unser, der du bist in dem Himmel --- es riecht nach Krieg.

\Melchior
Was schwatzt die Alte dort? Die wird noch mehr Gasmasken zu sehen bekommen.

\Kathrine
Ich werde im Leben keine Gasmaskenmehr zu sehen bekommen. Gott ist mir gnädig.

\Brix
Was soll das heißen?

\Kathrine
Ich brauche nichts mehr zu sehen. Gott sei gepriesen.

\Melchior
Das Weib ist närrisch.

\Brix
Was ist denn mit ihr? Was starrt sie mich so an? \direction{weicht einen Schritt zurück}

\Josef
Herr, sie ist blind.

\Brix
Melchior, wir haben keine Zeit hier zuwarten. Rasch, komm. Leg unseren Plan auf den Tisch, der Generaldirektor soll ihn hier finden.

\direction{Melchior legt ein Stück Papier auf den Tisch}

Hier ist alles verzeichnet, hier steht genau, wie der Nebel weiterrückt. Ihr bekommt Sauerstoff und Gasmasken, die Kinder bringt in den Keller. Alles Übrige werdet ihr noch erfahren. Melchior, setz deine Gasmaske auf.

\direction{Da \theMelchior zögert} Was hast du? Gehorche! Wir gehen. Kehrt euch!

\Melchior
\direction{stülpt die Gasmaske auf, sehr militärisch} Zu Befehl, Herr Oberst.

\direction{beide ab}

\Josef
Verflucht nochmal, das ist mir jetzt in die Glieder gefahren. Der Lange war ja das reine Gespenst. Sauerstoff und Gasmasken.

\Kathrine
Man soll nie vom Tod sprechen, sonst ist er dann plötzlich da.

\Josef
Ach halt die Schnauze, alte Hexe. Das war ein Offizier und sein Soldat. Aber was ist mit Barbara. Wenn nur Barbara ---

\direction{Barbara kommt mit einem Sack Kartoffeln herein}

\Barbara
Was war denn hier, was ist denn geschehen? Die Kinder sind wie außer Rand und Band. Ich war eben im Keller, da kam ein Mädchen heruntergestürzt, aber es war nichts aus ihr herauszukriegen.

\Josef
Es kommen böse Zeiten, Barbara. Der Nebel rückt näher. Man hat uns gewarnt. Sauerstoff und Gasmasken---

\direction{Kathrine ist aufgestanden und humpelt auf die Tür zu}

\Barbara
Wo willst du denn hin, Kathrine?

\Josef
Du hörst doch, dass der Nebel näherrückt.

\Barbara
Geh nicht hinaus.

\Kathrine
Mir kann kein Nebel mehr was anhaben.Ich will nach Haus, ich mag euren Sauerstoff nicht.

\Barbara
Du bist ja wahnsinnig. Wirst doch jetzt nicht fortgehen wollen. \direction{will sie zurückhalten}

\Kathrine
Lass los, ich mag nicht. Mich soll keiner mehr retten. Erst haben sie mir mein Mädelchen verdursten lassen, dann haben sie mir meinen Buben vergiftet. Mir aber wollen sie noch eine Gasmaske aufstülpen. Ich tu nimmer mit. Lass los. \direction{ab}

\Barbara
\direction{starrt ihr nach} Vielleicht hat sie recht.

\Josef
Sprich nicht so, Barbara, wir müssen jetzt handeln.

\Barbara
Was redest du von Handeln. Hier können wir uns höchstens noch wehren.

\direction{schüttelt die Kartoffeln in einen Eimer und setzt sich hin, um sie zu schälen}

\Josef
Es gilt nicht uns allein, Barbara. Wir haben das Haus voll Kinder. Fremde Kinder, das ist eine große Verantwortung.

\Barbara
Komm, nimm ein Messer und hilf mir Kartoffel schälen.

\Josef
Ich kann jetzt nicht Kartoffel schälen. \direction{rennt aufgeregt hin und her} Jeden Augenblick soll der Generaldirektor kommen. Man verspricht uns eine Sauerstoffpumpe und Gasmasken.

\Barbara
\direction{sieht auf} Gasmasken? Ja für wen denn Gasmasken?

\Josef
Es könnte doch sein, weißt du Barbara \direction{hantiert nervös an einem Fenster herum} dass der Nebel auch in die Stuben dringt, durch die Türritzen und den Fensterspalt,und dann ---

\Barbara
Und dann?

\Josef
Dann brauchen wir die Gasmasken auch. Dass du immer noch Kartoffel schälen kannst, Barbara.

\Barbara
Die Kinder müssen doch zu essen haben. Sag mal, bekommen die Kinder vielleicht auch solche Gasmasken?

\Josef
Ich denke schon.

\Barbara
\direction{hört auf zu schälen} Gibt es denn so kleine Gasmasken, so ganz kleine auch, ganz winzig kleine.

\Josef
Es wird wohl große und kleine geben. Aber zuerst werden wir die Kinder einmal in den Keller hinunterbringen. Du hast recht, sie müssen zu essen bekommen.

\direction{setzt sich neben sie und beginnt ebenfalls Kartoffeln zu schälen}

Möglichst bald. Hörst du sie in der Dachkammer oben. Das tobt und lärmt. Wir wollen gut sein gegen die armen Würmer. Stelle dir vor, wenn einmal unser eigenes

\Barbara
\direction{sehr schroff} Schweig still.

\Josef
Nun man kann nicht wissen. Gott wird es uns sicher vergelten. Und wenn einmal unser eigenes

\Barbara
\direction{rasend} so schweig doch schon, ich will das nicht hören.

\Josef
Aber Barbara

\Barbara
\direction{packt \theJosef{}s Arm} Unser eigenes Kind soll auf einer Wiese spielen, in einer wunderbaren, klaren, leuchtenden Luft, Josef, wir wandern aus in ein Land, wo es keinen solchen Nebel gibt, wo es keinen solchen Nebel geben kann. Josef, ich will hier kein Kind aufziehen. Nicht wahr, Josef, wir wandern aus. Auf eine Insel, mitten im Meer. Es muss doch noch einen Ort geben auf dieser Erde, wo solch ein Nebel nicht möglich ist.

\Josef
Aber Barbara, um Gotteswillen!

\Barbara
\direction{plötzlich zusammensinkend} Josef,Josef, ich fürcht' mich so. Mein Kind soll nicht in einem Keller zur Welt kommen. Man darf ihm keine Gasmaske auf das Köpfchen ---

\Josef
Barbara, es ist eine harte Zeit, Du wirst doch jetzt nicht den Mut verlieren, du warst ja so tapfer die letzten Tage. Schau, der Nebel ist wie eine Krankheit. Gott hat sie uns in das Land geschickt, aber auch dieser Nebel wird vergehen, jede Krankheit hört einmal auf zu wüten. Es hilft nichts, wenn man dem Unglück trotzt, das Schicksal ist ja doch stärker als wir.

\Maedchen{1}
\direction{mit der besonders hellen Stimme steckt den Kopf zur Tür herein} Ist der Tod noch da?

\marginnote{\emph{unartiges Kind}.\par(läufiges Mutterschwein)}
\Josef
Willst du wohl, du kleine Range.

\Maedchen{1}
\direction{kommt etwas näher} Gibt es bald was zu essen?

\Barbara
Sehr bald, mein Kind. Ruf doch einmal die andern in die Küche hinunter. Sie sollen Ordnung machen und mir helfen \direction{greift nach dem Messer und schält weiter Kartoffeln} Rasch Josef, die Kinder sollen nicht hungern.

\direction{\theJosef nimmt sein Messer, das kleine Mädchen ab}

\Josef
Hast recht Barbara, das ist jetzt das wichtigste. Kocht das Wasser schon? So mach doch kein so finsteres Gesicht, immer Kopf hoch, du warst doch sonst keine von denen, die leicht verzagen. Musst an unser Kleines denken Barbara. \direction{\theBarbara zuckt zusammen} Ja, Barbara, musst daran denken, was für so ein Kind gut ist ungesund. Du Barbara, ich hab dich schon lang nicht singen gehört.

\Barbara \direction{nimmt den Trog mit den geschälten Kartoffeln und trägt ihn in die Küche. Die Türe bleibt offen. Aus der Kücheheraus fragt \theBarbara sehr heiser} Was soll ich denn singen?

\Josef
Na, halt so ein kleines Lied, so wie früher, du weißt schon.

\Barbara
\direction{singt mit einer rauen, gebrochenen Stimme} Eia popeia, was raschelt im \direction{bricht ab}

\direction{Die Tür wird aufgerissen, \theAndreas stürzt herein. Er ist verstört und erhitzt}

\Andreas
Barbara, wo ist Barbara \direction{sinkt auf eine Bank}

\Barbara
\direction{in der Tür} Was ist geschehen, Andreas?

\Andreas
Jan ist verhaftet. Militär. Man will auch mich. Könnt Ihr mich verstecken?

\Barbara
\direction{ist ins Zimmer getreten} Was hast du getan, Andreas?

\Andreas
Das Maul aufgemacht.

\Josef
\direction{schließt die Tür zur Küche, in der ein paar \theKinder auftauchen} Pst, sprich nicht so laut, daneben sind Kinder.

\Barbara
\direction{bringt Andreas ein Glas Wasser} Da hast du Andreas, trink erst einmal. Und mach das Hemd zu. Sei ganz ruhig, dann kannst du erzählen. Hast ja nicht einmal einen Mantel an und rennst so durch den Nebel.

\Andreas
Das ist kein Nebel, das ist doch kein Nebel. Glaubt das nicht länger. Sie wollen uns ja nur was vorlügen. Schwindel, Schwindel, nichts als Schwindel.

\Josef
Mensch, nimm dich in acht. Was redest du da.

\Barbara
Kein Nebel?

\Andreas
In der Stadt weiå es beinah schon einjeder. Der Jan hat es gesagt, der Jan hat seine Nase in mehr hineingesteckt, als ihr ahnt, der kann auch was erzählen von Sauerstoffpumpen, die heimlich in den Mauern stecken. Deshalb haben sie ihn auch hopp genommen. Und die Luise hat geschrien und gebrüllt wie verrückt. Die hat ihn gern. Ob meine Agnes auch so geschrien hätte? Was meinst du Barbara?

\Barbara
Sprich nicht davon.

\Andreas
Ich wollte ihm jedenfalls helfen, dem Jan, und auch der Doktor war dabei, der junge, der Doktor Jonas. Vor dem Kino war es, ihr wisst schon, dem neuen. Jetzt kann man sich denken, weshalb es unter der Erde ist. Na, und da wollten sie auch mir an den Kragen. Ausnahmezustand heißt man das. Und da bin ich noch rasch auf und davon.

\Josef
Ich verstehe aber noch immer kein Wort.

\Barbara
Du hast gesagt, dass es kein Nebel ist?

\Andreas
Es ist kein Nebel, und wenn sie sich noch tausend gelehrte Kommissionen kommen lassen.

\Barbara
Was ist es denn Andreas? Um Gotteswillen, so sprich doch.

\Andreas
Ja wisst ihr es denn wirklich noch nicht. \direction{stößt die Worte mühsam hervor} Gift ist es --- Gas --- Giftgas.

\Josef
Das ist nicht wahr.

\Barbara
Herrgott im Himmel.

\Andreas
In unserer eigenen Fabrik erzeugt. Den Herrschaften ist was ausgekommen. Die wissen selber nimmer aus noch ein. Aber vorbereitet waren sie darauf, könnt ihr mich verstecken, wenn man mich suchen kommt?

\Josef
Du kannst im Schuppen bleiben oder in der Dachkammer. In den Keller kommen die Kinder.

\Barbara
Du Josef, hör zu. Wenn das nicht der Nebel ist, dann ist es ja gar kein Unglück und kein Schicksal und keine Krankheit.

\Josef
Gott hat uns schwer geprüft.

\Andreas
Gott! Gott hat keine Giftgasfabrik.

\Barbara
Du Josef, hör zu, mir fällt noch was ein. Wenn es nicht der Nebel ist, unsere Agnes, die ist uns ja dann gar nicht bloß gestorben.

\Josef
Was denn Barbara, was meinst du damit?

\Barbara
Josef, unsere Agnes, das Kind, das Mädel ist uns ermordet worden.

\Andreas
Ich spreng die Fabrik in die Luft, ich vertilge diese Bestien.

\Josef
Schweigt still, um Gotteswillen, die Kinder nebenan---

\Barbara
Sie haben unsere Agnes vergiftet.

\Andreas
Ihr bei lebendem Leib die Lungen verbrannt.

\Josef
Barmherziger Himmel, was redet ihr da, das kann doch nicht sein.

\Andreas
Geh raus in den Wald und sieh dir das an. Dort ist jeder Grashalm verreckt.

\Barbara
An unseren Weiden wird es heuer keine Kätzchen mehr geben.

\Andreas
Agnes hat diese Kätzchen so gern gehabt.

\Josef
Ich glaub es nicht, ich kann es nicht glauben.

\Andreas
\direction{springt auf} Hört ihr, ein Auto! Sie kommen schon, um mich zu holen.

\Barbara \direction{schiebt ihn zu einer Seitentüre hinaus} Rasch hinaus in den Schuppen und wenn sie herein sind, läufst du schon auf den Dachboden rauf.

\direction{Gleich darauf kommen der \theGeneraldirektor, \theAlexis, die \theHeilsarmeeschwester und drei \theSoldaten mit einer riesigen Kiste}

\Generaldirektor
Wir sind doch hier im Wirtshaus am Rand?

\Josef
Jawohl.

\Alexis
Sie sind der Wirt?

\Josef
Jawohl.

\Alexis
Das ist Ihre Frau?

\Josef
Jawohl. \direction{\theBarbara tritt mit verschränkten Armen in den Hintergrund}

\Generaldirektor
\direction{auf sie zutretend} Liebe Frau, wir bewundern Sie alle. Sie haben in diesen schlimmen Zeiten und unter den schwierigsten Umständen den schönsten und edelsten Mut bewiesen. Ich danke Ihnen.

\Barbara
Ich brauch keinen Dank.

\Generaldirektor
Sie sollen nicht denken, dass wir nicht wissen, was ein Haus voll fremder Kinder für Sie jetzt bedeutet. Und deshalb haben wir Ihnen die brave Schwester hier als Hilfe mitgebracht.

\Barbara
Ich brauch keine Hilfe.

\Heilsarmeeschwester
Liebe Freundin, Sie werden schon noch manche Hilfe brauchen. Weisen Sie mich nicht ab. Ich habe schwerere Arbeit geleistet.

\Alexis
Die Schwester wird die Kinder herrlich versorgen. Sie wird mit ihnen beten und mit ihnen singen.

\Barbara
In meinem Haus wird nicht mehr gebetet.

\Heilsarmeeschwester
Wie?

\Barbara
Und überhaupt nicht mehr gesungen.

\Generaldirektor
Aber liebe Frau---

\Josef
Barbara, was fällt dir denn ein?

\Barbara
In diesem Haus ist kein Platz mehr für Kinder. Ich weiß nicht, wer die Herren sind, aber falls Sie es noch nicht wissen sollten! In diesem Haus ist wer gestorben, den man vergiftet hat, ein junges Mädel, selber noch ein Kind.

\Generaldirektor
Aber Beste, was denken Sie denn. Wir wissen doch alle von Ihrem Unglück.

\Barbara
Das war kein Unglück, Herr.

\Generaldirektor
Wie meinen Sie?

\Barbara
Das war ein Verbrechen.

\Josef
Herr, entschuldigen Sie, sie hat den Tod der Schwester noch nicht überwunden.

\Barbara
Was sprichst du von Tod. Du weißt so gut wie ich%;B es war ein Mord.

\Alexis
So ist das wahnsinnige Gerücht auch schon bis hierher vorgedrungen.

\Generaldirektor
Um Gotteswillen, Sie werden das doch nicht glauben. Hören Sie! Drei Kommissionen von Chemikern und Ärzten haben entschieden, dass es der Nebel ist, nichts anderes als der Nebel. Wir bringen Ihnen die besten Schutzmaßnahmen, wir tun ja, was in unseren Kräften liegt. Vertrauen Sie uns doch. Sehen Sie, hier stehe ich vor Ihnen, der Generaldirektor der ungeheuren Werke---

\Josef
Der Generaldirektor?

\Generaldirektor
Ja, das bin ich.

\Josef
Dort auf dem Tisch liegt ein Zettel für den Generaldirektor. Ein Herr --- ein Mann --- ein Offizier hat ihn hingelegt. Für den Generaldirektor, hat er gesagt.

\direction{\theGeneraldirektor stürzt auf den Zettel zu}.

\Alexis
Wie? Was? Ein Offizier? Das war Brix.Das kann nur der Oberst gewesen sein. \direction{siehtdem Generaldirektor über die Schulter}

\Generaldirektor
Alexis, entsetzlich! Es wardie höchste Zeit. Der Nebel muss ja ineiner halben Stunde schon -

\Alexis
Unmöglich sein, das gibt es nicht.

\Generaldirektor
Brix sagt niemals, was ernicht weiss. Wenn er doch auf uns gewartethätte.

\Alexis
Dann ist jetzt keine Zeit mehr zu verlieren. \direction{zu den Soldaten} Packt die Gasmasken aus. Dann muss die Pumpe gleichaufgestellt werden. Alle Kinder sollensofort in den Keller.\direction{Die Soldaten werfen einen Haufen Gasmasken auf den Tisch}
\Generaldirektor
Ihr müsst alle Türen und Fenster verstopfen. Wo sind denn die Kinder?

\Josef
\direction{macht die Tür zur Kücke auf} Hier. Kommtmal herein.\direction{Ein paar Kinder kommen neugierig undverlegen aus der Küche}

\Junge{5}
Gibts was zu essen?\direction{Da erblicken die Kinder die Gasmaskenauf dem Tisch und stürzen kreischendhinaus}

\Alexis
Das ist ja eine nette Wirtschaft.

\Heilsarmeeschwester
Ich werde mit den Kleinensprechen. \direction{ihnen nach}

\Alexis
\direction{zu den Soldaten} Kommt gleich mit mir, damit wir die Pumpe aufstellen. \direction{zu \theJosef} Führen Sie uns sofort in den Keller. \direction{\theAlexis, \theJosef und die beiden \theSoldaten ab}

\Barbara
Ich brauch keine Gasmasken und keine Pumpe in meinem Haus. Nehmen Sie die Kinder fort von hier.

\Generaldirektor
Aber liebste, beste Frau, Sie werden uns doch jetzt nicht im Stich lassen wollen. Und alles nur wegen eines Gerüchts.

\Barbara
Wenn ein Gericht nicht wahr ist, sperrt man die Leute nicht ein. Wenn ein Gerücht nicht wahr ist, braucht man kein Militär.

\Generaldirektor
Aber das muss doch jetzt ganz gleichgültig sein. Jetzt gilt es Menschenleben zu retten. In Zeiten der Not, da halten doch alle Menschen zusammen.---

\Barbara
Das ist jetzt aber keine Not.

\Generaldirektor
Herr des Himmels, wenn das keine Not sein soll!

\Barbara
Not ist, wenn der Schnee das Dach eindrückt, wenn das Wasser die Mauern wegreißt, wenn die Hitze das Getreide verdorrt, wenn die Krankheit den Menschen frisst. Wenn aber der Mensch selber den Menschen frisst, ihm nichts zum leben lässt,nicht einmal mehr die Luft, das ist nicht Not, einerlei, wen es trifft.

\Generaldirektor
\direction{zurückweichend} Ja, was denn sonst als Not?

\Barbara
Krieg.

\Generaldirektor
Sie werden doch nicht behaupten wollen ---

\Barbara
Dort auf der Ofenbank ist mir das Mädel gelegen, es hat sie verbrannt, inwendig verbrannt, sie hat nach Wasser geschrien,und auch das Wasser hat sie verbrannt. Nehmen Sie die Kinder weg, gleich, sofort, hier ist kein Haus mehr für Kinder.

\Generaldirektor
Aber liebe Frau, Sie, die Sie doch selbst ein Kind erwarten ---

\Barbara
Sie verstehen das nicht \direction{geht zum Fenster und öffnet die innere Scheibe}

\Generaldirektor
Was machen Sie da?

\Barbara
\direction{mit der Hand am Griff des äußeren Fensters} Wenn dieses Kind einmal zur Welt kommt, dann soll keine gute Frau da sein, die ihm Kartoffel kocht und es in einen Keller sperrt, weil draußen überall so ein Nebel ist. Dann soll eine Frau kommen und ein Mann oder vielleicht auch viele Frauen und viele Männer, die ganz was anderes tun, wenn man ihnen die Luft verpesten und die Kinder vergiften will. Nehmen Sie die Kinder fort, oder ich reiße das Fenster auf.
\end{play}

% ---------


\scene{Loyalitäten}
\label{scene:V}
\characterlist{\theClarisse, \theGeneraldirektor, \theAlexis, \theJonas, \theThomsen, \theDiener, \theMelchior}
\setting{Im Herrenzimmer des \theGeneraldirektor{}s. Abends. Der Generaldirektor sitzt bei seinem Schreibtisch, zurückgelehnt in einen Sessel starrt er vor sich hin. Knapp nachdem der Vorhang aufgegangen ist, stürzt \theClarisse in Reisekleidern zur Tür herein.}

\begin{play}

\Clarisse
Paul!

\Generaldirektor
\direction{fährt herum, springt auf} Clarisse, um Himmelswillen! Was fällt dir ein, wo kommst du her?

\Clarisse
Geradewegs aus Paris. Ich hielt es nicht länger aus.

\Generaldirektor
Und die Kinder? Was ist mit ihnen?

\Clarisse
Was soll denn sein?

\Generaldirektor
Sind sie gesund?

\Clarisse
Selbstverständlich. Sonst wäre ich doch nicht abgereist. Du scheinst sehr nervös zu sein, mein armer Paul. Und wie du aussiehst. Bist ja ganz grau im Gesicht. Komm, gib mir wenigstens einen Kuss.

\Generaldirektor
\direction{küsst sie und nimmt ihr den Mantel ab} Ich habe dich doch so gebeten ---

\Clarisse
Aber Paul, du kannst mich doch nichteinfach in die Verbannung schicken, wenn du Sorgen hast. \direction{setzt sich auf das Sofa und zieht ihn neben sich}

\Generaldirektor
Sorgen, das ist nicht das richtige Wort, Clarisse. Ich glaube, ich werde wahnsinnig.

\Clarisse
Ja, was ist denn geschehen?

\Generaldirektor
Frag nicht. Ich weiß es nicht. Ich will es auch nicht wissen.

\Clarisse
Hast du solche Angst vor dem Nebel?

\Generaldirektor
Du hättest nicht zurückkehren dürfen in diese Hölle. Es war mein einziger Trost ---

\Clarisse
Aber Paul, hier ist doch keine Gefahr.

\Generaldirektor
Woher weißt du das?

\Clarisse
Es stand ja in allen Zeitungen. Die Zone hier liegt geschützt vom Wind, es ist sicher, dass der Todesnebel nicht auch in die Stadt dringen kann. Und wenn, so ist man hier gegen alles geschützt.

\Generaldirektor
Stand das auch in den Zeitungen?

\Clarisse
Ja.

\Generaldirektor
Und abgesehen von dem, was in den Zeitungen steht, sag mal Clarisse, du brauchst mich nicht zu schonen, was spricht man draußen in der Welt, was redet man von diesem Todesnebel?

\Clarisse
Mein Gott, man spricht nicht weiter darüber. Es ist eben ein Unglück.

\Generaldirektor
Und das ist alles?

\Clarisse
Ach, es passiert doch jetzt immer soviel. Gestern erst der abgestürzte Ozeanflieger und dann das Erdbeben in Japan.

\Generaldirektor
Mach keine Ausflüchte. Du musst mir die Wahrheit sagen.

\Clarisse
Du bist wirklich in einem entsetzlichen Zustand, Paul. Wie recht hatte ich,zurückzukehren. Ich kann doch nicht in Paris in die Oper gehen und mir Kleider kaufen, wenn du vor Sorgen und Arbeit beinahe zusammenbrichst. Ich will helfen, hörst du, Paul, dir und den andern.

\Generaldirektor
Welchen andern?

\Clarisse
Vor allem braucht man Geld. Ich werde eine großzügige Sammlung veranstalten. Dazu bilde ich ein Aktionskomitee. Wie ich höre, gibt es viele Flüchtlinge hier. Es soll ja ganz Dybern evakuiert sein. Die armen Leute müssen Ausspeisungen haben. Und Kinderhorte.

\Generaldirektor
Du hast ganz falsche Vorstellungen, Clarisse. Du kannst nicht unter die Leute gehen, das wäre das Letzte.

\Clarisse
Aber warum denn?

\Generaldirektor
Weil wir hier den Ausnahme zustand haben. Weil Militär die Flüchtlingslager bewacht.

\Clarisse
Militär?

\Generaldirektor
Es ist wegen der Ordnung, Clarisse, das musst du verstehen. Denn wenn die ganze Welt, die Natur sozusagen, in Unordnung gerät, dann müssen die Menschen wenigstens Ordnung halten, das ist doch klar, dazu gibt es eben ein Militär.

\Clarisse
Aber die Kinder werden doch nicht auch vom Militär in Ordnung gehalten?

\Generaldirektor
Sprich nicht von Kindern, das ist das schlimmste. Alles konnte ich ertragen, nur nicht den grässlichen Transport von gestern. Seither spüre ich erst,dass ich die Nerven verliere.

\Clarisse
Was war denn das für ein Transport?

\Generaldirektor
Wir brachten ein Dutzend Kinder in meinem Auto aus der gefährdeten Zone. Es war die höchste Zeit. Die Frau, bei der sie waren, war irrsinnig geworden, sie wollte sie nicht mehr haben, sie wollte die Fenster aufreißen --- durch die der Nebel --- und auch die Kinder waren irrsinnig. Sie brüllten vor Angst, sie wollten nicht mit uns, wir mussten sie schlagen, binden, Soldaten halfen uns, die armen Würmer fürchteten sich vor den Gasmasken. Wir trugen alle Gasmasken. Stell dir vor, Clarisse, wenn auch du solch eine Gasmaske

\Clarisse
Ich verstehe dich nicht. Wenn es sein muss, so trage ich eben auch eine Gasmaske.

\Generaldirektor
\direction{aufspringend und mit den Füssen stampfend} Ich aber will dich nicht darin sehen, niemals! Clarisse, du musst abreisen, fahr zurück nach Paris zu unseren Kindern, dort gehörst du hin. Ich kann dir das jetzt nicht erklären. Vielleicht kommt noch einmal der Augenblick, dass ich an einem ruhigen Sommerabend in einem ruhigen Zimmer und bei offenen Fenstern darüber sprechen kann --- ich halte die Luft hier nicht lange mehr aus. Ich ersticke.

\direction{sinkt in seinen Schreibtischsessel zusammen, den Kopf in den Händen}

\Clarisse
\direction{tritt auf ihn zu} Paul, du bist krank.

\Generaldirektor
Nein, ich bin nicht krank, ich bin ganz gesund. \direction{Es klopft kurz und \theAlexis kommt in das Zimmer}

\Alexis
Oh, gnädige Frau!

\Clarisse
Guten Abend, lieber Alexis. Sie starren mich ja an wie ein Gespenst. Und dabei kommen Sie ja sicher wieder, wie heißt es doch, in dringenden Angelegenheiten.

\Alexis
Richtig geraten, gnädige Frau. In sehr dringenden Angelegenheiten sogar.

\Clarisse
Nun, dann will ich wie immer, die Herren nicht stören. Auf Wiedersehen Paul, --- ich werde mich umkleiden. Du findest mich später im Wohnzimmer. Auf Wiedersehen, Herr Ingenieur. \direction{ab}.

\Alexis
Warum haben Sie nur nicht rechtzeitigmeinen Rat befolgt. Nun heißt es schleunigst eingreifen. Dieser Kerl, dieser Mensch, dieser Doktor

\Generaldirektor
Von wem sprechen Sie denn?

\Alexis
Von Jonas natürlich. Der Mann war mir gleich nicht geheuer. Er hat so was verstecktes, er denkt sich was bei allem, was man spricht. Allerdings, für verrückt hätte ich ihn nicht auch gehalten. Beidem ist mehr als eine Schraube los. Er wird gemeingefährlich.

\Generaldirektor
Ja, was tut er denn?

\Alexis
Es ist unerklärlich. Und dieser Mensch will ein Arzt sein. Sie wissen doch, er ist Leiter der Tuberkulosenabteilung im Krankenhaus. Nun stellen Sie sich vor, was er treibt: Gestern Nacht ließ er die schwerkranken Kinder auf das offene Dach hinaus schaffen, in den Sälen reißt er die Fenster auf

\Generaldirektor
Wie? Was sagen Sie? Er reißt die Fenster auf?

\Alexis
Eigenhändig, wenn die Schwestern ihm nicht gehorchen wollen. Er lässt den Nebel durch alle Zimmer ziehen, noch ist es ja nicht gefährlich, aber wenn, dann wird es zu spät. Thomsen ist machtlos.

\Generaldirektor
Er öffnet die Fenster. Das ist sehr, sehr merkwürdig. Wissen Sie, Alexis es scheint zwei Arten von Menschen zu geben: Die, die in solchen Zeiten das Fenster öffnen und die, die es schließen.

\Alexis
Sagen Sie lieber gleich, die Wahnsinnigen und die Vernünftigen.

\Generaldirektor
Sind Sie auch sicher, welche die Wahnsinnigen sind?

\Alexis
Herr Generaldirektor!

\Generaldirektor
Lassen Sie es gut sein, Alexis. Übrigens werden Sie Doktor Jonas gleich selber sprechen können, ich bat ihn nämlich mit Doktor Thomsen zu mir. Es handelt sich um seine Unterschrift. Die dritte Kommission braucht seine Unterschrift. Es wäre sehr peinlich, wenn er sich auch diesmal nicht mit seinem Kollegen solidarisch erklären wollte.

\Alexis
Und was wollen Sie dazu tun?

\Generaldirektor
Ich will ihn darum bitten.

\Alexis
Ich würde ihn ins Gefängnis sperren oder ins Irrenhaus.

\Generaldirektor
Man kann doch einen Menschennicht allein seiner Überzeugung wegen ---

\Alexis
Ach was, Überzeugung! Im Ausnahmezustand gibt es keine Überzeugung. In Zeiten der Not ---

\Generaldirektor
Halt Alexis, sprechen Sie nicht weiter. In Zeiten der Not--- ich kann die Worte nicht mehr hören, jeder spricht sie aus, jeder missbraucht sie, keiner weiß, was sie bedeuten sollen. Sagen Sie mir lieber eines, ehe die anderen jetzt kommen ich habe den Oberst übrigens auch zu uns gebeten, vielleicht erscheint er endlich einmal --- also sagen Sie mir jetzt eines, Alexis: was halten Sie von diesem Todesnebel?

\Alexis
Ein Naturereignis.

\Generaldirektor
Das glauben Sie selber nicht.

\Alexis
Herr Generaldirektor, ich, der Leiterder Abteilung A ---\direction{Der \theDiener öffnet die Türe und lässt \theThomsen und \theJonas herein}

\Generaldirektor
\direction{auf \theJonas zugehend} Ich danke Ihnen, dass Sie gekommen sind, Herr Doktor.\direction{\theJonas verbeugt sich stumm. Die andern begrüßen einander schweigend. Man setzt sich so, wie man im zweiten Bild gesessen ist}

\Generaldirektor
Wir sind schon einmal hier so zusammen gesessen, um zu beraten, als Kameraden, wenn Sie mir diesen Ausdruck gestatten. Nun ist die allgemeine Lage um vieles ernster geworden. Jeder von uns muss selbstvergessen auf seinem Posten stehen. Sie auch Herr Doktor Jonas.

\Jonas
Herr Generaldirektor, ich weiß nicht, was Sie meinen. Ich bin Vorstand der Tuberkulosenabteilung in unserem Krankenhaus. Das ist mein Posten. Ich bin Arzt. Als solcher tue ich meine Pflicht.

\Alexis
Betrachten Sie es auch als Ihre Pflicht, Ihre Patienten mutwillig einem anerkannt gefährlichen Nebel auszusetzen?

\Jonas
Kranke Lungen brauchen ständig frische Luft.

\Generaldirektor
Sie leugnen also, dass der Nebel tödliche Keime enthält? Ist dies der Grund, weshalb jede der Kommissionen auf Ihre Unterschrift verzichten muss?

\Jonas
Das Urteil der Kommissionen war falsch. Das wissen wir alle, die wir hier als --- Kameraden beisammen sitzen. Nein, meine Herren, entschuldigen Sie, ich kann das Wort Kameraden nicht für uns gebrauchen. Ich bin nicht Ihr Kamerad. Und deshalb bekommen Sie meine Unterschrift nicht.---

\Generaldirektor
Was soll das heißen?

\Jonas
Das soll heißen, dass ich für mein ganzes Leben den schwersten Kampf der Menschheit aufgenommen habe: den Kampf gegen die Natur. Sie aber, meine Herren, wollen sich mit der Natur gegen die Menschheit verbünden. Die Folgen werden unabsehbar sein. Die Elemente lassen nicht mit sich spielen.

\Alexis
\direction{aufspringend} Sie behaupten also?

\Jonas
\direction{ebenfalls aufspringend} Ich behaupte, dass die Luft vergiftet ist. Durch das Gas, das Ihre Werke erzeugen.

\Alexis
Und wenn das wahr ist, weshalb setzenvSie dann Ihre Kranken diesem Giftgas aus?

\Jonas
Weil ich mir nicht vorstellen kann, dass es einen Teufel gibt, der nicht bloß im Wald und am Fluss, sondern auch hier in der Stadt sein Giftgas loslässt, wenn er weiß, dass kranke Lungen nach Luft lechzen.

\Alexis
Sie machen also ein Experiment?

\Jonas
Auch ein anderer macht ein Experiment. Meines ist ungefährlicher.

\Generaldirektor
Wie meinen Sie das?

\Jonas
Ich experimentiere mit der barmherzigen menschlichen Seele. Jeder andere jedoch mit der unbarmherzigen Natur. Dem Kerl muss das Handwerk gelegt werden.

\Generaldirektor
Sie sprechen in Rätseln.

\Alexis
Das sind ja Wahngebilde.

\Thomsen
Herr Kollege, Sie phantasieren.

\Jonas
Wahngebilde! Herr Ingenieur Alexis, Leiter der Abteilung A! Hier an dieser Stelle haben Sie erklärt, dass Sie sich eine Kugel durch die Schläfen jagen wollten, wenn es nicht der Nebel wäre. Behaupten Sie auch heute noch, dass es eine Unvorsichtigkeit nicht geben kann?

\Alexis
Ich verbürge mich.

\Generaldirektor
Ich könnte die Stunde nicht überleben, in der ich wüsste, dass ich mitschuldig an den vielen Todes opfern bin.

\Jonas
Aber Sie sehen gefasst der Stunde entgegen, in der Sie mitschuldig an vielen hunderttausend Todes opfern sind. Und Sie auch Herr Kollege, der Sie den Todesnebel konstatieren halfen. Und Sie auch Herr Ingenieur, der Sie in der Abteilung A das grauenhafteste Gift erzeugen.

\Alexis
Und selbst wenn es wahr wäre, wo nehmen Sie den Mut zu Ihren hirnrissigen Behauptungen her? Wenn wir Gift erzeugen --- was hiermit noch lange nicht zugegeben wird --- so sind wir eben auch nur gerüstet.

\Jonas
Nur gerüstet. Das genügt. Wo Giftgas erzeugt wird, dort muss es auch einmal ausbrechen, früher oder später, meine Herren, das wissen Sie so gut wie ich.

\Alexis
Ja glauben Sie denn, dass wir unser eigenes Werk nicht beherrschen. Dass unsere chemischen Produkte selbständig und auf eigene Faust über die Erde hinausziehen?

\Jonas
Kann sein, dass Sie Ihre Produkte noch beherrschen. Kann sein, dass diese Produkte noch nicht Sie selbst beherrschen, Sie, den Leiter der Abteilung A und nicht einmal den Generaldirektor. Kann sein, dass die unendliche Macht, die Sie besitzen, Ihnen noch nicht zu Kopf gestiegen ist, weil Sie diese Macht noch gar nicht durchschauen. Aber vielleicht hat ein anderer diese Macht erkannt. Wie soll der Mensch bei Sinnen bleiben, wenn er Herr wird über Leben und Tod einer Welt? Sie verbürgen sich für die Abteilung A. --- Verbürgen Sie sich auch für alle Personen der Abteilung A?

\Alexis
Halten Sie uns denn für Verbrecher?

\Generaldirektor
Wie kommen Sie nur auf diese unerträglichen Ideen?

\Thomsen
Es ist mir völlig unbegreiflich. ---

\Jonas
Man hat mich hergebeten, um mich um meine Unterschrift zu bitten. Ich aber bitte Sie denken Sie doch einmal nach. Denken Sie nach, meine Herren, ehe es zu spät wird.

\Alexis
Sie aber haben noch nicht darüber nachgedacht, dass Ihr Verhalten schwerere Gefahren bedeuten kann als der Nebel. Gegenden Nebel sind wir gerüstet. Besser als Sie wahrscheinlich ahnen. Wenn aber Zweifel und Misstrauen die Bevölkerung vergiften, wenn man uns hasst, uns, die wir alle doch nur helfen wollen, wenn wir Gewalt anwenden müssen, jawohl müssen, um die Leute in Schranken zu halten ---

\Jonas
Sie fürchten also die Menschen mehr als das Gift?

\Alexis
Ja.

\Jonas
Ich nicht.

\Alexis
Wir haben Gasmasken für eine ganze Armee, Sauerstoffpumpen, um eine Stadt damit zu versorgen. Wenn die Bevölkerung sich gefasst und vernünftig verhält, so braucht es nicht zu Katastrophen zu kommen. So vollkommen sind unsere Abwehrmaßnahmen.

\Jonas
Wenn aber die Bevölkerung wissen will,woher sie zu solchen Abwehrmaßnahmen kommen?

\Alexis
So werde ich: Maul halten! kommandieren, verstanden, Herr Doktor? Maul halten. Und ich werde ein paar Kerle an die Wand stellen lassen, damit nicht ganze Massen am Nebel zugrunde gehen.

\Jonas
Am Nebel?

\Alexis
Beweisen Sie, dass es nicht der Nebel ist?

\Jonas
Führen Sie mich in die Abteilung A und ich werde es beweisen. Herr Doktor Thomsen, kommen Sie mit. Untersuchen auch Sie die Abteilung A. Wir werden Schlimmeres finden als Tuberkel und Typhusbazillen.

\Generaldirektor
Sie verlangen Unmögliches. Die Abteilung A ist ein Laboratorium. Dient Versuchszwecken.

\Thomsen
Ich bin nur ein Arzt. Es kommt mir nicht zu, in eine chemische Fabrik einzudringen.

\Jonas
Wer Arzt ist, hat jeder Krankheit nachzugehen, bis er ihre Erreger findet. Und wenn es eine Fabrik gäbe, die systematisch Pestbazillen erzeugt, so wäre der Arzt ein Verräter, der die Pestfälle in der Gegend nicht erkennen wollte.

\Thomsen
Herr Kollege, was erlauben Sie sich.

\Generaldirektor
Um Gotteswillen, Sie werden uns zu den schärfsten Maßnahmen zwingen.

\Alexis
Ich habe es gleich gesagt: Der Mann ist gemeingefährlich.

\Jonas
\direction{nochmals aufspringend} Herr Ingenieur Alexis, Leiter der Abteilung A, wer hat das Giftgas aus dieser Abteilung auf den Weidenweg von Dybern gebracht? Unvorsichtigkeit ausgeschlossen. Ich halte mich an Ihr Wort. Wer ist der Verbrecher? Der Wahnsinnige? Wer wagt es, hier zu experimentieren?

\direction{Der \theDiener kommt herein, verlegenes Schweigen. \theJonas lässt sich erschöpft in seinen Stuhl zurückfallen}

\Diener
\direction{sehr befangen} Herr Oberst Brix.

\Generaldirektor
Ich lasse bitten. \direction{geht auf die Tür zu}

\direction{Herein kommt Jakob \theMelchior. Plump, verlegen und frech auf einmal}

\Generaldirektor
\direction{zurückfahrend} Das ist wohl ein Irrtum ---

\Melchior
Entschuldigen die Herren, aber es ist kein Irrtum. Wenn es auch nicht der Herr Oberst selber ist, aber ich bin sozusagen an seiner Stelle. Melchior ist mein Name. Jakob Melchior, und bin sozusagen dem Herrn Oberst seine rechte Hand und Stütze. Und war ich auch einmal eine militärische Person, Feldwebel im vierten Infanterieregiment. Jawohl. Auch unter dem Herrn Oberst. Und genieße ich jetzt das höchste Vertrauen. Bin ich doch dem Herrn Oberst sein erster und einziger Laborant \direction{verbeugt sich mehrmals gegen jeden einzelnen} Jakob Melchior.

\Generaldirektor
Mensch, was wollen Sie hier?

\Melchior
Ich, oh bitte, ich will nichts. Ich bin ja sozusagen überhaupt nicht hier, sondern nur für den Herrn Oberst hergekommen. An seiner Stelle. Weil der Herr Oberst jetzt sehr beschäftigt ist. Weil der Herr Oberst jetzt große und wichtige Experimente macht.

\Generaldirektor
Dann richten Sie aus, was Ihnen aufgetragen wurde.

\Melchior
\direction{sieht sich um} Gestatten die Herren, dass ich erst ein wenig Platz nehme. Der Dienst bei Oberst Brix ist recht anstrengend. Es ist nicht leicht für einen einfachen Menschen, den ganzen Tag und die ganze Nacht mit einem großen Geist beisammen zu sein. \direction{hat während der letzten Worte einen Stuhl herangezogen und setzt sich ein wenig im Abstand von den anderen}

\Alexis
Kerl, was unterstehen Sie sich!

\Melchior
Ich bin kein Kerl, Herr Ingenieur, nur eine armselige Kreatur, die ihr bescheidenes Wissen in den Dienst der Allgemeinheit stellt. Und da ist es doch nur recht und billig, wenn man mich auch ein bisschen sitzen lässt. Finden Sie nicht auch meine Herren.

\Generaldirektor
Schwatzen Sie nicht so viel,sondern richten Sie lieber aus, was Oberst Brix Ihnen aufgetragen hat.

\Melchior
Der Oberst hat mir gar nichts aufgetragen. Aber als der Herr Generaldirektor seinen Boten schickte, da dachte ich: Wozu die Herren immer auf sitzen lassen. Und da kam ich denn selbst, an Stelle des Oberst sozusagen.

\Alexis
Der Kerl ist ja übergeschnappt.

\Melchior
Wo denken Sie hin, Herr Ingenieur. Übergeschnappt --- das kann sich unsereiner nicht leisten. Was so ein armer Teufel ist wie ich, der bleibt schön vernünftig und tut seine Pflicht. Übergeschnappt das sind höchstens die Feinen und die Ganz gescheiten.

\Generaldirektor
\direction{aufstehend} Lieber Mann, wir haben jetzt für Sie wahrhaftig nicht Zeit. Wir haben Wichtigeres im Kopf.

\Melchior
Wird auch nicht wichtiger sein als der Nebel von Dybern.

\Alexis
Was soll das heißen? Was wissen Sie vom Nebel von Dybern?

\Melchior
Ich weiß nichts, bei Gott, ich weiß fast nichts, ich weiß nur sehr wenig. Aber was der Herr Oberst ist, der weiß sehr viel. Und wenn er selber nicht kommen will, so muss doch ich kommen und was reden, an seiner Stelle sozusagen. Wenn ich auch nur der Jakob Melchior bin.

\Alexis
Der Kerl kann einen rasend machen.

\Generaldirektor
So sprechen Sie doch schon endlich geradeaus. \direction{setzt sich wieder}

\Melchior
Da gibt es kein geradeaus. Ich war viel mit dem Herrn Oberst im Wald, vor allem auf dem gewissen Weidenweg. Das ist schon eine Zeitlang her. Und der Herr Oberst hat immer Versuche gemacht, nichts Genaueres kann ich nicht sagen, ich verstehe ja auch nichts von Chemie. Er hat so mancherlei in die Erde gegraben, ich hab es ihm immer hinreichen müssen, ich weiß natürlich nicht, was es war, ich bin ja auch nicht sein Freund und sein Vertrauter, immer nur sein Diener, sein Laborant. Ich kann also nichts Genaueres gewiss nicht sagen, wenn ich auch

\Jonas
\direction{ist aufgestanden, tritt auf \theMelchior zu, blickt auf ihn herab, die Hände in den Hosentaschen} Sagen Sie mal, was kostet Ihre Mitteilung?

\Melchior
Meine Mitteilung kann nichts kosten. Das Geld ist in Europa nicht aufzutreiben und auch nicht in ganz Amerika dazu, so teuer ist sie und so viel wert. Aber ich weiß doch einiges, wenn ich auch ein einfacher Mensch geblieben bin, und wenn ich will, ich muss bloß wollen, dann gibt es eben Europa nicht mehr.

\Jonas
\direction{wir vorher} Wie hoch ist der Preis?

\Melchior
Da gibt es überhaupt keinen Preis. Ich könnte ja auch in ein anderes Land gehen, was meint Ihr, was bekäme ich da. Hinter der Grenze drüben zum Beispiel. Aber so einer bin ich nicht. Ich bin ein Ehrenmann und nicht mehr ganz jung, da will man gerne sein kleines Haus und ein paar Felder und vielleicht noch eine Garage für den Wagen, einen Obstgarten, eine Schweinezucht wäre auch nicht schlecht, und noch ein paar Baracken für das Gesinde, ein bisschen was Sicheres in der Bank, man will doch heiraten und auch die Kinder was werden lassen, käme dazu noch eine kleine Fabrik ---

\Generaldirektor
Schweigen Sie. Ich wünsche Ihre Erpressungsversuche nicht länger anzuhören.

\Alexis
Das ist ja unmöglich. Ich suche persönlich den Oberst auf.

\Melchior
\direction{steht auf, sehr gekränkt} Ich bin kein Erpresser, und überhaupt, ich lasse mich nicht beleidigen. Und was den Herrn Oberst betrifft, den kann der Herr Ingenieur jetzt lange suchen. Der rennt herum in Wald und Feld, der ist, mit Verlaub zusagen, wirklich übergeschnappt. Der will die ganze Welt kaputt machen und hat da bei nicht einmal was davon. Der wartet darauf, dass die Sonne kommt oder der Frost oder Gott weiß was noch, damit dann doch nicht alle Menschen sterben, und hat sogar auch da nichts davon. Ich aber will mein bescheidenes Auskommen haben

\Alexis
\direction{springt auf ihn zu und schüttelt ihn} Jetzt sprichst du deutsch, Kanaille, oder---

\Melchior
\direction{tritt zurück} Ich habe den Herren nichts mehr zu sagen. Ich bin ein einfacher Mensch und da denk ich mir: Schad' um die Welt, wenn sie kaputt geht, und dann doch nicht ein einziger was davon hat. Da ist doch besser, sie bleibt und einer hat was davon, dass sie bleibt. Und ist ja möglich, dass auch andere noch meiner Meinung sind. Ich hab nichts gesagt, meine Herren, und ich bin ja ohnehin sozusagen nur für den Oberst gekommen, auszurichten, dass der Herr Oberst keine Zeit nicht hat. Gute Nacht allerseits. \direction{ab}

\Alexis
Da soll doch der Donner ---

\Generaldirektor
Brix muss ja wahnsinnig sein, so einen Kerl zu halten.

\Jonas
Brix ist überhaupt lange schon wahnsinnig.

\Thomsen
Man sollte den Mann nicht so ohne weiteres wieder fortgehen lassen. 

\direction{grelles Telefonklingeln}


\Generaldirektor
Hallo, jawohl --- 

ja, ja, ich bines --- 

Wie, wer? --- 

Brix, Oberst Brix --- 

Herr des Himmels, dass Sie endlich zum Vorscheinkommen. Ich habe Ihnen ---

wie, was sagen Sie da das ist ja wunderbar Mond und Kälte, frostklare Nacht der Nebel ist von Dybern gewichen ---

wo sind Sie denn? ---

In Dybern nicht mehr ---

Um Gotteswillen, das kann doch nicht wahr sein ---

das ist jafurchtbar, das ist unabsehbar ---

welche Frau, wie heißt die Frau also gut, in fünf Minuten vor der Fabrik ---

höchste Zeit, dass Sie kommen.

\direction{legt den Hörer ab, vollständig verstört}

Meine Herren, der Nebel ist von Dybern gewichen. Aber das Kino brennt, --- die Leute rasen, haben es selber angesteckt. Allen voran die Frau vom Wirtshaus am Rand. Man zieht los gegen das ganze Werk. Es ist nicht auszudenken.

\Alexis
Wir müssen die Verteidigung selbst in die Hand nehmen. Noch verfügen wir über das Militär. Tränengas und wenn es sein muss, ein paar Salven.

\Generaldirektor
Kommen Sie Alexis, Brix scheint endlich Vernunft anzunehmen. In ein paar Minuten ist er in der Fabrik. Er weiß ja am besten, worum es geht. Kommen Sie rasch.

\direction{zu \theJonas und \theThomsen} Wir sehen und später, meine Herren. Hoffentlich. 

\direction{ab mit \theAlexis}


\Thomsen
\direction{steht sehr langsam auf} Ich gehe auf meinen Posten in das Krankenhaus. Kommen Sie mit, Herr Kollege? \direction{Da \theJonas sich nicht rührt} Leben Sie wohl, Herr Kollege.

\direction{Zögert einen Augenblick, tritt dann auf ihn zu und hält ihm die Hand hin. \theJonas tut, als bemerkte er es nicht. \theThomsen geht langsam auf die Tür zu, bleibt dann nochmals stehen und wendet sich um}

Meine Frau hat morgen eine Gallenblasenoperation. Mein einer Sohn ist ein lebenslänglicher Krüppel, der andere soll mein Nachfolger werden. Sie sind jung und mutig, Herr Kollege. Ich beneide Sie, Herr Kollege.

\Jonas \direction{ohne ihn anzusehen, in die leere Luft hinein} Gehen Sie auf Ihren Posten, Herr Kollege.
\end{play}

% ---------


\act{Zeiter Akt}

\scene{Irrungen}
\label{scene:VI}
\characterlist{
	\theLuise, \theBarbara,
	\quad \theGregor, \theJosef,
	\quad \theGeneraldirektor, \theClarisse, \theAlexis,
	\quad \theErsterMann, \theZweiterMann, \theJonas,
	\quad \theSalwin, \theHeilsarmeeschwester,
	\quad \theBrix, \theMelchior
}
\setting{Visionär und traumhaft zu spielen. Der ausgebrannte Vorraum des Kinos, der in ein offenes Feld übergeht. Ein paar verkohlte Überreste der Mauern, rechts wie ein schwarzes eisernes Gerüst das Skelett der Sauerstoffpumpe. Dicke Rauchschwaden ziehen über die Bühne.}

\subscene
\vspace{\baselineskip}
\setting{Von links kommen mit Tüchern vor den Augen \theBarbara und \theLuise. Sie scheinen beide zu weinen. In der Mitte der Bühne bleiben sie einen Augenblick wie erschrocken vor dem Pumpenskelett stehen, dann wenden sie sich dem Hintergrund zu, aber plötzlich reißt \theLuise \theBarbara zurück.}

\begin{play}

\Luise
Pass auf, Barbara, hier sind Stufen.

\Barbara
Wir können auch über die Stufen gehen.

\Luise
Aber, Barbara, wo willst du denn hin?

\Barbara
Ich weiß es nicht. Ist ja auch einerlei.

\Luise
\direction{zieht \theBarbara neben sich auf ein Stück Mauer} Komm, setz dich jetzt. Ich kann nicht mehr weiter. Und hierher laufen sie uns ja doch nicht nach. \direction{sitzt mit dem Tuch vor den Augen}

\Barbara
Arme Luise, du solltest nicht so weinen.

\Luise
Ich weine doch nicht, Barbara. Es sind ja nur Tränen. Ich weine wirklich nicht, ich kann gar nicht weinen. Aber die Augen schwimmen mir fort.

\Barbara
Es ist ja alles ganz einerlei.

\Luise
Jan im Gefängnis und Andreas tot. Ich sah ihn noch zusammenbrechen. Dann sind sie über ihn getrampelt. Du weinst ja selber, Barbara.

\Barbara
\direction{mit den Augen in ihrer Schürze} Meinst du wirklich, dass ich weine?

\Luise
Ach, Barbara, vielleicht weine ich auch. Die Tränen hören gar nicht auf.

\Barbara
Vielleicht weinen wir beide.

\Luise
Und alles ist grau, Barbara. \direction{greift nach ihrer Hand} Deine Hand ist ganz nass.

\Barbara
Tränen.

\Luise
\direction{blickt um sich} Barbara, ich sehe nichts als Wolken.

\Barbara
Das ist Nebel.

\Luise
Das ist Rauch.

\Barbara
Das ist Gas.

\Luise
Du denkst doch nicht wirklich.

\Barbara
Es ist Gas, Luise. Aber leider nicht das richtige. Es lässt uns nicht sterben, sondern nur weinen. Ja, wenn wir zu dem andern, dem richtigen vorgedrungen wären, dann brauchten wir jetzt nicht mehr weinen. Dann gäbe es keine Tränen mehr.

\Luise
Ich verstehe dich nicht.

\Barbara
\direction{mit dem Blick nach oben} Die ganze Fabrik erstorben, die Maschinen selbst wären erstickt. Alle Häuser verätzt, kein Atemzug in der ganzen Stadt. Das Land ringsum tot, wer noch lebt, wirft eine Fackel darauf, Feuer. Gibt es noch jemanden, der die Glocken läuten lässt, die Glocken läuten von selbst, der Himmel glüht über der Erde. \direction{zusammen sinkend, die Hände vor den Augen} So aber können wir nichts als weinen.

\Luise
Barbara, sprich nicht so. Das alles ist doch nie unsere Absicht gewesen. Wir alle wollten das Gas nur finden. Wir wollten uns seiner bemächtigen. Damit es nie mehr ausbrechen kann.---

\Barbara
Es ist ja nur, weil sie so gar nicht wissen, was sie tun. Man muss den Menschen zeigen, was sie tun. Denn vorher merken sie es nicht. Du hast ja selbst in den Gifthöhlen gearbeitet, Luise.

\Luise
\direction{steht auf} Komm, Barbara, komm, wir müssen nach Hause gehen.

\Barbara
Da war einer bei mir, ein großer Herr, der Generaldirektor. Aber auch er hat mich nicht verstanden. Bei lebendem Leib hat es das Mädel verbrannt. Auch das Wasser hat sie verbrannt. Da hilft es nichts, wenn man die Fenster schließt, so lange draußen noch so ein Nebel ist.

\Luise
Schau, Barbara, der Nebel hebt sich dort. Wir wollen weiter, wir können ja hier nicht bleiben. Du bist müde, Barbara.

\Barbara
Ich bin nicht müde.

\Luise
Du bist krank. Du solltest schlafen. \direction{legt die Hand vor die Augen}

\Barbara
Jetzt weinst du wirklich, Luise. \direction{eine dichte Nebelwolke hüllt sie langsam ein}

\Luise
So komm doch schon. \direction{zieht sie an der Hand empor}

\direction{kurzes klägliches Kinderwimmern}

\Luise
Was ist das? Ein Kind.

\Barbara
\direction{streckt beide Arme aus, geht wie gezogen ein paar Schritte nach hinten}

\Luise
Barbara, du wirst über die Stufen fallen.

\direction{\theBarbara verschwindet mit \theLuise im Hintergrund im Nebel.}


\end{play}


\subscene
\vspace{\baselineskip}
\setting{Von links kommen \theJosef und \theGregor}

\begin{play}

\Gregor
\direction{zieht \theJosef an der Hand} Komm nur, komm, sie sind sicher nach Hause gegangen.

\Josef
Schweig still. Hat da nicht ein Kind geweint?

\Gregor
Was fällt dir ein. Hier gibt es keine Menschenseele weit und breit.

\Josef
Und dabei ist mir, als wären überall Leute. In jedem Nebelschwaden. Man braucht nur die Hand auszustrecken. Pst, Gregor, pass auf. Sind das nicht Stimmen?

\Gregor
Ich höre nichts.

\Josef
Dass man sogar nichts sehen kann. Sind deine Augen auch so nass? Wer weiß, vielleicht ist Barbara dicht neben uns. Soll ich sie rufen?

\Gregor
Um Gotteswillen, nein. Es könnten ja doch auch Soldaten hier stecken. Und Luise hat Barbara längst schon nach Hause gebracht. Die ist bestimmt nicht zurückgeblieben.

\Josef
\direction{sieht um sich} Wo sind wir denn? \direction{merkt die Pumpe} Und was ist das für ein schwarzes Gespenst?

\Gregor
Mir ist, als wäre ich schon einmal hier gewesen. Wart mal, das wird doch nicht das Kino sein. Dann ist dort hinten die Treppe --- stimmt. Das haben sie jetzt gründlich ausgebrannt. Und dort steht dem Jan seine Pumpe.

\Josef
Andreas soll der erste gewesen sein, der das Kino angezündet hat. Nicht wahr, du weißt es auch. Barbara ist ihm dann bloß gefolgt mit den anderen.

\Gregor
Nun ist er tot. Erschossen.

\Josef
Gregor, meinst du, dass die Soldaten auch Barbara ---

\Gregor
Komm, gehen wir zu ihr nach Hause. Eine schwangere Frau rührt keiner an. So gottlos ist nicht einmal der Teufel. Sie trägt doch das Leben.

\Josef
Wenn sie nicht schwanger wäre, so wäre sie auch nicht so geworden. Da ist was in ihr, was ich gar nicht verstehen kann. Sie ist überhaupt nimmer meine Frau.

\Gregor
Du darfst nicht weinen, Josef, du musst ganz ruhig sein und sie zu Bett bringen. Was soll denn sonst aus dem Kindchen werden.

\Josef
Mir ist, als wäre das schon längst auf der Welt. Vielleicht rennt es herum und ruft nach uns. Gregor, ich kann kein Kind mehr weinen hören. Ich habe solche Angst um das unsrige. 

\direction{fährt zusammen} Hast dugehört? Schon wieder. Da weint ein Kind im Nebel.

\Gregor
Du irrst bestimmt, ich habe nichts gehört.

\Josef
Es war ganz deutlich. Und auch ganz nah. \direction{man hört eine Autohupe}

\Gregor
\direction{zieht Josef nach links} Rasch, rasch, ein Auto. Man kann nicht wissen; besser, dass man uns hier nicht erblickt.

\Josef
Aber wenn vielleicht doch ein Kind ---

\direction{verschwindet mit \theGregor links an der
Seite im Nebel}

\end{play}


\subscene
\vspace{\baselineskip}
\setting{Gleich darauf kommen von rechts in Automänteln der \theGeneraldirektor, \theClarisse und \theAlexis}

\begin{play}

\Generaldirektor
Nein, auch hier ist alles wie ausgestorben.

\Alexis
Die Bande hat sich nach allen Seiten zerstreut.

\Clarisse
Sagt mal, wo sind wir denn eigentlich?

\Generaldirektor
Das soll wohl das Kino sein.

\Alexis
Eine Niedertracht sondergleichen. Ich werde nicht ruhen, bis die Schuldigen gefunden sind.

\Generaldirektor
So lassen Sie doch. Ein Todesopfer ist ohnehin schon zu beklagen. Und wie viele verletzt wurden, weiß man nicht.

\Alexis
Wir können doch nicht warten, bis sie wirklich die Fabrik erstürmen. Wir brauchen Militär, noch viel, viel mehr Militär. Ich sagte es ja gleich. Hier an dieser Stelle warnte ich Sie zum ersten Mal. Von hier ging damals das Übel aus. Hätte der Kerl nicht die Pumpe entdeckt --- und jetzt, sehen Sie doch, das sind die Überreste. Es ist zum Verzweifeln.

\Clarisse
Was für Überreste?

\Alexis
Das ist alles, was von unserem großen schönen herrlichen Kino blieb. Das Skelett einer Pumpe.

\Clarisse
Ach, interessant. Und das kann man zu gar nichts mehr brauchen? \direction{wischt sich die Augen.}

\Generaldirektor
Was hast du, Clarisse? Du wirst doch nicht weinen.

\Clarisse
Wo denkst du hin. Das ist doch nur das dumme Tränengas.

\Alexis
Wir müssen tun, was in unseren Kräften steht, um uns der Frau von Wirtshaus am Rand zu bemächtigen.

\Generaldirektor
Sie soll doch jeden Tag ein Kind bekommen.

\Alexis
Dann sperren wir sie eben in ein Krankenhaus. Jedenfalls muss sie aus der Gegend verschwinden. Alles Übel kommt vom Wirtshaus am Rand.

\Generaldirektor
Meinen Sie nicht, Alexis, dass das Übel von ganz wo anders herkommt. Nicht vom Kino und nicht vom Wirtshaus am Rand.

\Alexis
Nein, das meine ich nicht. Was wir erzeugt, das können wir auch beherrschen. Aber den Pöbel beherrschen wir nicht. Außer ---

\Clarisse
\direction{hat sich mit dem Taschentuch vor den Augen unter die Pumpe gesetzt} Ich kann hier nicht bleiben. Mich schmerze die Augen.

\Generaldirektor
\direction{wischt sich die Augen} Sonderbar. Die Welt ist zum Weinen. Und wir vergiessen künstliche Tränen. Gehen wir.

\Alexis
Ich hätte gerne noch auf den Oberst gewartet. Er wollte doch kommen.

\Generaldirektor
Das kann noch Stunden dauern.

\Alexis
Aber er kommt bestimmt. Er will den Schaden besichtigen. Ach, Brix ist doch ein wahrhaft großer Mann. Wer ihn heute gesehen hat, so ruhig, so umsichtig, so besonnen --- wer weiß, ob wir das Werk sonst gehalten hätten.

\Generaldirektor
Er hat uns gerettet.

\Alexis
Wenn Brix uns weiter hilft, wird alles gut.

\direction{wieder das Wimmern}

\Clarisse
\direction{aufblickend} Was ist denn das? Weint hier ein Kind?

\Alexis
Keine Spur. Das ist nur die Pumpe.

\Generaldirektor
Meinen Sie wirklich?

\Alexis
Ich höre es. Es knirscht oben im Wind.

\Clarisse
\direction{steht auf} Sonderbar, dass eine Pumpe wimmern kann.

\Generaldirektor
Man sollte das Feld hier durchsuchen lassen.

\Clarisse
Ich will nach Hause. Der Nebel wird so gelb. Paul sieh doch, wie er sich näher wälzt. \direction{nimmt ihn an der Hand, zieht ihn nach rechts} Komm, komm, ich fürchte mich.

\Alexis
\direction{ihnen nach} Keine Angst, gnädige Frau, die Gasmasken liegen im Wagen.

\direction{sie verschwinden rechts im Nebel}

\end{play}


\subscene
\vspace{\baselineskip}
\setting{Gleich darauf hört man den Motor des Autos. Die beiden Männer aus Szene \ref{scene:III} huschen von rechts her auf die Bühne.}

\begin{play}

\ErsterMann
Pst. Gut, dass sie fort sind.

\ZweiterMann
Was hast du denn da aus dem Auto gezogen?

\ErsterMann
\direction{hält eine Gasmaske vor sich hin} Es lag vorn auf dem Sitz. Ich hielt es für eine Tasche. Pfui Teufel.

\ZweiterMann
Was ist es denn?

\ErsterMann
Eine Gasmaske.

\ZweiterMann
Ein feiner Griff!

\ErsterMann
Halt's Maul, wir werden sie vielleicht noch brauchen können.

\direction{schnuppert} Merkst du nichts?

\ZweiterMann
Was soll ich denn merken.

\ErsterMann
\direction{schnuppert} Der Nebel gefällt mir nicht.

\ZweiterMann
Mir rinnen die Augen.

\ErsterMann
Es riecht nach Senf.

\ZweiterMann
Ist mir alles eins. Ich kann nicht mehr weiter. \direction{ist nach hinten getreten} Was ist denn das? Da führt ja eine Treppe hinunter.

\ErsterMann
Schafskopf. Weißt du denn nicht, wo wir sind.

\ZweiterMann
\direction{setzt sich auf die oberste Stufe und zieht Brot und Wurst aus der Tasche} Nein.

\ErsterMann
Das ist doch das Kino, du Idiot. Das Kino, aus dem wir ausgebrochen sind. Erinnerst dich noch? \direction{setzt sich neben ihn} Die Heilsammeeschwester mit ihrem ewigen, ,,ihr lieben Leute``. War das eine Gans. Was hast du denn da?

\ZweiterMann
Lass sein. Ist für mich allein nicht genug.

\ErsterMann
Gib her oder --- hast es doch selber geklaut \direction{stürzt sich auf ihn}

\ZweiterMann
Au, au, lass los, du gemeines Schwein.

\direction{sie balgen. Das Wimmern. Die Beiden fahren auseinander}

\ZweiterMann
Was war denn das?

\ErsterMann
Da weint wo ein Kind.

\ZweiterMann
Was geht das mich an. Sind wir ohnehin alle kaputt.

\ErsterMann
Du, gib her. Die Hälfte wenigstens.

\ZweiterMann
Da nimm das Brot.

\ErsterMann
Das ist zu wenig.

\direction{\theJonas stürzt von links auf die Bühne}

\Jonas
Wo ist es? Wo? Habt ihr gehört? Da weint ein Kind.

\ZweiterMann
Wird nicht so gefährlich sein.

\Jonas
Aber vielleicht hat es sich verlaufen. Erstickt im Nebel.

\ErsterMann
Krieg ist Krieg.

\ZweiterMann
Da frisst eben einer den andern.

\Jonas
Wie? Was meint ihr?

\ErsterMann
Wenn die Herrschaften uns das Land verpesten, können wir nicht die kleinen Kinder retten.

\Jonas
Wo war es? Hier? Dort? Von welcher Seite ist es gekommen?

\ErsterMann
\direction{kauend, zeigt auf die Pumpe} Von dort oben.

\ZweiterMann
Vielleicht war es nur das schwarze Skelett.

\Jonas
Vielleicht. \direction{wirft mit der Taschenlampe einen grellen Lichtstrahl auf die Pumpe}

oder \direction{wirft einen Lichtstrahlnach hinten, gegen die Treppe, die beiden Männer sind verschwunden} Hallo, wo seid ihr?

\direction{geht nach hinten} Hört ihr denn nicht? \direction{leuchtet um sich herum grell ins Publikum hinein, verschwindet im Nebel}

\end{play}


\subscene
\vspace{\baselineskip}
\setting{Von rechts kommt \theSalwin. Er zieht die \theHeilsarmeeschwester hinter sich her}

\begin{play}

\Salwin
Kommen Sie nur meine Liebe. Jetzt können Sie keine Seelen mehr retten. Ein verfluchtes Pech, dass das Motorrad eine Panne hat. Aber ich kenne den Weg. Bin ihn oft genug gefahren. So kommen Sie doch.

\Heilsarmeeschwester
\direction{stehen bleibend} Nein, nein, ich kann nicht, ich kann wirklich nicht. Es ist ja Fahnenflucht, jetzt meinen Posten zu verlassen. Die armen Leute brauchen mich.

\Salwin
Die armen Leute werden Sie erschlagen. Und zwar bei der nächsten Gelegenheit. Die armen Leute sind sehr gereizt gegen Sie. Sie haben keine Lust mehr zu beten.

\Heilsarmeeschwester
Aber warum denn? Ich verstehe das nicht. Ich habe mir die Augen ausgeweint.

\Salwin
Alles nur Tränengas, meine Liebe.

\Heilsarmeeschwester
Ich tat ja, was in meinen schwachen Kräften stand.

\Salwin
Das ist ganz gleichgültig. Kommen Sie endlich. Wir können hier doch nicht stehen bleiben. Was ist denn das für ein Gespenst?

\Heilsarmeeschwester
Wissen Sie nicht, wo wir sind? Das ist doch das Kino. Die Treppe hinten führt in den Zuschauerraum. Dort sang ich meine schönsten Chorale. Nun ist alles ausgebrannt.

\Salwin
Sagen Sie mal --- meinen Sie, dass auch das Telefon

\Heilsarmeeschwester
\direction{mit gefalteten Händen} Lieber Gott, hilf mir, dass ich stark und mutig bleibe, lieber Gott, lass mich nur jetzt nicht im Stich.

\Salwin
Ich muss so bald als möglich in die Stadt zurück. Natürlich in einem Bogen und nicht wieder durch das verfluchte Gas.

\Heilsarmeeschwester
Hören Sie, hier weint ein Kind.

\Salwin
Sie phantasieren, Vorwärts, ich muss in die Stadt.

\Heilsarmeeschwester
Es war aber doch genauso, als ob ---

\Salwin
Unsinn, nur keine Halluzinationen. Die Wirklichkeit ist abscheulich genug \direction{zieht die \theHeilsarmeeschwester nach rechts hin}

\Heilsarmeeschwester
Sind Sie ganz überzeugt? Passen Sie auf. Es ist doch deutlich.

\Salwin
Ich höre nichts. Es klingt nicht anders, als eine schlecht geölte Tür. Ich will nichts hören.

\Heilsarmeeschwester
Sie können doch ein Kind, ein Menschenwesen nicht im Stich lassen wollen.

\Salwin
Bilden Sie sich nicht ein, dass Sie es retten können. Lächerlich. Bei diesem Nebel müssen wir froh sein, wenn wir selber nach hause finden.

\Heilsarmeeschwester
Und wenn es erstickt?

\Salwin
Umso besser für das arme Wurm. Mir scheint gar, ich beginne auch schon zu flennen. \direction{wischt sich die Augen} Kommen Sie, mir reißt die Geduld.

\Heilsarmeeschwester
Aber es wäre doch meine Pflicht---

\Salwin
Ich pfeife auf Ihre Pflicht. Und ich will Sie hier nicht zurück lassen. Weil ich Sie nun einmal zufällig kenne. Und weil ich Ihnen auf dem Weg hierher begegnet bin. Gott weiß, wer sonst noch im Nebel steckt und zugrunde geht. Ich sah die Leute nicht. Ich will sie nicht sehen.

\Heilsarmeeschwester
\direction{sieht schwankend nach oben} Vielleicht kommt es nur von dieser hohen Pumpe. Vielleicht ist es wirklich kein Kind.

\Salwin
Natürlich, hören Sie nicht, es kommt von der Pumpe. Und überhaupt, dieser Wind. Der Nebel wickelt sich einem um die Beine. Vorwärts Schwester.

\direction{Die Nebelschwaden ziehen bewegt durcheinander, mischen sich, ballen sich, Salwin und die Schwester verschwinden im Nebel, die Pumpe ist nicht mehr zu sehen.}

\end{play}


\subscene
\vspace{\baselineskip}
\setting{Graues Wogen. Plötzlich teilt sich der Nebel, ein nicht sehr großer Fleck vorne wird frei. \theBrix, in Gasmaske, liegt ausgestreckt auf den Ellbogen gestützt. Vor ihm hockt \theMelchior, der sich die Augen wischt}.

\begin{play}

\Brix
Wie? Was meinst du? Was redest du da?

\Melchior
Herr Oberst können die Maske schon abnehmen. Es sind nur Tränen. Tränen schaden nicht.

\Brix
Sag mal, Melchior, hast du schon lange nicht geweint?

\Melchior
Ich --- ich weiß nicht, Herr Oberst. Man weint so manches mal, an Geburtstagen und bei hohen Festlichkeiten, wie sich's halt trifft.

\Brix
Hast du Kinder, Melchior?

\Melchior
Herr Oberst, ich bin nicht verheiratet.

\Brix
Aber du möchtest wohl Kinder haben?

\Melchior
Herr Oberst, wenn es Gott gibt und ich hab außerdem noch ein gutes Auskommen.

\Brix
Und ein kleines Haus und ein paar Felder und eine Garage für den Wagen, einen Obstgarten und eine Schweinezucht -- was wendest du dich denn auf einmal ab?

\Melchior
Ich --- es sind nur die Tränen.

\Brix
Sieh doch, wie gelb der Nebel ist. Der bringt noch ganz was anderes als Tränen. Das weißt du sehr gut. Das wissen auch andere, Melchior.

\Melchior
Herr Oberst, ich ich hab mein Lebtag nichts gesagt. Hab immer nur dem Herrn Oberst gedient und ihm helfen wollen.

\Brix
Hast dir viel davon versprochen, Melchior. So viel Geld gibt's in Europa nicht und in ganz Amerika. Und wer weiß, ob nicht die drüben, jenseits der Grenze ---

\Melchior
Herr Oberst, das alles sind doch nur Verleumdungen.

\Brix
Rühr dich nicht! Ich hab dich einmal auf dem Rücken geschleppt, im großen Krieg, drei Stunden lang, ganz blau warst du, der einzige, der es überstanden hat. Ich ließ dir wieder die Luft einpumpen.

\Melchior
Herr Oberst, meine ewige Dankbarkeit!

\Brix
Halt's Maul. Deine Dankbarkeit hab ich nie verlangt. Aber wenn einer mal so was mitgemacht hat, wenn einer selber beinahe verreckt ist am Gas, da sollte man das nicht für möglich halten.

\Melchior
Herr Oberst, ich weiß ja nicht, was die Leute sagen ---

\Brix
Dann wart es ab. Es wird bald einer kommen, der dir's erzählt.

\Melchior
Wollen wir hier dann noch lange bleiben?

\Brix
So lang, bis er kommt. Er ist uns auf der Fährte, Melchior. Er hat mir deine Geheimnisse verraten.
Wenn wir es abwarten, kann er dir auch meine Geheimnisse erzählen.

\Melchior
Beim lebendigen Gott, ich schwöre es bei meiner Seele Seligkeit, ich weiß gar nicht, von wem der Herr Oberst spricht.

\Brix
Bist ein dummer Kerl, Melchior. Da gehst du hin und willst verkaufen, was ohnehin die ganze Fabrik schon kennt. Die wissen selber, wie man die Welt verpestet, sogar dieser Schwachkopf von einem Alexis bringt das zusammen. Aber das andere --- wissen Sie nicht. Setz dich nieder Melchior, sofort. Es hilft dir nichts mehr.

\Melchior
Es ist nur --- weil der Nebel wird so sonderbar rot.

\direction{rötlicher Schimmer im Nebel}

\Brix
Das ist die Sonne, Melchior. Und du hast das Patent verkaufen wollen. Bist ein dummer Kerl. Willst ein Häuschen haben mit Obstgarten und Schweinezucht und auch noch etwas Geld in der Bank ja, sieh mich nur an --- und rings um dich geht die Welt kaputt. Glaubst, dass dein Häuschen ganz allein übrig bleibt. Ach Melchior, du bist ein Idiot.

\Melchior
Herr Oberst, melde gehorsamst, die Tränen blenden mich, es brennt in der Brust. Herr Oberst, mir wird plötzlich ganz schlecht.

\Brix
Hab keine Angst, dir wird nicht lange mehr schlecht sein.

\Melchior
Und irgendwo --- ich hab ein Kindchen wimmern gehört.

\Brix
Wenn der Nebel kommt, du weißt schon, Melchior, der Nebel, den du verkaufen wolltest, dann wimmert kein Kindchen mehr, weit und breit.

\Melchior
Herr Oberst, ich möchte nach Haus', Herr--- Oberst, ich ich will nicht sterben. --- \direction{greift nach seiner Gasmaske}

\direction{ziemlich starkes rötliches Licht}

\Brix
Die Maske wird dir nicht viel helfen können. Nur die Sonne hilft, wenn sie den Nebel durchdringt. Warten wir es ab.

\Melchior
Aber wenn wir rasch noch fliehen. \direction{springt auf}

\Brix
Stillgestanden! Kehrt Euch! Niedersetzen!! \direction{\theMelchior gehorcht automatisch} Es wird ein Mann kommen, ein Mann, der alles erfahren soll. Ich hab ihn herbestellt. Er heißt Jonas.

\Melchior
Wenn aber der Mann auch im Nebel stecken bleibt? Der kommt bis morgen nicht, der findet vielleicht überhaupt nimmer her.

\Brix
Wie? Was redest du da?

\Melchior
Ich mein ja bloß, wenn's uns erwischt, so kann es ihn doch auch erwischen.

\Brix
\direction{springt auf} Halts Maul, Kanaille!

\Melchior
\direction{salutiert} Mit Verlaub, Herr Oberst,ich hab nichts gesagt.

\direction{Wimmern}

\Melchior
Da, schon wieder. Was wimmert denn da?

\Brix
\direction{hebt den Kopf} Es kommt von oben.

\Stimme
\direction{Männerstimme, lang gezogen} Hilfe!

\Melchior
\direction{nach hinten zeigend} Das kommt von dort unten.

\Brix
\direction{macht einen Schritt nach hinten, fährt aber zurück, denn im selben Augenblick ruft eine Frauenstimme}

\Stimme
Ich brenne!

\direction{Nebel glühend rot}

\Melchior
\direction{springt auf, packt seine Gasmaske, stülpt sie auf den Kopf} Gott sei unsgnädig!

\Brix
\direction{stellt sich ihm in den Weg} Halt! Du bleibst da. Das ist das Ende. Ich wartete auf einen Menschen. Nun bin ich allein mit einer Kreatur. Knie nieder, Melchior, knie nieder!

\Melchior
\direction{gehorcht zitternd}

\Brix
\direction{hin und her jagend} Nein, ich habe es nicht gewollt. Es war ein Experiment, verstehst du Melchior, ein Experiment. Am Fluss drüben, wo kein Mensch im Winter hinüber kommt. Es war doch nur an einer einzigen Stelle, du weißt es ja selbst. Und dann sollte die Sonnenkraft aus meiner Phiole. Ich bin doch kein Mörder, Melchior. Aber ich bin auch nicht Gott. Hast du denn nicht bemerkt, dass ich den Nebel vernichten wollte?

\Melchior
\direction{mit aufgehobenen Händen} Herr Oberst, mir wird ganz heiß im Schlund.

\Brix
Mit den Mächten des Himmels lässt sich nicht kämpfen. Sieh doch die Sonne. Gottes Sonne wird wieder blass. Sie dringt nicht durch. Sie allein könnte noch alles erretten. Melchior, heb' deine Hände nochmal, bete Kanaille, bete, dass die Sonne kommt, helle, strahlende, blendende Sonne, bete, du Schurke, du Judas, du Verräter, bete zu Gott, zum Himmel, zum Licht, bete

\Melchior
\direction{automatisch} Vater unser, der du bist in dem Himmel

\end{play}

% ---------


\scene{Zusammenbruch}
\label{scene:VII}
\characterlist{}
\setting{Wirtsstube wie in Szene \ref{scene:I} und \ref{scene:III}. Helles, weißes Sonnenlicht fällt in breiten Streifen durch das Fenster. Die \theHeilsarmeeschwester steht über eine offene Tischlade gebeugt und wühlt darin. Auf der Fensterbank sitzt \theJosef, kaut, während er spricht, an seiner kalten Pfeife. Die Tische leer, peinliche Ordnung im Zimmer.}

\begin{play}

\Josef
Was suchen Sie denn da schon wieder,Schwester?

\Heilsarmeeschwester
Die Kinderwäsche.

\Josef
Ich sag' Ihnen doch, die werden Sie nicht finden.

\Heilsarmeeschwester
Aber du lieber Himmel, irgendwo muss sie doch sein. Alle Leute erzählen, dass Ihre Frau vom ersten Monat an besonders fleißig daran gearbeitet hat. Und nun, da jeden Augenblick das Kindchen---

\Josef
Jeden Augenblick! Das heißt es schon die ganze Woche.

\Heilsarmeeschwester
\marginnote{Zeitabschnitten}
\direction{auf ihn zutretend} Sie dürfen nicht verzagen, lieber Mann. So was kommt vor, so was hat weiter nichts zu bedeuten, besonders zu so ungewöhnlichen Zeitläuften.

\Josef
Acht Tage lang Wehen. Wie soll die Frau denn das aushalten.

\Heilsarmeeschwester
Frauen halten oft vieles aus. Viel mehr als Ihr Männer euch vorstellen könnt.

\Josef
So, woher wissen Sie denn das?

\Heilsarmeeschwester
\direction{sich abwendend} Man erlebt so manches. \direction{öffnet einen Schrank}

\Josef
Suchen Sie nur nicht wieder weiter,Schwester. Wenn die Wäsche nicht da ist, wird Barbara sie eben fortgegeben haben.

\Heilsarmeeschwester
Aber lieber Mann, wie stellen Sie sich das vor? Das Kindchen muss doch was auf den Leib bekommen. Bis jetzt ist keine Windel im ganzen Haus.

\Josef
Luise ist in die Stadt gegangen. Ich hab ihr Geld gegeben. Sie wird das Nötigste besorgen.

\Heilsarmeeschwester
\direction{nimmt etwas Tischwäsche aus dem Schrank und untersucht sie} Wenn aber doch schon alles vorbereitet war. Ich kann nicht verstehen, weshalb die Sachen mit einem mal nicht gebraucht werden sollen.

\Josef
Da müssen Sie Barbara selber fragen.

\Heilsarmeeschwester
Und warum sprechen nicht Sie mit ihr?

\Josef
Ich kann mit ihr überhaupt nicht sprechen.

\Heilsarmeeschwester
Das ist es eben. Sie dürfen sie nicht so viel allein lassen.

\Josef
\direction{steht auf, stellt sich an das Fenster, mit dem Rücken gegen die Schwester} Da liegt sie mit ganz großen grauen Augen. Ich weiß ja gar nimmer, was sie denkt. Und wenn sie stöhnt, das ist gar nicht sowie ein Mensch. Wie im Wald ein Tier, das man nicht sehen kann. Ich trau mich nicht mehr, sie anzurühren.

\Heilsarmeeschwester
Das wird alles anders, wenn das Kind erst einmal geboren ist.

\Josef
\direction{dreht sich um} Auf dem Tisch dort sind die Hemdchen gelegen. Agnes hat auch daran mitgenäht. Schwester, warum kommt denn das Kind nicht zur Welt? Warum kommt es denn nicht?
 
\Heilsarmeeschwester
Die Hebamme hofft zuversichtlich auf heute Nacht. Sie wird bis dahin wieder zurück sein. Sie musste nur noch zu der Geburt nach Dybern.

\Josef
\direction{hin und hergehend} Es sollte einer singen in diesem Haus. Die Sonne scheint ja so hell. \direction{singt} Eia popeia, was raschelt im Stroh \direction{sich unterbrechend} Verflucht noch einmal.

\Heilsarmeeschwester
Nein, nein, das dürfen Sie nicht. Sie werden sich doch jetzt nicht an Gott versündigen. Nun, da alles Missgeschick behoben erscheint, da der böse Nebel endlich gewichen ist.

\Josef
Reden Sie nicht vom Nebel, Schwester. Ich mag die frommen Lügen nicht mehr.

\Heilsarmeeschwester
Aber lieber Mann, dass plötzlich auch Sie---

\Josef
Ja, auch ich. Ich hab mich mein Lebtag auf den Herrgott verlassen, hab gebetet, war ein guter Christ. Wenn aber so was möglich ist, wenn das wirklich die Menschen waren und nicht das Wetter und das Schicksal und eine Krankheit, wenn sowas möglich ist und heute wieder die Sonne scheint, als wäre nicht geschehen, dann---

\Heilsarmeeschwester
Ach sprechen Sie doch bitte nicht weiter.

\Josef
Dann glaub ich an keinen Herrgott mehr.

\Heilsarmeeschwester
\direction{mit dem Blick nach oben} Der Himmel steh euch bei. Was ist nur plötzlich in euch gefahren.

\direction{\theJan kommt aus der Küche. In Hemdärmeln, blass und elend}

\Heilsarmeeschwester
Nun, da kommt ja unser Patient. Wie geht es heute, Schon besser? Soll ich Ihnen was zu essen bringen?

\Jan
\direction{wirft sich auf die Ofenbank} Lassen Sie mich endlich einmal in Ruh.

\Heilsarmeeschwester
Sie dürfen nicht so unvernünftig sein. Einen ausgehungerten Magen muss man langsam an Nahrung gewöhnen. Ein bisschen warme Milch

\Jan
Ich kotz' darauf.

\Heilsarmeeschwester
Nein, nein, da irren Sie ganz bestimmt. Wir wollen es wenigstens einmal versuchen. \direction{geht in die Küche}

\Jan
Was ist mit Barbara, Josef?

\Josef
Immer dasselbe.

\Jan
Und Luise?

\Josef
Ging in die Stadt. Es gibt nicht eine Windel im Haus.

\Jan
Kannst du das verstehen?

\Josef
Nein.

\Jan
Und wie lange soll das noch dauern?

\Josef
Das weiß keiner. Heut' war der Doktor hier. Er wollte auch noch nach dir sehen.

\Jan
\direction{sehr heftig} Das Schwein kommt mir nicht mehr an den Leit. Der hat mitgeholfen. Zwangsernährung heißt man das. Drei von den Kerlen hielten mich. Und Thomsen führte die Röhre ein. Ich kann kein Essen mehr sehen.

\Josef
Weshalb bist du denn in den Hungerstreik getreten?

\Jan
Weil gar nichts anderes mehr zu machen war, Aber sie haben mich klein gekriegt. Da muss einer gesund sein und feste Lungen haben, sonst kommt er nicht durch. Sie haben mich klein gekriegt, Josef. Ich möchte mich am liebsten ins Bett legen und gar nicht mehr aufstehen.

\Josef
Lass gut sein, Jan, es wird schon anders werden. Musst nur Ruhe haben und wieder zu essen versuchen.

\Jan
Ach was, halt das Maul.

\Heilsarmeeschwester
\direction{steckt den Kopf zur Tür herein} Darf ich auch ein bisschen Kaffee in die Milch hineintun?

\direction{\theJan antwortet nicht. Drückendes Schweigen}

\Josef
\direction{steht auf, geht auf den Radioapparat zu} Wollen mal hören, was es Neues gibt.

\character{Radio}
Und hat sich die Stimmung der Bevölkerung im allgemeinen völlig beruhigt. Mit dem plötzlich eingetretenen frostklaren Sonnenhimmel schwand der beängstigende Nebel, der besonders zarten und gefährdeten Lungen verderblich war. In der Gegend von Dybern atmet man erleichtert auf, alles geht wieder fröhlich an die gewohnte Arbeit, die bedauerlichen Vorfälle, deren Schauplatz das neue Kino geworden war, sollen so rasch als möglich der Vergessenheit anheim fallen und ---

\Jan
\direction{hat einen Aschenbecher, der neben ihm auf den Tisch stand, genommen und schleudert ihn gegen den Radioapparat}

\Josef
\direction{springt entsetzt auf den Apparat zu, der immer weiter spricht} Bist du verrückt!\direction{Die \theHeilsarmeeschwester steckt den Kopf zur Tür herein}

\character{Radio}
\direction{fortwährend} Gedenkt man doch keine weiteren Konsequenzen

\Josef
\direction{drückt auf den Knopf}

\Jan
Der Teufel soll die Lügenbande holen.

\direction{Die \theHeilsarmeeschwester ist wieder verschwunden}

\Josef
Sag mal, weißt du es denn gar so bestimmt?

\Jan
Ich weiß es von Doktor Jonas. Aber der kommt nicht durch mit dem, was er sagt. Der ist ein anständiger Mensch. Und weißt du, wer die Pest ausgestreut hat?

\Josef
Man sagt, der Oberst.

\Jan
Der ist nun tot. Erstickt am eigenen Gift. Jonas fand ihn mit seinem Diener im Nebel. Aber was nützt uns das. Es wird ein anderer kommen, der das gleiche tut.

\Josef
Warum glaubst du das?

\Jan
Wo Giftgas erzeugt wird, da muss es auch einmal ausbrechen. Das sagt Doktor Jonas. Aber er kommt nicht durch damit.

\Josef
Eigentlich kein Wunder, dass Barbara ihr Kind nicht bekommen kann. In so einer Welt.

\Heilsarmeeschwester
\direction{bringt ein Tablett mit Kaffee und Butterbroten. Stellt es vor \theJan hin} Da mein Lieber, greifen Sie zu.

\Jan
Bringen Sie das alles wo anders hin, Schwester Leutnant.

\Heilsarmeeschwester
Wohin denn?

\Jan
\direction{schiebt das Tablett zur Seite} Denen, die essen wollen, nicht denen, die hungern wollen.

\Heilsarmeeschwester
Aber warum, um des Himmelswillen, wollen Sie denn um jeden Preishungern?

\Jan
Weil mir nichts mehr schmeckt.

\Heilsarmeeschwester
\direction{zu \theJosef} Verstehen Sie das?

\Josef
Die Schwester hat recht, Jan. Du darfst dich nicht so gehen lassen. Kommst ja ganz von Kräften. Und Luise weint sich die Augen aus. Du hast das Mädel doch lieb.

\Jan
\direction{hebt den Kopf} Halt, was ist das. War das nicht Barbara? ---

\Josef
Ich hab nichts gehört.

\Jan
Es war wie ein Stöhnen.

\Josef
Da hätt ich doch auch was hören müssen.

\Heilsarmeeschwester
Ich will für alle Fälle einmal nachsehen gehen. \direction{ab}

\Jan
Dass du das Frauenzimmer in deinem Haus haben kannst. Diese schmalzige Stimme. Und was braucht sie denn eine Uniform.

\Josef
Mein Gott, sie hilft eben mit und sie will es so. Die Hebamme kann doch nicht ewig hier sitzen, es kommen auch andere Kinder zur Welt. Und dann sie meint es ja gut, die Schwester.

\Jan
Es ist sehr einfach, es gut zu meinen. Darauf kommt's nicht an.

\Josef
Auf was denn sonst?

\Jan
Man muss das Richtige tun.

\Josef
Weisst du denn, was das Richtige ist?

\Jan
Nein, nicht ganz. Aber was das Unrichtige ist, das weiß ich.

\Heilsarmeeschwester
\direction{kommt aufgeregt} Lieber Mann, kommen Sie rasch, Ihre Frau hat sich eingesperrt. Ihre Frau antwortet nicht auf mein Klopfen.

\Josef
\direction{springt auf} Was soll das sein!

\Heilsarmeeschwester
Kommen Sie, kommen Sie. \direction{ab mit \theJosef}

\direction{Kurzes Klopfen. In die Tür tritt die alte \theKathrine}

\Jan
\direction{hebt den Kopf} Was willst denn du hier?

\Kathrine
\direction{in der Tür stehen bleibend} Ist das Kind schon da?

\Jan
Mach die Tür zu. Mich friert.

\Kathrine
Die Sonne ist kalt. Mich blendet sie nimmer. Ist das Kind schon da?

\Jan
Komm herein oder geh.

\Kathrine
\direction{schnuppernd} In diesem Haus ist keine gute Luft.

\Jan
Was redest du?201

\Kathrine
\direction{mit ausgestrecktem Stock} Es riecht nach Senf. Da geh ich lieber \direction{schließt dieTür hinter sich}

\Jan
\direction{schnuppernd} Das Weib ist verrückt.

\Josef
\direction{kommt zurück mit der \theHeilsarmeeschwester} Jan, Jan, was sollen wir tun? Barbara hat sich eingeschlossen.

\Heilsarmeeschwester
Sie gibt überhaupt keine Antwort mehr.

\Jan
\direction{aufhorchend} Pst. Schweigt still.

\Heilsarmeeschwester
Was ist denn?

\Jan
Es war wie ein Stöhnen.

\Josef
Jetzt hab ich es auch gehört.

\Heilsarmeeschwester
Ich glaube, ihr irrt.

\Jan
Es war sehr leise.:

\Josef
\direction{packt Jan am Ärmel} Das Kind, Jan, das Kind!

\Heilsarmeeschwester
Wir werden die Tür aufsprengen müssen.

\Josef
Das Kind, Jan, ich hab solche Angst.

\Heilsarmeeschwester
Wo haben Sie denn Ihr Werkzeug, Mann?

\Jan
So schweigen Sie doch. Merken Sie nicht, dass diese Frau jetzt allein sein will.

\Josef
Ach Jan, warum hat sie sich eingeschlossen. \direction{sinkt an einem Tisch zusammen}

\Heilsarmeeschwester
\direction{rüttelt Josef} Sie werden sie in ihrer schweren Stunde doch nichtallein lassen wollen. Sie sind ihr Mann. Sie müssen Mut beweisen und Kraft.

\Josef
\direction{mechanisch} Barbara hat sich eingeschlossen.

\Heilsarmeeschwester
Es ist unsere Pflicht, dass wir---

\direction{Kurzes, hartes Klopfen an der Tür. Herein kommen der \theSergeant und sechs \theSoldaten Feldgraue Uniformen}

\Sergeant
Ist hier das Wirtshaus am Rand?

\direction{Betretene Pause}

\Heilsarmeeschwester
Ja.

\Sergeant
Wo ist die Frau?

\Heilsarmeeschwester
Die Frau liegt in den Wehen.

\Sergeant
Unmöglich. Das hieß es ja schon vor einer Woche.

\Heilsarmeeschwester
Die arme Frau hat es sehr schwer.

\Sergeant
Wir wollen sie selber sehen.

\Heilsarmeeschwester
Das --- das geht jetzt nicht.

\Sergeant
Warum?

\Heilsarmeeschwester
Weil --- sie sie hat ihre Tür versperrt.

\Sergeant
Binden Sie uns keinen Bären auf. Wir haben Befehl, nicht ohne die Frau zurückzukehren. Sie kann das Kind gleich mitnehmen.

\Josef
Wie? Wo soll sie denn hin mit dem Kind?

\Sergeant
Sind Sie der Mann? Nun, dann machen Sie keine Flausen. Im Ausnahmezustand gibt's keinen Pardon. Aufruhr und Brandstiftung sind schwere Verbrechen.

\Jan
Und das Kind steckt ihr gleich mit ins Gefängnis?

\Heilsarmeeschwester
Um Himmelswillen, es ist ja noch nicht einmal auf der Welt. Und die Tür ist versperrt. So glaubt uns doch.

\Sergeant
\direction{achselzuckend} Wir werden warten. \direction{setzt sich an die Seite an einen Tisch.Hinter ihm stehen die sechs \theSoldaten}

\Jan
\direction{steht auf, feierlich} Pst. Habt ihr gehört?

\Sergeant
\direction{hebt den Kopf} Dort oben geht jemand hin und her\ldots

\Heilsarmeeschwester
Ich muss ein Werkzeug haben. Wer hilft die Tür aufsprengen.

\Sergeant
\direction{hebt den Kopf} Dort oben singt jemand.

\Josef
\direction{hebt den Kopf} Barbara.

\Heilsarmeeschwester
Ich geh zu ihr. Sie muss mich hineinlassen.

\direction{zu \theJosef} Kommen Sie mit.

\direction{Kurze Pause}

\Barbara
\direction{Stimme, leise und rauh} Das sind die lieben Gänslein, die haben kein ---

\direction{Die Küchentür wird aufgestoßen. \theBarbara wankt herein. Während der folgenden Szene lehnt sie links hinten am Ofen, rechts an der Seite hinten \theJosef, \theJan und die \theHeilsarmeeschwester, etwas weiter vorne, abseits von ihnen die \theSoldaten}

\Barbara
Guten Tag. Willkommen die Herren Soldaten. Man holt mich wohl ab.

\Josef
Barbara, warum---

\Barbara
Ach Josef, es ist nicht leicht, eine Mutter zu sein.

\Josef
Warum hast du dich eingeschlossen?

\Barbara
Und ich hab mich schon so gefreut auf das Kind.

\Josef
\direction{steht mühsam auf} Barbara, du hättest nicht aufstehen sollen.

\Barbara
Bleib ruhig. Komm nicht her. Es darf mir keiner mehr in die Nähe.

\Heilsarmeeschwester
\direction{händeringend} Liebe Frau

\Barbara
\direction{Stützt sich auf den Vorsprung des Ofens} Schwester, wir brauchen keine Hemdchen mehr. Und auch keine Windeln. Mein schönes Kind, mein liebes Kind. Auf einer Wiese hätte es spielen sollen. In der Sonne. Aber die Sonne da \direction{zeigt gegen das Fenster} ist nicht echt.

\direction{Nach kurzer Pause zu den \theSoldaten}

Was wollt ihr von mir? Was gafft ihr euch an? Was wundert ihr euch? Ihr habt wohl eure Gasmasken vergessen. Mein Kind soll keine Gasmaske tragen. Mein Kind soll nicht im Nebel ersticken. Deshalb hab ich ihm gleich---

\direction{macht mit der hohlen rechten Hand eine langsame krampfartige Bewegung, als wollte sie mit dem Daumen etwas eindrücken}

\end{play}

\end{document}
