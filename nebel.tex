% KOMA-Script layout settings
\documentclass[
	final,
	a4paper,
	ngerman,
	mpinclude = true, % include marginpar in textwidth for headsepline
	twoside = true,
	open = right,
	cleardoublepage = plain,
	DIV = 13,
	BCOR = 1cm,
	titlepage = firstiscover,
	]{scrbook}

\usepackage[T1]{fontenc}
\usepackage[utf8]{inputenc}
\usepackage[utf8]{luainputenc}

\usepackage{xspace}
\usepackage{calc} % \widthof

% margin notes
\setlength{\marginparwidth}{1.8\marginparwidth}
\setlength{\marginparsep}{10mm} % line numbers
\newcommand{\marginnote}[1]{\marginpar{\singlespacing\raggedright\footnotesize#1}}

% custom formatting for acts and scenes
\addtokomafont{sectioning}{\rmfamily\scshape\mdseries\centering}
\newcommand{\act}{\chapter}
\renewcommand*{\raggedsection}{\centering}
\renewcommand*{\chapterformat}{}
\renewcommand*{\chaptermarkformat}{}
\newcommand{\scene}{\setcounter{subscene}{1}\section}
\RedeclareSectionCommand[style=chapter]{section}
\renewcommand*{\sectionformat}{Szene \thesection~— }
\renewcommand*{\sectionmarkformat}{Szene \thesection~— }
\newcommand{\direction}[1]{(\textit{#1})}
\newcommand{\setting}[1]{\vspace{-0.5\baselineskip}\centering\textit{#1}}
\newcommand{\hiat}{%
	\begin{center}
		\tiny
		\raisebox{0.5ex}{\rule{0.3\linewidth}{0.4pt}}
		\textit{fickstrich}
		\raisebox{0.5ex}{\rule{0.3\linewidth}{0.4pt}}
	\end{center}
}
% TODO refstepcounter
\newcounter{subscene}
\setcounter{subscene}{1}
\newcommand{\subscene}{\marginnote{Szene \arabic{chapter}.\arabic{section}.\alph{subscene}}\stepcounter{subscene}}

% "Elements of Typographic Style" table of contents
\DeclareTOCStyleEntry[
		raggedpagenumber=true,
		linefill = {},
		entrynumberformat = {\phantom},
		indent = 0cm,
	]{tocline}{chapter}
\newlength{\scenenumwidth}
\setlength{\scenenumwidth}{\widthof{Szene 8.88 }}
\DeclareTOCStyleEntry[
		raggedpagenumber = true,
		linefill = {},
		indent = 0.3\linewidth,
		entrynumberformat = {{\footnotesize{\textsc{Szene}}}\enspace},
		numwidth = \scenenumwidth,
	]{tocline}{section}

% header
\usepackage[
		automark,
		headsepline,
		headwidth=textwithmarginpar,
	]{scrlayer-scrpage}
	\pagestyle{scrheadings}
	\ihead{}
	\chead{}
	\ohead{}
	\cehead{\leftmark}
	\cohead{\rightmark}

\usepackage[utf8]{inputenc}
\usepackage{babel}

% typography
\usepackage{ebgaramond}
\usepackage{microtype}
\usepackage{setspace} % one-half spacing
\usepackage[modulo,running]{lineno} % line numbers
	%\renewcommand{\thelinenumber}{\thesection.\arabic{linenumber}}
	\renewcommand{\thelinenumber}{\arabic{section}.\arabic{linenumber}}
\usepackage{csquotes}
\usepackage{siunitx}

% special version for directors
\usepackage{substr}
\newcommand{\ifdirectorsversion}[2]{%
	\IfSubStringInString{\detokenize{regie}}{\jobname}{#1}{#2}
}

% two column layout for character names and lines
\usepackage{enumitem}
\newlist{play}{description}{1}
\newlength{\widthofchar}
\setlength{\widthofchar}{\widthof{\textsc{Schwester\quad}}}
\setlist[play]{
	labelwidth=\widthofchar,
	leftmargin=!,
	font=\rmfamily\mdseries\scshape,
	itemsep=0pt,
	before={\linenumbers*}
}

% gray out deletions
\usepackage{xcolor}
\usepackage{comment}
\ifdirectorsversion{%
	\newenvironment{deletion}{%
		\vspace{0.25\baselineskip}
		\hrule
		\vspace{0.25\baselineskip}
		\color{darkgray}
	}{
		\color{black}
		\vspace{0.25\baselineskip}
		\hrule
		\vspace{0.25\baselineskip}
	}
}{%
	\excludecomment{deletion}
}

% list of characters at the beginning of a scene
\newcommand{\characterlist}[1]{{\begin{center}\textit{Personen:}\quad{}#1\end{center}}}

% PDF options
\usepackage[final,hidelinks]{hyperref}
	\hypersetup{
		unicode     = true,
		linktoc     = all,
		pdftitle    = {Der Nebel von Dybern},
		pdfauthor   = {Maria Lazar},
		pdfsubject  = {Ein Drama},
		pdflang     = de-DE,
		pdfdisplaydoctitle = true,
	}
	\ifdirectorsversion{\hypersetup{pdftitle={Der Nebel von Dybern (Regie-Version)}}}{}
	\addto\extrasngerman{
		\renewcommand{\chapterautorefname}{Akt}
		\renewcommand{\sectionautorefname}{Szene}
	}
\usepackage{bookmark} % toc in PDF bookmarks

% shortcuts for characters
% within line
\newcommand{\thecharacter}[1]{\textup{\textsc{#1}}\xspace}
\newcommand{\theBarbara}{\thecharacter{Barbara}}
\newcommand{\theJosef}{\thecharacter{Josef}}
\newcommand{\theKathrine}{\thecharacter{Kathrine}}
\newcommand{\theGregor}{\thecharacter{Gregor}}
\newcommand{\theJan}{\thecharacter{Jan}}
\newcommand{\theAndreas}{\thecharacter{Andreas}}
\newcommand{\theLuise}{\thecharacter{Luise}}
\newcommand{\theAgnes}{\thecharacter{Agnes}}
\newcommand{\theGeneraldirektor}{\thecharacter{Generaldirektor}}
\newcommand{\theClarisse}{\thecharacter{Clarisse}}
\newcommand{\theAlexis}{\thecharacter{Alexis}}
\newcommand{\theThomsen}{\thecharacter{Thomsen}}
\newcommand{\theJonas}{\thecharacter{Jonas}}
\newcommand{\theSalwin}{\thecharacter{Salwin}}
\newcommand{\theBrix}{\thecharacter{Oberst Brix}}
\newcommand{\theMelchior}{\thecharacter{Melchior}}
\newcommand{\theHeilsarmeeschwester}{\thecharacter{Heilsarmeeschwester}}
\newcommand{\theErsterMann}{\thecharacter{Erster~Mann}}
\newcommand{\theZweiterMann}{\thecharacter{Zweiter~Mann}}
\newcommand{\theSergeant}{\thecharacter{Sergeant}}
\newcommand{\theDiener}{\thecharacter{Diener}}
\newcommand{\theKinder}{\thecharacter{Kinder}}
\newcommand{\theLeute}{\thecharacter{Leute}}
\newcommand{\theSoldaten}{\thecharacter{Soldaten}}

% speaker of line
\newcommand{\character}[1]{\item[#1]}
\newcommand{\Barbara}{\character{\theBarbara}}
\newcommand{\Josef}{\character{\theJosef}}
\newcommand{\Kathrine}{\character{\theKathrine}}
\newcommand{\Gregor}{\character{\theGregor}}
\newcommand{\Jan}{\character{\theJan}}
\newcommand{\Andreas}{\character{\theAndreas}}
\newcommand{\Luise}{\character{\theLuise}}
\newcommand{\Agnes}{\character{\theAgnes}}
\newcommand{\Generaldirektor}{\character{Direktor}}
\newcommand{\Clarisse}{\character{\theClarisse}}
\newcommand{\Alexis}{\character{\theAlexis}}
\newcommand{\Thomsen}{\character{\theThomsen}}
\newcommand{\Jonas}{\character{\theJonas}}
\newcommand{\Salwin}{\character{\theSalwin}}
\newcommand{\Brix}{\character{\theBrix}}
\newcommand{\Melchior}{\character{\theMelchior}}
\newcommand{\Heilsarmeeschwester}{\character{Schwester}}
\newcommand{\ErsterMann}{\character{1. Mann}}
\newcommand{\ZweiterMann}{\character{2. Mann}}
\newcommand{\Sergeant}{\character{\theSergeant}}
\newcommand{\Diener}{\character{\theDiener}}
\newcommand{\Kinder}{\character{\theKinder}}
\newcommand{\Leute}{\character{\theLeute}}
\newcommand{\Soldaten}{\character{\theSoldaten}}
\newcommand{\Stimme}{\character{\emph{Stimme}}}
\newcommand{\Junge}[1]{\character{Junge #1}}
\newcommand{\Maedchen}[1]{\character{Mädchen #1}}

% cover
\usepackage{pdfpages}

% title page
\addtokomafont{titlehead}{\scshape\lsstyle}
\titlehead{\centering Wery Important Production Berlin}
\title{Der Nebel von Dybern}
\subtitle{Ein Drama}
\author{Maria Lazar}
\date{\ifdirectorsversion{-- Regie-Version --}{}}
\publishers{S. Fischer Verlag}
\uppertitleback{%
	\centering
	\addsec*{Dramatis Person\ae}
	\vspace{\baselineskip}
	\raggedright
    \theBarbara, \quad schwanger\\
	\theJosef, \quad ihr Mann,%
		\quad  (\ref{scene:I}, \ref{scene:IV})\\
    \theKathrine, \quad blind\\
    \theGregor\\
    \theJan\\
    \theAndreas\\
    \theLuise\\
    \theAgnes\\
    Paul, der \theGeneraldirektor der Chemiefrabrik\\
    \theClarisse, \quad seine Frau\\
    \theAlexis, \quad Ingenieur\\
    Doktor \theThomsen, \quad Arzt\\
    Doktor \theJonas, \quad Arzt\\
    \theSalwin, \quad Pressekorrespondent\\
    \theBrix, \quad Militär, Chemiker\\
    Jakob \theMelchior, \quad Militär\\
    \theHeilsarmeeschwester\\
    \theErsterMann\\
    \theZweiterMann\\
    \theSergeant\\
    \theDiener\\
    \theKinder\\
	\theLeute und \theSoldaten.
}
\lowertitleback{%
	\footnotesize
	\centering
	Version vom \today.\\
	\vspace{0.5\baselineskip}
	Erschienen 1932 bei \textsc{S. Fischer, Verlag A.G. Berlin}.\\
	\vspace{0.5\baselineskip}
	Gesetzt mit \LaTeX{} und \KOMAScript{} in EBGaramond.\\
}

\begin{document}
\pagenumbering{alph}
%\includepdf{cover/cover}
\cleardoubleoddemptypage

\pagenumbering{roman}
\maketitle

\pdfbookmark[chapter]{\contentsname}{toc}
\tableofcontents
\cleardoubleoddpage

\pagestyle{headings}
\pagenumbering{arabic}
\doublespacing

\act{Erster Akt}
\scene{Der Nebel}
\label{scene:I}
\characterlist{\theBarbara, \theJosef, \theKathrine, \theGregor, \theJan, \theAndreas, \theThomsen, \theAgnes}
\setting{Eine einfache, saubere Wirtsstube. Frühe Nachmittagsdämmerung, schläfriges Licht. Auf der Fensterbank sitzt \theJosef mit einer Zeitung, an den Kachelofen gelennt, hockt die alte \theKathrine. \theJosef ist ein behäbiger, etwas dicker Mensch. \theKathrine ist blind. Sie starrt immer vor sich hin, als ob sie etwas sehen würde.}

\begin{play}

\Barbara
\direction{Eine tiefe volle Frauenstimme singt} \emph{\ldots Eia popeia, was raschelt im Stroh ---}

\Josef
\direction{hebt den Kopf gegen die Decke} Barbara

\Barbara
\direction{singend} \emph{Ja ---}

\Josef
Wenn du schon wach bist, dann komm doch herunter. Wir wollen Kaffee.

\Barbara
Ist jemand da?

\Josef
Mutter Katharine.

\Barbara
\direction{weiter singend} \emph{.. das sind die lieben Gänslein, die haben kein' Schuh, der Schuster hat's Leder, kein' Leisten dazu.} \direction{Stimme verklingt}

\Kathrine
Lass sie in Ruh. Sie kommt immernoch früh genug. Sie kommt viel zu früh,und wir brauchen keinen Kaffee.

\Josef
Ja, ja, schon gut. \direction{blättert in der Zeitung}

\Kathrine
Was steht denn da in der Zeitung drin? Du liest doch die Zeitung. Schöne Geschichten, Sonntagsgeschichten?

\Josef
Ja, ja, so was ähnliches.

\Kathrine
Ich brauch keine Zeitung, ich kann sie nicht lesen, ich hab keine Augen. Ich hab meine Ohren.

\Josef
Fang nur nicht wieder an mit den alten Geschichten.

\Kathrine
Es sind gar keine alten Geschichten. Das weißt du sehr gut, davon spricht einjeder. Erst heut' nach der Kirche ---

\Josef
Jetzt schweig schon still, ich will nichts weiter mehr hören. Und überhaupt, wenn Barbara herunterkommt. Frauen in ihrem Zustand ---

\Kathrine
In ihrem Zustand, in ihrem Zustand. Wer hat sie denn in den Zustand gebracht. Das treibt's und vögelt und denkt dabei an die Folgen nicht weiter.

\Josef
Halts Maul, man wird noch sein Kind kriegen dürfen.

\Kathrine
Sein Kind kriegen dürfen. Barmherziger Himmel! Hast' denn Milch für dein Kind? Und reines Wasser? Und saubere Luft?

\Josef
Ja, ja, ja und noch viel, viel mehr.

\Kathrine
Mir ist mein Mädelchen an der Brust verhungert. Und meinen Buben haben sie mir aus dem Feld gebracht, ich hab ihn nimmer erkannt. Gott sei Dank, dass ich jetzt nicht mehr sehen brauch. Ihr aber, ihr müsst Kinder kriegen.

\Josef
Verflucht nochmal! \direction{springt auf} Das ist doch --- das ist doch zehn, zwölf, fünfzehn Jahre her. Wir haben keinen Krieg mehr. Hast du verstanden!

\Kathrine
Das sagen alle, aber es nicht wahr. Es ist eine schlechte Luft in der Welt. Wenn es auch nicht in der Zeitung steht.

\Josef
Kein Wort weiter. Barbara kommt.

\Barbara
\marginnote{schwanger}
\direction{stößt die Tür auf. Sie ist eine große Frau, stark in der Hoffnung}

Tag, Mutter Kathrine. Schön, dass du wiedermal zu uns gefunden hast, \direction{räkelt sich} Ach Gott, ach Gott, bin ich faul. Wie kann man nur am Nachmittag so schlafen.

\Kathrine
Das ist gut, das ist recht, das ist so am besten. Schlaf du nur. Kannst garnicht genug schlafen.

\Barbara Wie? Was meinst du?

\Josef
\direction{steht auf, ungeduldig, macht ein Zeichen an der Stirn} Lass sein, Barbara. Was ist mit unserem Kaffee?

\Barbara
Gleich, gleich, das Wasser ist schon aufgestellt. Aber hier ist ein Dampf. Zum Ersticken. Hast wie der einmal nichtschlecht gepafft. \direction{geht zum Fenster und stößt es auf}

\Kathrine
\direction{schnuppernd} Macht das Fenster zu, macht das Fenster zu.

\Barbara
Ach lass doch. Das bisschen frische Luft.

\Kathrine
Das ist nicht Luft, das ist Nebel.

\Barbara
\direction{beugt sich hinaus} Was für ein komischer gelber Nebel. Man sieht ja nichteinmal die Linde mehr.

\Josef
\direction{schließt das Fenster} Genug gelüftet, es kommt kalt herein. \direction{zeigt auf einenTisch} Und nimm dort doch die Kinderwäsche weg. Es werden sicherlich bald Gäste kommen.

\Barbara
\direction{legt die Wäsche zusammen} Eins, zwei, drei, vier, fünf, sechs Hemdchen. Nocheinmal sechs, dann sind zwei Dutzend voll. Die feinen Säumchen näht Agnes. Das Mädel hat wirklich unglaubliche Augen.

\Josef
Wo ist Agnes denn heute?

\Barbara
\marginnote{Unweit der belgischen Stadt Ypern war im April 1915 von deutscher Seite erstmals Giftgas eingesetzt worden.}
Sie ist schon früh morgens nach Dybern gegangen. Zu Annemarie. Die liegt immer noch krank. Agnes bringt ihr Kirschenkompott.

\Josef
Nach Dybern. Sag mal, du hast sie doch nicht allein gehen lassen?

\Barbara
Warum denn nicht?

\Josef
Es ist nur---ich meine---es ist scheussliches Wetter --- nass und kalt. Und dann plötzlich stock finster am hellichten Tag. \direction{dreht das Licht an}

\Barbara
\direction{ist inzwischen in die Küche gegangen, wo man sie, die Tür bleibt offen, herumhantieren sieht} Sie geht den Weg ja nicht zum ersten Mal.

\Josef
Wird sie denn nicht auf der Straße kommen?

\Barbara
Das glaub' ich kaum. Der Weidenweg ist doch viel näher. \direction{summt} \emph{Die Gänslein gehen barfuß und haben kein Schuh.}

\Josef
\direction{ist inzwischen auf und ab gegangen} Wann kommt sie denn zurück?

\Barbara
\direction{von der Küche her} Wie?

\Josef
Wann soll Agnes zurück sein?

\Barbara
\direction{kommt mit einem Tablett herein} Eh' es finster wird.

\direction{stellt den Kaffee vor \theKathrine} Da, Kathrine, greif zu.
\marginnote{er ist stark}
Im Kaffee ist viel Haut, und da hast du auch ein paar feine Kuchen.

\direction{sieht plötzlich erstaunt zum Fenster hin}
Ach, du meine Güte, es ist ja schon finster.

\Josef
Du hättest das Mädel doch nicht so allein hinauslassen sollen.

\Barbara
Sag mal, Josef, was hast du denn heute?

\Josef
Ach, gar nichts. Gib mir die Tasse her.

\Barbara
Da ist was los, du verschweigst mir Was.

\Josef
Ich verschweig' dir nichts. Ist ja alles nur dummes Gewäsch. Gut, dass du heute nicht auf dem Kirchplatz warst\ldots

\Barbara
Du, Josef, Jetzt will ich aber wirklich schon wissen ---

\Josef
Da gibts nichts zu wissen. Lass mich in Ruh'. Man wird ja selber schon ganz verblödet, so einen gottsverfluchten, nebligen Sonntag lang. Von Gerüchten darf man sich nicht in's Boxhorn jagen lassen, und überhaupt, eine Frau wie du, in deinem Zustand ---

\Barbara
Josef, jetzt mach mir nicht länger was vor. Ich merk' es doch die ganzen letzten Tage. Da wird immerfort nur geflüstert und getuschelt, und keiner schaut einem mehr grad ins Gesicht. Glaubst du, ich merk so was nicht? Ich spür' es schon auf der ganzen Haut.

\Kathrine
Es ist eine schlechte Luft in der Welt.

\Barbara
Wie? Was heißt das?

\Josef
Hör' nicht auf sie. Sie hat wieder ihren verrückten Tag. Man hat ein paar tote Rehe gefunden, zwischen den Weiden, und hinter Dybern, im Straßengraben auch noch eine verreckte Kuh.

\Barbara
Und?

\Josef
Und den Leuten wurde schlecht, als sie das Vieh so liegen sahen. Ein alter Bauer war es und sein Sohn. Müssen ja rechte Helden sein, die beiden.

\Barbara
Und?

\Josef
Und weiter nichts. Kannst das alles selbst in der Zeitung lesen. Dort steht auch von den unsinnigen Gerüchten.

\Barbara
Wenn du an die Gerüchte nichtglaubst, weshalb verschweigst du sie mir?

\Josef
Weißt du, Barbara, in deinem Zustand --- und seit du uns unlängst erst der Länge nach auf der Nase lagst wegen dem Försterhund

\Barbara
Wenn einem plötzlich mitten am Tag ein Hund in der Stube erschossen wird, und das Vieh liegt da und hat ganz blaue Augen da kann einem leicht bisschenschwindlig werden. ---

\Josef
Lass gut sein, Barbara, reg' dich nicht wieder auf.

\Barbara
Ach was, sprich nicht immer mit mir, als wär' ich jetzt nicht ganz bei Trost. Und was das Kleine ist, das liegt gut und sicher in meinem Bauch. Gib die Zeitung her. \direction{setzt sich mit der Zeitung an einen Tisch} \direction{\theJosef drückt auf den Knopf des Radios, leichte Schlagermusik}

\Barbara
\direction{hebt nach ein paar Sekunden plötzlich den Kopf und sagt sehr laut} Sag mal Josef, weiß man genau, dass der Rörsterhund wirklich die Tollwut hatte?

\Josef
\direction{zuckt die Achseln} --- Ich hab noch niemand gefragt.

\direction{die Tür wird aufgestoßen. Herein kommen \theGregor, \theJan, \theAndreas und \theLuise. \theGregor ist ein älterer Mann. \theJan ein spindeldümrer, zappliger Kerl, \theAndreas ein kräftiger schöner Bursche, \theLuise ein schlankes Mädchen, Typus der intelligenten Arbeiterin}

\Gregor
Oh, hier ist fein warm.

\Jan
Und immer Musik. Da gibts keine steifen Beine nicht. Komm, Luise. \direction{legt den Armum ihre Hüften}

\Luise
Lass los! Grüß Gott, Barbara.

\Gregor
\direction{zu \theAndreas, der als letzter kommt} Mach die Tür zu, Andreas.

\Luise
Mein Gott, der Nebel, das ist ja schon wie Rauch.

\Jan
Aber hier ist es gleich. Hier sind wir fidel. \direction{johlt zur Musik} \emph{Denn das Wirtshaus am Rand, das ist ja bekannt im ganzen Land.}

\Gregor
\direction{zu \theJosef} Einen Scharfen, der einheizt!

\Josef
\direction{in dem er ihm einschenkt} Der Jan kriegt nichts. Der ist ja jetzt schon besoffen. \direction{stellt das Radio ab}

\Jan
Oho.

\Josef
Woher kommt ihr denn?

\Gregor
Von dem neuen Kino. Dort gibt es jetzteinen großen Ausschank.

\Josef
Spielt es denn schon?

\Gregor
Nein, noch nicht, aber man kann es sich ansehen. Was rennst du denn so herum, Andreas? Was suchst du denn?

\Jan
Na der, der sucht doch natürlich die Agnes.

\Andreas
So schweig schon einmal.

\Jan
Frau Wirtin, wo ist denn das Fräulein Schwester? Das entzückende, das reizende Fräuleinchen Schwesterchen?

\Barbara
\direction{sieht verwirrt von der Zeitung auf} Agnes?

\Josef
Sie wird gleich kommen, sie war in Dybern, ein Krankenbesuch. Sie muss jeden Augenblick da sein. Aber erzählt doch lieber, wie ist das Kino?

\Jan
Fein, pickfein, viel zu fein für uns arme Leute. Wenn du mal erst die Marmortreppe runtersteigst ---

\Josef
Ist es wahr, dass es ganz unter der Erde ist?

\Jan
Ganz unter der Erde. Das ist jetzt das Neueste, das Modernste. Das ist das Schönste und das Gesündeste und das Billigste. Einen eigenen Architekten haben sie sich dazu herbestellt. Dicht an der Fabrik ist es auch. Man braucht sich nachder Arbeit bloß die Hände waschen ---

\Gregor
\direction{schlägt auf den Tisch} Und jetzt sagmir nur, was dir schon wieder daran nicht recht ist?

\Josef
Der Jan ist bös', wenn er nicht Grund genug zum Stänkern hat.

\Jan
Und ihr seid alle miteinand' Idioten. Wenn euch die hohe Direktion mal ein Zuckerstück hinhält, dann schnappt ihr danach. Sonst kann sie getrost auf eure Köpfespucken.

\Gregor
Es spuckt keiner auf unsere Köpfe. Und wenn sie uns ein Kino hinstellen, so verdienen sie schließlich selber daran.

\Luise
\direction{nachdenklich} Das muss ungeheuer viel gekostet haben, so ein Riesenkino ganz unter der Erde.

\Jan
Jetzt sag mir nur einer, warum ist esdenn ganz unter der Erde?

\Gregor
Damit du das Gras wachsen hörst, du Rotzbub. Das tust du ja ohnehin so gern.

\Andreas
Ich versteh aber auch nicht, warum sie es so hinunter bauen.

\Gregor
Jetzt fängst du auch an.

\Jan
Wir sind doch keine Maulwürfe.

\Gregor
Brauchst ja nicht runter, wenn es dirnicht passt.

\Kathrine
Mein Bub war auch in so einem Kino unter der Erde. Es war sehr nass.

\Josef
Schon gut, schon gut, Kathrine, sprich da nicht mit.

\Kathrine
Ihr werdet alle noch hinunter müssen. Damals hat auch ein jeder geglaubt, es trifft nur den andern.

\Josef
Aber es ist doch ein Kino, Kathrine, ein Kino.

\Jan
Schrei nicht so, sie ist ja nicht taub, nur blind. \direction{greift nach \theGregor{}s Glas}

\Gregor
\direction{hält ihn zurück} Lass sein, bist jaohnehin schon besoffen.

\Jan
\direction{reißt ihm das Glas aus der Hand und trinkt es aus} --- Natürlich bin ich besoffen, ich bin ja immer besoffen, und wenn mir mal die Luft ausgeht in unserer alten Stinkbude drüben, dann bin ich auch nur besoffen, fragt doch den Doktor Thomsen, der hat das gesagt, und wenn ein Mädel umfällt, mitten in der Arbeit, dann bin ich auch nur besoffen, und wenn einer die gewissen Flecken kriegt, die blauen Flecken, erst an den Händen, dann in den Augen ---

\Barbara
\direction{die bisher mit der Zeitung vor sichwie teilnahmslos gesessen ist} Was für Flecken, was für blaue Flecken?

\Luise
Halts Maul, Jan, du redst dich noch umdeinen Kopf.

\Gregor
Pack dein Bündel und geh', hat dich ja keiner nicht hergebeten, bist doch so nur ein Fremder. Geh du dort hin, wo es dem Arbeiter wirklich schlecht geht, wo er kein Fressen hat, kein Dach überm Kopf. Dort kannst du deine Reden halten, Gefahren gibts in jeder Fabrik.

\Josef
Und in unserer Gegend sind die besten Löhne. Das weiß ein jeder. Es liegt ein Segen über dem ganzen Land. Blumen hinter allen Fensterscheiben. Wenn ich denke, wie es früher gewesen ist. Nichts auf's Brot haben die Leute gehabt.

\Barbara
\direction{steht auf} --- Sagt mal, was sind denn das für blaue Flecken?

\Gregor
Nur aus Unvorsichtigkeit. Ihr könnt euch drauf verlassen, immer nur aus Unvorsichtigkeit. Was predige ich nicht täglich den Leuten. Eine Stickstoffabrik ist schließlich kein Kinderzimmer.

\Luise
Nein, Gregor, das ist eine Gemeinheit, was du sagst. Und außerdem bestimmt nicht wahr. Es tut nicht gut, wenn einer von uns so spricht.

\Gregor
Dir ist natürlich lieber, wenn einer hetzt, wie dein Jan. Siehst' dich wohl auch schon Versammlungen halten. Und was kommt dabei raus --- nichts als Not und Elend. Schau' du lieber, dass du einen braven Mann kriegst und ein paar Kinder und ein Häuschen in unserer Siedlung.

\Andreas
\direction{sieht auf die Uhr} Wann soll die Agnes von Dybern zurück sein?

\Josef
\direction{sieht zum Fenster hinaus} Es ist stock finster. Vielleicht sollte man ihr entgegengehen.

\Luise
Aber die Straße ist doch gut beleuchtet.

\Barbara
Sie kommt auf dem Weidenweg.

\Andreas
Auf dem Weidenweg!

\Jan
Herr des Himmels, was schickt ihr sie denn auf den Weidenweg!

\Gregor
Auf dem Weidenwog ist der Nebel am schlimmsten. Dort steigt er auf vom Fluss.

\Barbara
\direction{sehr heftig} Ja seid ihr denn alle verrückt geworden. Das Mädel hat doch eine Taschenlampe bei sich. Die geht den Weg zum hundertsten Mal.

\Luise
Aber der Nebel.

\Barbara
Am Nebel ist noch keiner gestorben.

\Andreas
\direction{steht auf} Gebt mir eine Lampe. Ich geh' ihr entgegen.

\Josef
\direction{geht eine Laterne holen}

\Andreas
Macht rasch, rasch --- Agnes --- mein Gott Agnes es wird ihr doch nichts geschehen sein. \direction{reißt \theJosef die Laterne aus der Hand und stürzt hinaus.}

\Barbara
\direction{ steht mitten im Zimmer und zählt während des Sprechens an den Fingern}
Ein paar Rehe, eine Kuh, einem Bauern ist schlecht geworden und seinem Sohn. In der Zeitung steht, das sind nur Gerüchte, in der Zeitung steht, es istnicht der Nebel, in der Zeitung steht, man kann ja auch im Wald krank werden und auf offenem Feld, und Tiere sterben eben --- aber der Hund --- sag mal, Gregor, hat der Hund wirklich die Tollwut gehabt?

\Gregor
Was weiß denn ich.

\Barbara
Ja, warum wisst ihr das denn alle nicht?

\Luise
Es ist ein ungesunder Herbst, Barbara. Und wenn der Nebel einmal so dick wird wie eine nasse Mauer ---

\Josef
Sollst dir nicht so viel den Kopf zerbrechen, Barbara. Gesunden Lungen macht der Nebel nichts. Nur wenn einer ohnehin Asthma hat oder sonstwie schweren Atem

\Jan
Und wo doch die Hauptsache ist, dass die Bevölkerung ruhig bleibt, immer nur ruhig, wie es an unserer Kirche angeschlagen steht, und der Herr Pfarrer hat ja auch gepredigt, dass Gott es gar nicht so böse meint. Man braucht doch nicht in den Wald zu gehen und zum Fluss

\Josef
Halts Maul, Kerl, oder ich schmeiß' dich hinaus.

\Barbara
\direction{sinkt plötzlich auf einem Stuhl zusammen} Agnes --- o Gott, warum habt ihrmir das nicht früher gesagt.

\Luise
Aber, Barbara, es ist doch nichts geschehen.

\Josef
Das kommt nur von den verfluchten Gerüchten.

\Gregor
Im Nebel kann doch keine Krankheit sein.

\Jan
\direction{steht auf} Ich geh dem Andreas nach.

\Barbara
Was ist denn aus dem Bauern gewordenund seinem Sohn?

\direction{\theJosef und \theGregor zucken die Achseln}

\Barbara
Ja, warum wisst ihr das denn alle nicht?

\direction{man hört den Motor eines Autos und hupen. \theJosef geht hinaus und kommt gleich darauf mit Doktor \theThomsen zurück. \theThomsen ist ein älterer, robuster Mann}

\Thomsen
\direction{schüttelt sich vor Nässe} \ldots Guten Abend.

\Gregor
Guten Abend.

\Luise
Gott sei Dank, der Doktor.

\Thomsen
Wieso? Was ist denn?

\Josef
Es ist keine Ruhe im Land, Herr Doktor

\Thomsen
Ja, ja, schon gut. Gebt mir rasch cinen starken Grog, und dann hilft mir einer bei meiner Panne. \direction{setzt sich mürrisch in eine Ecke}

\Josef
\direction{aus der Küche heraus} Wollen Herr Doktor auch was essen?

\Thomsen
Nein.

\Barbara
Woher kommt denn der Doktor?

\Thomsen
Aus Dybern.

\Luise
Von einem Kranken?

\Thomsen
Nein.

\Gregor
Von sonst einem Besuch?

\Thomsen
Nein.

\Barbara
\direction{steht auf, stellt sich vor \theThomsen, voll Angst} woher denn sonst?

\Thomsen
Von einem Toten.

\direction{Stille}

\Thomsen
Von zwei Toten, wenn ihr es wissen wollt.

\Barbara
Der Bauer und sein Sohn. \direction{sinkt wieder in ihrem Stuhl zusammen}

\Josef
\direction{indem er den Grog bringt} \ldots Asthma, Herr Doktor, nicht wahr, ein schlimmer Atem.

\Luise
\direction{sehr klar} Ist es wahr, dass eine Seuche im Nebel steckt? Dass man krank wird im Wald?

\Thomsen
Es scheint so zu sein.

\Luise
Warum sperrt man dann den Wald nicht ab?

\Thomsen
Weil man nicht wissen kann, wie der Wind sich dreht.

\Gregor
Und woher kommt die Krankheit so plötzlich?

\Thomsen
Das weiß Gott allein.

\Kathrine
Du sollst den Namen Gottes nicht eitel nennen.

\Thomsen
\direction{fährt zusammen} Was ist das?

\Josef
Entschuldigen, Herr Doktor, es ist nur unsere alte Kathrine.

\direction{macht wieder das Zeichen an der Stirn. Zu \theBarbara, die plötzlich ein Tuch vom Haken reißt undzur Tür geht}

Was ist, wohin willst du?

\Barbara
Ich geh dem Mädel entgegen. \direction{stockt, denn man hört ein paar schrille Pfiffe}

\Luise
\direction{springt auf} Das ist Jan, das ist seine Pfeife.

\Andreas
\direction{Stimme}
\ldots Macht auf, macht rasch auf.

\direction{\theJosef reisst die Tür auf, \theJan und \theAndreas tragen \theAgnes herein. \theAgnes, ein fünfzehnjähriges zartes Mädchen, gelb im Gesicht, mit entsetzlichen Augen}

\Agnes
Wasser --- Wasser

\Thomsen
\direction{wirft seinen Mantel auf eine Bank} \ldots legt sie her, sofort.

\Agnes
Wasser \direction{mit den Händen an der Brust} Ich brenne.

\Luise
\direction{hält ihr ein Glas Wasser hin} Hier.

\Agnes
\direction{trinkt} Ich brenne, ich brenne.

\Josef
Herr Doktor, Herr Doktor!

\Barbara
\direction{schiebt \theAgnes ein Tuch unter denKopf} Agnes.

\Andreas
\direction{wirft sich neben ihr hin} Agnes.

\Agnes
Ich brenne.

\Thomsen
Nicht weiter trinken. \direction{er reißt ihrdas Glas aus der Hand}

\Agnes
Das Wasser brennt. \direction{sinkt einen Augenblick zurück}

\Kathrine
\direction{sie ist bis jetzt im Hintergrundan der anderen Seite des Zimmers gesessen. Nun geht sie, den Stock gehoben, in der Richtung von \theAgnes wie gezogen auf das Mädchen zu} \ldots Es riecht nach Senf.

\Agnes
Ich verbrenne.

\end{play}

% ----------


\scene{Kriesensitzung}
\label{scene:II}
\characterlist{\theGeneraldirektor, \theClarisse, \theAlexis, \theThomsen, \theJonas, \theDiener, \theSalwin}
\marginnote{Ledersessel}
\setting{Ein elegantes, diskretes Herrenzimmer, Bücherregale, Tisch mit Klubfauteuils. Schreibtisch, Telephon. Es ist Abend, An dem Schreibtisch, der mit Zeitungen übersät ist, sitzt der \theGeneraldirektor, ein angenehmer, etwas dicker Mensch, mit gutmütigem Gesicht. Er liest in einer Zeitung. Wie der Vorhang aufgeht, tritt eben seine Frau, \theClarisse, in die Tür hinter seinem Rücken. Sie ist hübsch und gepflegt.}

\begin{play}

\Clarisse
Paul!

\Generaldirektor
Ja, mein Kind?

\Clarisse
Sag mal, Paul, ist es wahr, kommst du wieder nicht zum Abendessen?

\Generaldirektor
\direction{mit einer Handbewegung gegen den Schreibtisch} Geschäfte. Dusichst doch, ich weiß nicht, wo mir derKopf steht.

\Clarisse
\direction{setzt sich auf die Armlehne von seinem Sessel} Brrr, die vielen Zeitungen. Kann das nicht dein Sekretär machen? Du musst das Zeug doch nicht selber lesen.

\Generaldirektor
Lass sein, Kind, das verstehst du nicht.

\Clarisse
Aber, dass du was essen musst, das verstehe ich. Ich werde dir ein paar Brötchen bringen lassen und Tee. Und ich werde dir dabei Gesellschaft leisten.

\Generaldirektor
Du bist lieb.

\Clarisse
Immer allein im Speisezimmer unten, das halte ich nicht aus. Zu Mittag sind noch wenigstens die Kinder bei mir.

\Generaldirektor
Es dauert jetzt wirklich nurmehr ein paar Tage.

\Clarisse
Und dann?

\Generaldirektor
Dann --- fahren wir vielleicht nach Paris.

\Clarisse
Warum nur vielleicht?

\Generaldirektor
Schau, Clarisse, du darfst mich nicht quälen, es gehen wichtige Dinge vor im Werk. Vielleicht fährst du allein voraus mit den Kindern.

\Clarisse
Mit den Kindern? Du bist wohl nicht recht gescheit. Was soll ich mit den Kindern in Paris? Die brauchen doch keine Abwechslung.

\Generaldirektor
Wir werden uns das alles nochüberlegen. Am besten ist, \direction{das Telephonläutet, er nimmt don Hörer} Entschuldige---

jawohl, Doktor Thomsen, jawohl, ich habe angerufen--- .

Hören Sie mal---ich habe da ich habe da eine kleine Konferenz bei mir zu Hause---

Sie wissen schon---

einige Herren unseres Betriebes---

ich möchte Sie sehr bitten, kommen Sie doch auch herüber---

ja, ja, sofort---

Wie? schonwieder? Und wieder bei Dybern---

ja, dasverstehe ich, ich begreife vollkommen, eben deshalb wäre es wichtig, dass Sie jetzt kämen---

natürlich können Sie den Herrn auch mitbringen --- Doktor Jonas, nicht wahr---

er ist doch verlässlich---

ja,ja Also Sie kommen sofort.

\direction{legt den Hörer auf}

\Clarisse
Was ist denn los, Paul? Was ist denndas für eine Konferenz hier bei uns zu Hause um neun Uhr abends? Und wozu brauchst du dazu Doktor Thomsen?

\Generaldirektor
Ich kann dir das jetzt wirklich nicht in der Eile erklären. Die Herren müssen jeden Augenblick kommen. Ich bitte dich, du darfst mir nicht böse sein, aber ich bin sehr nervös, überreizt, überarbeitet --- geh lieber schlafen.

\Clarisse
Da will ich dich nicht länger stören, gute Nacht.

\direction{geht beleidigt hinaus. In der Tür prallt sie zusammen mit Ingenieur \theAlexis, der sich flüchtig vor ihrverneigt. \theAlexis ist ein etwas geschniegelter junger Mensch}

\Generaldirektor
\direction{schüttelt ihm die Hand} Gut, dass Sie da sind, Alexis. Sie sind der Erste. Nehmen Sie Platz. Hier, eine Ziegarette. Haben Sie den Oberst erreicht?

\Alexis
Er war natürlich unerreichbar wie immer. Im Laboratorium, darf nicht gestört werden.

\Generaldirektor
Er muss aber kommen. Es ist ganz unmöglich.

\Alexis
Herr Melchior, oder wie der Kerl heisst, den er sich da hält, versprach, ihm alles auszurichten. Eine scheußliche Visage hat dieser Bursche.

\Generaldirektor
Halten Sie es für sicher, dass der Oberst kommt?

\Alexis
Wir wollen hoffen.

\Generaldirektor
Ich habe Brix seit--- seitden unheimlichen Ereignissen überhauptnicht mehr gesehen. Es ist nicht angenehm, mit einem Menschen zu arbeiten, mit dem man so gar keinen Kontakt halten kann. In einem Fall wie jetzt, wenn die fürchterlichsten Gerüchte plötzlich entstehen, Verdächtigungen, die wir allen icht auf uns sitzen lassen können.

\Alexis
Wieso? Wer?

\Generaldirektor
Hier noch niemand. Bei uns in der Gegend noch niemand. Aber das Ausland, die Zeitungen --- noch wagt kein Mensch auszusprechen, was zwischen den Zeilen steht, noch schreibt man überall nur von dem rätselhaften Nebel --- es ist übrigens schon wieder ein Todesfall. Ein achtjähriger Junge in Dybern.

\direction{in diesem Augenblick steht \theClarisse wieder in der Tür}

\Clarisse
Ich bitte vielmals um Entschuldigung,ich ich muss dich rasch noch etwas fragen.

\Generaldirektor
Bitte?

\Clarisse
Im Kinderzimmer sind alle Fenster geschlossen. Die Kleinen haben doch noch keine Nacht ohne frische Luft geschlafen. Die Nurse behauptet, du hättest Auftraggegeben?

\Generaldirektor
Stimmt. Es ist ein hässlicher Nebel draussen, der sich auf die Lungen legt. Wir wollen für ein paar Tage ein Ausnahme machen.

\Clarisse
Aber Nebel allein kann doch nicht schaden

\direction{in diesem Augenblick führt ein \theDiener Doktor \theThomsen und Doktor \theJonas herein. \theJonas noch jung, hager und eckig}

\Clarisse
\direction{zu \theThomsen} Sagen Sie selbst, Herr Doktor, ob Nebel allein Kindern etwas schaden kann.

\Thomsen
\direction{sehr förmlich} Darüber kann ich keine Auskunft geben, gnädige Frau.

\direction{zum \Generaldirektor} Erlauben Sie, Herr Generaldirektor, dass ich Sie mit meinem Kollegen bekannt mache: Doktor Jonas. \direction{Händeschütteln}

\Generaldirektor
Doktor Jonas --- Ingenieur Alexis \direction{auf \theThomsen zeigend} Die Herren kennen sich wohl schon.

\Clarisse
\direction{im Hintergrund, betroffen und ängstlich} Paul.

\Generaldirektor
\direction{legt den Arm um sie undführt sie hinaus} \ldots Du solltest wirklich

\direction{kommt gleich darauf wieder zurück, etwas feierlich} Ich bitte, die Herren Platz zu nehmen. Leider sind wir noch immer nicht ganz vollzählig. Oberst Brix fehlt. Er muss aber jeden Augenblick kommen. Ich hoffe wenigstens. Wünschen die Herren Zigarren?

Bitte, \direction{läutet dem \theDiener}

iberst Brix ist leider nicht telefonisch erreichbar. Es gehört zu seinen Schrullen.

\direction{zum Diener} Whisky! ---

Es gehört zu seinen Schrullen, dass er um nichts in der Welt ein Telephon haben will. \direction{der \theDiener kommt mit Whisky} Er ist ein Sonderling, aber ein großer Gelehrter.

\Alexis
Ein Genie.

\Generaldirektor
Er ist die Seele unseres Werkes. Seine Erfindungen sind unabsehbar. Nur von praktischen Dingen will er nichts hören. Diesmal jedoch werden wir ihm das nicht ersparen können. Die Situation ist unerträglich.

\direction{einen Augenblick Schweigen, plötzlich wendet er sich unvormittelt an \theThomsen} Der Junge ist also wirklich tot? Und unter denselben Symptomen?

\Thomsen
Unter denselben Symptomen wie die andern. Sie können meinen Kollegen fragen. \direction{Hand bewegung gegen \theJonas}

\Jonas
Es ist immer dasselbe Bild.

\Generaldirektor
Und zwar?

\Thomsen
Erstickung.

\Jonas
Vergiftung.

\Generaldirektor
Herr Doktor, was meinen Sie damit?

\Alexis
\direction{fast gleichzeitig auffahrend} Wie können Sie so was behaupten?

\Jonas
Ich behaupte gar nichts. Es sind Vergiftungssymptome.

\Generaldirektor
Ich muss Sie bitten, sich deutlicher auszudrücken. An welche Art von Gift denken Sie?

\Jonas
An keines, das ich kenne.

\Generaldirektor
Also?

\Thomsen
Der Nebel ist vergiftet.

\Alexis
Dann ist es ja doch der Nebel.

\Thomsen
Es steckt eine neue, eine gefährliche Krankheit im Nebel. Eine Krankheit, die wir Ärzte noch nicht kennen.

\Alexis
Also eine neue Seuche, eine Epidemie?

\Thomsen
Es sieht so aus.

\Generaldirektor
Meine Herren, wenn wir mit unserer eigentlichen Konferenz auch auf den Oberst warten müssen \direction{Blick auf die Uhr} so dürfen wir doch auch jetzt nicht aneinander vorbei reden. Wir müssen den Tatsachen ins Auge sehen, wir müssen die Dinge beim Namen nennen.

\direction{in diesem Augenblick bringt ihm der \theDiener eine Visitkarte, die er erregt auf den Tisch wirft}

Da haben wir es. Ich wusste es ja. Ein Herr von der Pressekorrespondenz. Und um diesen Zeitpunkt. Eben angekommen. Nein, ausgeschlossen, unmöglich. Ich bin jetzt nicht zu sprechen. Der Herr soll morgen wiederkommen. Das heißt, nein, wenn er will, kann er warten. Er kann auch heute noch Bescheid erhalten. Führen Sie ihn in das Billardzimmer.

\direction{\theDiener ab}

\Alexis
\direction{nimmt die Karte} Salwin. Den Menschen kenne ich. Mit dem ist nicht zu spaßen.

\Generaldirektor
Am besten ist, wir lehnen die Verantwortung ab. Wir verlangen eine Kommission, ehe es zu spät wird. Chemische Werke sind keine Spielwarenfabrik. Was immer in der Gegend geschieht, es fällt auf uns. Dringt der Verdacht einmal in die Bevölkerung, so ist unabsehbar, wasnoch geschehen kann.

\Alexis
Welcher Verdacht?

\Generaldirektor
Sie wissen sehr gut, was ich meine.

\Alexis
Nein, ich weiß es nicht. Bei Gott, ich weiß es nicht. Und wer denn sonst, als ich, sollte es wissen. Es ist ausgeschlossen, ich schwöre Ihnen, meine Herren, es ist ausgeschlossen, dass dass unsere Industrie die Schuld daran trägt.

\Jonas
Sie glauben also an eine Nebelkrankheit?

\Alexis
Herr Doktor, was soll diese spöttische Frage?

\Thomsen
Mein Kollege meint ja selbst, dass es der Nebel ist. Er --- wir verstehen nur nicht, wieso.

\Alexis
Es muss doch jedom Kind begreiflich sein, dass die --- die gefährlichen Stoffe aus unserer Fabrik nicht plötzlich in den Wald von Dybern kommen können, ohne dass ein Mensch in der Fabrik selbst oder in der Stadt erkrankt. Wir sitzen doch hierdern schließlich Wand an Wand mit den großen Laboratorien --- aber in Dybern stirbt das Vieh am Fluss.

\Jonas
Und auch die Menschen, Herr Ingenieur. Denken Sie an das Wirtshaus am Rand, an das junge Mädchen, das Sonntag dort verbrannte, bei lebendige Leib inwendig verbrannte.

\Thomsen
Der Fall hat ungeheures Aufsehen erregt, vielleicht noch mehr als alle anden. Das Mädchen starb in der Wirtsstube vor den Augen der Gäste. Seither traut kein Mensch sich mehr auf den Weidenweg.

\Alexis
Nehmen Sie an, wir fabrizieren Gift, entsetzliches, gefährliches Gift. Nehmen Sie an. Aber dann bedenken Sie, wie käme dieses Gift gerade auf den Weidenweg. Herr Generaldirektor, helfen Sie mir doch.

\Generaldirektor
Es ist unbegreiflich, es ist nicht auszudenken, aber jeder Verdacht setzt sich durch, und wenn er noch so unsinnig ist. Wir brauchen eine Kommission.

\Alexis
Wir brauchen keine Kommission. Horrgott im Himmel, sind wir denn Verbrecher? Wir arbeiten im Dienste der Wissenschaft, des Friedens und des Staates. Wir sind nicht verpflichtet, Spione zu uns hereinzulassen. Unsere Forschungen sind noch nicht beendet. Oberst Brix wird das nicht gestatten.

\Generaldirektor
Oberst Brix scheint schon wieder einmal unerreichbar. Und ohne ihn sind mir die Hände gebunden.

\Alexis
Dann verlassen Sie sich auf mich. Wirmüssen die schändlichen Gerüchte bloss im Keim ersticken. Ich verbürge mich, ich, der Leiter der Abteilung A.

\Generaldirektor
Und wenn der Nebel bis nachDybern selber dringt? Es braucht sich derWind bloss etwas weiter westlich zu drehen.

\Alexis
Wir sind für den Wind nicht verantwortlich. Dann wird Dybern eben evakuiert.

\Jonas
Und wenn der Wind sich noch weiter westlich oder gar südlich dreht?

\Alexis
Dann schliessen wir unsere Häuser abund versuchen unsere neuen Sauerstoffpumpen.

\Jonas
Sie scheinen ja schon auf alles gerüstet?

\Alexis
Sind wir auch. Wir fürchten uns vorkeinem Nebel.

\Generaldirektor
Alexis, wir sind jetzt ganz unter uns --- sind Sie so sicher, dass es der Nebel ist?

\Alexis
\direction{aufspringend} Meine Herren, wenn esnicht der Nebel wäre, ich bin verantwortlich. Ich bin der Leiter der Abteilung A --- meine Herren, hier vor Ihren Augen würde ich mir eine Kugel durch die Schläfen jagen.

\Thomsen
Unvorsichtigkeit oder dergleichen ausgeschlossen?

\Alexis
Ausgeschlossen.

\Generaldirektor
Ausgeschlossen!

\Jonas
Und Verbrechen?

\Alexis
Sie sind ja wahnsinnig. \direction{der \theDiener kommt herein}

\Diener
Der Herr kann nicht länger warten. Der Herr muss heute Nacht noch weiterfahren.

\Generaldirektor
\direction{steht auf} Dann führen Sieden Herrn herein. \direction{gleich darauf wird vom Diener hereingeführt \theSalwin. Klein, geschmeidig, unterwürfig und unverschämt}

\Salwin
\direction{sich umsehend} Herr Generaldirektor?

\Generaldirektor
\direction{auf ihn zutretend} Das binich. Ich begrüße Sie, Herr Salwin. Sie kommen eben zu einer kleinen Abendgesellschaft. Darf ich Sie mit meinen Gästen bekanntmachen? Herr Salwin von der Pressekorrespondenz. Doktor Thomsen, Ingenieur Alexis, Doktor Jonas. \direction{kurze Verbeugungen}

\Salwin
Ich bitte tausendmal um Entschuldigung, wenn ich so plötzlich hier eindringe. Es handelt sich nur um eine ganz kurze Information.

\Generaldirektor
Wünschen Sie mich allein zusprechen?

\Salwin
Ich weiß nicht aber die Fragen, die ich zu stellen habe, sind doch vielleichtmehr diskreter Natur. Es handelt sich --- es handelt sich um den Nebel von Dybern.

\Generaldirektor
Nein wirklich, davon war ja eben unter uns die Rede. Eine unbegreifliche Geschichte. Aber bitte, nehmen Sie doch Platz. Ich selber werde Ihnen nicht viel sagen können, ich komme den ganzen Tag aus meinem Büro nicht heraus. Aber Herr Doktor Thomsen ist Arzt, er kennt die einzelnen Fälle.

\Salwin
Sehr angenehm, Herr Doktor, sehr interessant. Sie begreifen ja, die ganze Öffentlichkeit harrt gespannt auf eine Aufklärung. \direction{zieht ein Notizbuch hervor} Peinliche Gerüchte kursieren im Ausland. Es wird ja heutzutage alles politisch ausgebeutet. Woran sterben die Leute in Dybern? Was ist Ihre Meinung?

\Thomsen
Ich habe keine bestimmte Meinung. Eine Kommission von Ärzten soll darüberentscheiden.

\Salwin
Aber dieser Kommission werden Sie beitreten?

\Thomsen
Selbstverständlich.

\Salwin
Und Ihre Meinung wird sein? Gift?

\Thomsen
\direction{zurückfahrend} Gift? Was für Gift?

\Alexis
Das ist doch eine gottverfluchte Gemeinheit.

\Salwin
Wie meinen Sie, Herr Ingenieur Alexis? Leiter der Abteilung A, nicht wahr?

\Alexis
\marginnote{\emph{Feigling}.\par(Das äußere Geschlechts\-teil einer Hündin.)}
Ich meine, dass jeder ein Hundsfott ist und ein Vaterlands verräter, der solche Gerüchte verstreut.

\Salwin
Ganz Ihrer Ansicht, Herr Ingenieur. Solche Gerüchte sind sogar Hochverrat, bedeuten Krieg und Schlimmeres vielleicht. Unsere Grenzen können morgen in Flammen stehen.

\Alexis
Ja, zum Teufel, weshalb fragen Sie dann noch?

\Salwin
Wie bitte? Ich? Meine Aufgabe ist es, die Wahrheit ans Licht zu bringen, meine Aufgabe ist es, die Öffentlichkeit zu beruhigen. Herr Doktor \direction{notiert} Thomsen, nicht wahr, woran sterben die Leute von Dybern?

\Thomsen
\direction{sehr kurz} Am Nebel.

\Salwin
Es gibt also eine neue und geheimnisvolle Nebelkrankheit?

\Thomsen
Ja.

\Salwin
\direction{notiert} Nur bei Dybern?

\Thomsen
Ja.

\Salwin
\direction{in seinem Notizbuch lesend} Und das Mädchen vom Wirtshaus am Rand?

\Generaldirektor
War aus der Gegend von Dybern gekommen.

\Salwin
Keine Ursache zur Beunruhigung?

\Jonas
Solange der Wind sich nicht weiter nach Westen dreht.

\Salwin
Wie bitte?

\Jonas
Eine Witterungskrankheit muss wohl vom Winde abhängig sein. Aber das werden Sie kaum brauchen können.

\Salwin
Ich verstehe Sie nicht, Herr --- Herr Doktor Jonas, nicht wahr?

\Jonas
Ich meine, das gehört nicht auch zur Beruhigung der Öffentlichkeit. Und darauf kommt es Ihnen ja an.

\Salwin
Natürlich kommt es mir darauf an. Die Öffentlichkeit hat ein Recht darauf, die Wahrheit zu erfahren. \direction{zum \theGeneraldirektor} Ich möchte nur noch einige unbeträchtliche Informationen über Ihre chemischen Werke \direction{das Telefon läutet Sturm}

\Generaldirektor
\direction{stürzt hin, hebt den Hörerab, hört lange zu, tiefe Stille} Ich bines selbst --- ja, wo --- gewiss --- ich danke Ihnen wir werden dafür Sorge tragen \direction{legt den Hörer zurück}

Meine Herren,eine schlimme Nachricht Dybern liegt im tiefsten Nebel. Erstickungsanfälle. Der Wind hat sich gedreht. Es herrscht eine Panik. Man flieht zu uns. Die ersten Leute sind schon im Wirtshaus am Rand.

\Salwin
\direction{stürzt an das Telefon, schreit hinein} Überland!

\Thomsen
\direction{schon an der Tür} Ich fahre gleich in mein Krankenhaus. \direction{ab}

\Jonas
\direction{zum \theGeneraldirektor} Ich muss nach Dybern. Verschaffen Sie mir sofort ein paar Gasmasken.

\Generaldirektor
Gasmasken? Ich --- ich habe keine Gasmasken.

\Salwin
\direction{wie vorher ins Telefon} Überland!

\Alexis
\direction{zum \theGeneraldirektor} Im Kino haben siebenhundert Leute Platz. Lassen Sies ofort Auftrag erteilen. Unsere Notbetten ---

\Jonas
\direction{zum \theGeneraldirektor, der wie versteinert dasteht} Gasmasken! Ich brauche Gasmasken.
\end{play}

% ---------


\scene{Im Flüchtlingslager}
\label{scene:III}
\characterlist{\theHeilsarmeeschwester, \theJan, \theLuise, \theGregor, \theErsterMann, \theZweiterMann, \theGeneraldirektor, \theAlexis, \emph{Gestalten}}
\setting{Vorraum des Kinos. Photos an den Wänden. Im Hintergrund eine breite Treppe, die nach abwärts führt. Links ein großes, leeres Buffet, rechts eine Kasse, in der eine junge \theHeilsarmeeschwester sitzt, Herein kommen \theJan, \theLuise und \theGregor. \theLuise und \theGregor bleiben etwas betroffen in der Türe stehen, \theJan geht mitstark posierter Unbefangenheit auf die Kassezu.}
\begin{play}

\Jan
Tag, schönes Kind. Wir brauchen drei Billets für die heutige Abendvorstellung.

\Heilsarmeeschwester
Pst, sprechen Sie doch nicht so laut. Die armen Leute untens chlafen schon.

\Jan
Was Sie nicht sagen. Die Leute schlafen. Das ist mir ja ein nettes Kino.

\Luise
Quatsch nicht, Jan. Was sollen die dummen Witze. Frag lieber die Schwester ---

\Jan
Wieso Schwester. Siehst du denn nicht, dass das ein Leutnant ist. Da muss unsereiner wohl salutieren. \direction{salutiert}

\Heilsarmeeschwester
Der junge Mann ist abergut gelaunt.

\Gregor
Hören Sie nicht auf ihn, liebe Schwester. Und seien Sie uns nicht böse, wenn wir stören. Wir kamen nur eben vorbei, und es ist hier so merkwürdig still.

\Heilsarmeeschwester
Die armen Leute sind müde von alle den Aufregungen. Sie liegen schon seit einer halben Stunde in ihrenBetten. Vorher hatten wir noch einen wunderbaren Gottesdienst. Wollen Sie vielleicht auch morgen einem solchen beiwohnen?

\Jan
\direction{indem er am Buffet herumschnüffelt} Und was zu trinken da war, habt ihr ausgesoffen, Pfui Teufel, ist ja alles leer.

\Heilsarmeeschwester
Gott bewahre. Wir haben Alkoholverbot.

\Luise
Finden da unten wirklich siebenhundert Menschen Platz?

\Heilsarmeeschwester
Die armen Leute liegen zu Dritt und zu Viert in den Betten. Einige sogar auf dem nackten Fussboden. Aber sie ertragen ihr hartes Schicksal mit Geduld.

\Gregor
Und wie lange soll das dauern?

\Heilsarmeeschwester
Das weiss Gott.

\Jan \direction{auf dem Buffet sitzend, mit den Beinen baumelnd} Du sollst den Namen Gottesnicht eitel nennen.

\Heilsarmeeschwester
Wie? Was meint der junge Mann?

\Gregor
Sie dürfen nicht auf ihn hören, Schwester. Er ist ein Hanswurst. Geh, schäm dich, Jan.

\Jan
Warum denn? Ich versteh euch nicht. Das ist doch ein frommer Satz, nicht wahr,Schwester Leutnant. Sogar einer von den zehn Geboten. Wo hab ich das nur unlängst gehört? Wo war das Luise?

\Luise
Ach, Jan, erinnere mich nicht daran.Es war zu grässlich. \direction{setzt sich auf einen kleinen Hocker neben dem Buffet}

\Jan
Du darfst nicht böse sein, Luise, aber es fällt mir wirklich erst jetzt wieder ein.Das war doch die alte Kathrine.

\Luise
Ich begreife überhaupt nicht, dass Barbara sie so auf die Dauer erträgt. Sie geht ihr ja nicht von der Seite seit -seit damals. Und dazu noch das Haus voll von Flüchtlingskindern.

\Heilsarmeeschwester
Sie reden wohl von der Frau vom Wirtshaus am Rand. Ja, das muss eine gute Frau sein. Die gehört nicht zu denen, die den Kopf hängen lassen und an Gott verzweifeln, obwohl er sie doch selbst so hart geschlagen hat.

\Gregor
\direction{neben dem Billetschalter} Woher wissen Sie denn von ihr?

\Heilsarmeeschwester
Wir haben mehrere Frauen dort unten, die ihre Kinder bei ihr zurücklassen mussten. Hallo, was ist das, was wollt ihr denn? \direction{zwei Männer in Hemdärmeln, nur dürftig bekleidet, sind eben die Treppe heraufgekommen}

\ErsterMann
Ich halt es nicht aus, wir sind zu Dritt in einem Bett.

\ZweiterMann
Und ich ersticke.

\Heilsarmeeschwester
Aber, aber, wer wird sich denn so gehen lassen. \direction{die beiden Männer setzen sich mutlos auf die oberste Treppenstufe}

\ErsterMann
Die Frau neben mir stöhnt die ganze Zeit.

\ZweiterMann
Und ich geh auch nicht mehr runter in diese Holle Dann lieber noch in den Nebel hinaus. Da wird einer wenigstens gleich kaputt.

\Heilsarmeeschwester
Wollt ihr die ersten sein, die verzagen? Habt ihr nicht mitgesungen heute Abend bei unserem tröstenden Gottesdienst?

\Jan
\direction{Ist vom Buffet gesprungen und geht suchend an den Wänden hin und her.} Sagen Sie mal, fromme Schwester, gibt es denn in diesem großartigen und herrlichen Kino keine anständige Ventilation

\Heilsarmeeschwester
Ich weiß von nichts. Aber ich bitte euch, liebe Leute \direction{tritt aus dem Schalter heraus und auf die beiden Männer zu} wollet jetzt keine Ausnahme machen. Geht hinunter zu euren Brüdern und ertragt geduldig das gemeinsame Los.

\ErsterMann
Ich pfeif' auf dieses gemeinsame Los. Da steckt man uns in so ein Hundeloch.

\ZweiterMann
Und ich hol meine paar Sachen und geh. Und wenn ich an den Türen betteln muss.

\Heilsarmeeschwester
\direction{mit aufgehobenen Händen} Bitte, bitte, tut das nicht. Niemand weiß, was der Himmel noch über uns vorhängt. Nur so lange wir wahrhaft einig bleiben, kommt zu dem Zorn des Herrn nicht auch noch die rohe Gewalt der Menschen. Werdet nur jetzt nicht zu Landstreichern. Gehorcht der Ordnung und Disziplin.

\ErsterMann
Das Mensch ist verrückt.

\ZweiterMann
Kann ihr keiner das Maul stopfen.

\Heilsarmeeschwester
Ich bete für euch \direction{steht abgewendet mit gefalteten Händen}

\Gregor
\direction{tritt auf die beiden Männer zu} Ihr seid aus Dybern?

\ErsterMann
Ja.

\Gregor
Evakuiert?

\ErsterMann
Was fragst du noch?

\Gregor
Einer von euch kann bei mir schlafen,

\ErsterMann
Oh.

\Gregor
Für zwei ist meine Kammer zu klein,

\ErsterMann
Dann soll der da mit. Er ist noch elender.

\ZweiterMann
Geh du nur selbst, ich bring mich auch durch auf der Straße.

\Heilsarmeeschwester
Halleluja, gepriesen sei Gott und unser Heiland, der Erlöser.

\Jan
\direction{der immer noch an den Wänden herumschleicht} Der andere kann zu uns, nicht wahr, Luise, du hast nichts dagegen?

\Luise
Wir werden schon noch einen Strohsack auftreiben.

\Jan
Aber du bleibst im Bett, und schnarchen darf er nicht.

\Heilsarmeeschwester
Halleluja, halleluja.

\Jan
Hören Sie doch auf mit dem Geplärr, Schwester Leutnant, Sie wecken ja die armen Leute unten.

\Heilsarmeeschwester
Gesegnet sei der Herr, der das menschliche Herz erweicht und den menschlichen Sinn.

\direction{\theErsterMann und \theZweiterMann stehen auf, sehr verlegen}

\ErsterMann
Ich dank auch schön

\ZweiterMann
--- ich dank auch schön.

\Jan
\direction{klopft plötzlich an die Wand} Hallo, Schwester, was ist denn das? Das klingt ja hohl.

\Heilsarmeeschwester
Ich weiss es nicht, ich weiß es nicht.

\Jan
Kommt mal alle her und hört euch das an. \direction{klopft weiter, die andern stellen sich um ihn herum}

\Heilsarmeeschwester
Lassen Sie doch unsere Wände in Ruh.

\Jan
Passt auf, die Wand muss sich sicher wo öffnen lassen. Weg, Schwester. \direction{schiebt sie zur Seite} Sie haben hier nichts zu befehlen.

\Heilsarmeeschwester
Ich habe hier alles zusagen, ich, wer denn sonst. Mir ist das Flüchtlingslager überantwortet, und ichverbiete euch

\Jan
Seht ihr den feinen Strich dort an derWand. Dünn wie Spinn web. Da muss doch irgendwo ein Griff oder ein Hebel sein.

\Heilsarmeeschwester
Hier werden keine Untersuchungen angestellt. Und ich befehle euch im Namen des Herrn

\Jan
Welches Herrn?

\Heilsarmeeschwester
Im Namen Gottes, der denGehorsam verlangt: Geht nach Hause, meine Lieben, ich bitte euch. Ihr werdet es sicherlich nicht bereuen. Und nehmt diebeiden Gäste mit, die ihr in eurer Herzensgüte \direction{die andern sprechen jetzt fast gleichzeitig}

\Luise
Sich doch mal nach hinter dem Schalter,Jan.GREGOR Jetzt möchte ich aber schon selberwissen

\ErsterMann \direction{gemeinsam}
\ZweiterMann
\direction{klopfen an die Wand} Esklingt hohl, ja es klingt wirklich hohl.

\Jan
\direction{hinter dem Schalter, zur Schwester, die ihn zurückhalten will}

Gehen Sie zum Teufel, Sie Leutnant.

\direction{in diesem Augenblick wird die Türe aufgerissen und herein kommen der \theGeneraldirektor und \theAlexis. Beide fahren zuerst zurück}

\Alexis
Schwester, was ist das für eine Volksversammlung?

\Generaldirektor
Wir haben Sie doch ausdrücklich ersucht, für unbedingte Ruhe des Nachts zu sorgen.

\Heilsarmeeschwester
Verzeihung, liebe Herren,aber es ist nicht meine Schuld.

\Alexis
Wer sind die Leute hier? Flüchtlinge?

\Heilsarmeeschwester
\direction{auf die beiden Männer deutend} Nur diese beiden, lieber Herr.

\Generaldirektor
Und die andern?

\Heilsarmeeschwester
Ich weiß es nicht. Vorübergehende Spaziergänger.

\Jan
Wollen eben auch einmal ins Kino gehen.

\Generaldirektor
Hier gibt es weiß Gott keine Abendbelustigung.

\Alexis
Vorübergehende haben hier nichts zusuchen.

\Luise
Ja ist denn hier ein Gefangenenlager?

\Alexis
Sie haben keine Fragen zu stellen.

\Gregor
Komm, Luise, komm, wir wollen gehen.

\Luise
Und wenn die Leute da unten ersticken, wird man sie wohl zu sich einladen dürfen.

\Generaldirektor
Wie? Was reden Sie da? Was soll das heißen?

\Luise
Der Mann dort ist unser Gast. Er kommt mit uns.

\Alexis
Halt, das gibts nicht. Wie konnten Sie nur erlauben, Schwester

\Heilsarmeeschwester
Die guten Menschen warenvoll Erbarmen, als sie die beiden Männer heraufkommen sahen.

\ErsterMann
Und was mich betrifft, die Herren müssen entschuldigen. Ich halt es indem Gestank nicht aus.

\ZweiterMann
Und mich bringt man nicht mehrin diese Hölle.

\Generaldirektor
Aber das geht doch nicht, ihrgehört doch zum Lager, da kann man jetzt keine Ausnahme machen.

\Alexis
Es ist völlig unmöglich,

\direction{in dem Augenblick tauchen sechs, acht Gestalten, alle eben aus dem Bett aufgestanden, hinten auf der Treppe auf}

Also sehen Sie, da haben wir schon die Bescherung.

\Generaldirektor
\direction{auf die eben Erschienenen zutretend und sehr höflich} Ich bittetausendmal um Entschuldigung, falls wirSie stören. Wir kamen nur --- um die Ventilation zu untersuchen.

\Heilsarmeeschwester
\direction{mit ausgebreiteten Armenauf sie zutretend} Geliebte Brüder, vertraut doch auf uns. Die Herren sind nur gekommen, um euch Gutes zu erweisen, dennin Zeiten der Not da blüht erst die Nächstenhilfe.

\Leute
Wir können aber nicht schlafenEs ist zu stickig\ldots Das ganze Lager ist ja auf \ldots Wenn wir der Schwester nichtso fest versprochen hätten

\Heilsarmeeschwester
Seht ihr, ihr habt es mir versprochen. Und wollt ihr nun euer Wort nicht halten \ldots Ich aber komme zu euch\ldots{} ich will mit euch beten \ldots --- Wir wollensingen --- einen Choral --- kommt.

\direction{die \theLeute lassen sich von ihr zurücktreiben, sie folgt ihnen die Treppe hinunter}

\direction{die beiden Männer haben inzwischen \theJan, \theGregor und \theLuise zur Tür hingezogen.}

\ErsterMann
Na denn fix, verduften wir.

\ZweiterMann
Jetzt lasst uns nur nicht auchnoch im Stich.

\Gregor
\direction{zu \theJan} So komm doch schon. Waswillst du denn noch?

\direction{\theErsterMann, \theZweiterMann, \theGregor, \theLuise und \theJan schlüpfen zur Tür hinaus}

\Alexis
\direction{merkt es im letzten Augenblick, will ihnen nach} Halt, halt, das geht nicht.

\Generaldirektor
\direction{hält ihn zurück} So lassen Sie doch. Sie können die Leute ja auch nicht anbinden.

\Alexis
Ich fürchte Unruhen, Aufstand und Gewalt mehr als jeden Nebel.

\Generaldirektor
Jetzt ist nicht Zeit, darübernachzudenken. Jetzt haben wir Wichtigereszu tun. Wo ist der Schlüssel? \direction{sperrt dieTür ab} Und die Tür muss abgesperrtbleiben. Wenn nur die Schwester nichtkommt.

\Alexis
Sie singt doch unten einen Choral. Hören Sie nicht? \direction{man hört auch wirklich nicht zu laut einen Choral}

\Generaldirektor
Dass der Oberst nicht hierist, habe ich erwartet. Es nützt nichts, ihn zu bestellen. Ich glaube, wir müssen auf ihn verzichten.

\Alexis
Und dabei ist doch eben hier alles sein Werk. Es ist und bleibt mir unverständlich---

\Generaldirektor
Es ist nicht Zeit, sich jetztüber ihn den Kopf zu zerbrechen. Rasch, rasch. Die Schwester kann jeden Augenblick kommen.

\Alexis
Solange sie singt, sind wir ungestört. Und auch nachher hält sie die Leute in Zucht. Wenn der Oberst mit uns die Pumpeuntersuchen wollte.

\Generaldirektor
Sie sind ein Narr, Alexis. Wie können Sie immer noch auf ihn warten. Gott weiß, an welchem Problem er brütet, für ihn existiert ja die Wirklichkeit nicht. Er ist der große Theoretiker. Sie aber, Alexis, sind ein Mann der Tat. Zeigen Sie sich als sein Stellvertreter.

\Alexis
Soll ich wirklich ohne sein Beisein ---

\Generaldirektor
Mir scheint gar, Sie fürchten sich.

\Alexis
Herr Generaldirektor

\Generaldirektor
Nehmen Sie sich zusammen, Herr Ingenieur. Wir müssen die Pumpe jetzt ausprobieren. Wenn der Wind sich weiter nach Süden dreht ---

\Alexis
Nein, nein, nein.

\Generaldirektor
Wo ist der geheime Hebel?

\Alexis
\direction{geht zaudernd auf den Billetschalterzu} Hier. Halt. Die Leute hören zu singen auf.

\Generaldirektor
\direction{lauschend} Sie beten jetzt. \direction{man hört fernes Gemurmel} Rasch! \direction{\theAlexis drückt auf einen geheimen Knopf. Die Wand öffnet sich, eine riesengroße Luftpumpenanlage wird sichtbar. Der \theGeneraldirektor stellt sich beobachtend vor die Treppe} Untersuchen Sie rasch, ob alles in Ordnung ist.

\Alexis
\direction{prüft einige Griffe} Es klappt.

\Generaldirektor
Dann los!

\Alexis
Wollen wir wirklich?

\Generaldirektor
Um Gotteswillen, versuchen Sie es doch. Sofort, sofort, man singt noch einen Choral. \direction{man hört wieder gedämpft einen Choral. Alexis manipuliert an der Pumpe herum. Zischen. Ganz kurz. Dann springt \theAlexis wieder an den Billetschalter zurück und schließt durch den Hebel die Wand}

\Generaldirektor
Ausgezeichnet, es wirkt wie ein kühler Wind. Warum nicht mehr?

\Alexis
Die Leute dürfen nicht Verdacht schöpfen.

\Generaldirektor
Wollen Sie sie lieber unten verschmachten lassen?

\Alexis
Es ist unabsehbar, was geschieht, wonnman nur ahnt, worauf wir gerüstet sind.Wir dürfen uns im Frieden doch nicht wieim Kriegsfall benehmen.

\Generaldirektor
Ich glaube, man würde dankbar sein für etwas frische Luft.

\Alexis
Man würde glauben, dass wir Schuldtragen am Nebel. Was jetzt als Unglück scheint und als Schicksalsschlag, erschiene dann als ---

\Generaldirektor
Als Verbrechen.

\Alexis
Herr General direktor!

\Generaldirektor
Fürchten wir uns nicht vor Worten, Alexis. Um so mehr als wir unschuldig sind.

\Alexis
Das sind wir bei Gott.

\Generaldirektor
\direction{Auf die Wand zeigend} Wieviel Sauerstoff ist dort drin?

\Alexis
Für vierzehn bis sechszehn Tage. \direction{in diesem Augenblick erscheint die Heilsarmeeschwester wieder. Sie ist ganz in Ekstase}

\Heilsarmeeschwester
Gelobt sei der Herr! Halleluja, er sei gepriesen. Er hat ein Wunder getan. Beten Sie mit mir. Danken Sie ihm mit mir.

\Generaldirektor
Was ist denn los, was ist geschehen, Schwester?

\Heilsarmeeschwester
Als ich unten mit der Schar der Verzweifelten auf den Knieenlag, als wir beteten, und als wir sangen,da --- es war wie ein kühler Wind. Die armen Leute atmeten auf.

\Generaldirektor
Schon gut, Schwester.

\Alexis
Ist jetzt Ruhe unten?

\Heilsarmeeschwester
Ruhe und Gottvertrauen.

\Alexis
Sie dürfen aber nicht mehr jeden zu Tür hereinlassen. Weisen Sie alle Neugierigen ab. Hier ist doch schließlich kein Theater.

\Generaldirektor
Wissen Sie überhaupt, wer die Menschen waren?

\Heilsarmeeschwester
Nein, ich weiß es nicht,---Aber es waren gute und ehrliche Leute. Nur der eine junge Mann war so sonderbar und so übermütig. Vielleicht betrunken. Er klopfte immer zu auf die Wand.

\Alexis
Wie, was? An die Wand?

\Heilsarmeeschwester
Er sagte, es klänge hohl.

\Alexis
Schwester, Schwester, um Himmel swillen! Und Sie wissen nicht einmal seinen Namen. \direction{dringt auf sie ein}

\Heilsarmeeschwester
\direction{zurück weichend} Ich --- ich weiss doch nichts.

\Alexis
Und zwei Mann vom Lager sind mitgekommen. Nicht auszudenken, was das für ein Gerede geben wird. Vermutungen tauchen auf, Verdächtigungen --- Herr Generaldirektor, wir brauchen Militär.

\Generaldirektor
Was fällt Ihnen ein.

\Alexis
\direction{packt ihn am Ärmel} Kommen Sie sofort. Die Leute müssen verhaftet werden und mit ihnen dieser Doktor Jonas, derüberall nach Gasmasken sucht. Begründungeinerlei.

\Generaldirektor
Sie sind verrückt, Alexis. Wie stellen Sie sich das eigentlich vor?

\Alexis
Die Heilsarmee ist zu schwach. Ich habe es gleich gesagt. Wir brauchen Militär, echtes Militär.

\Heilsarmeeschwester
Die armen Leute sind doch ohnehin so geduldig.

\Alexis
\direction{zieht den Generaldirektor weiter am Ärmel} Rasch, rasch, im Wagen will ich alles weiter erklären. Schwester, Sielassen keinen Menschen mehr herein.

\direction{es klopft sehr heftig an die Tür}

\Stimme
Aufmachen! Aufmachen! Zum Teufelnoch mal! Sofort aufmachen!

\Alexis
Wer ist da?

\Stimme
Machen Sie jedenfalls sofort auf. \direction{wütendes Klopfen} Aufmachen! Aufmachen!

\Alexis
Wer ist draussen?

\Stimme
\direction{Durch das Klopfen hindurch} Der Generaldirektor --- Nebel --- Scharen --- Auto ---Aufmachen!

\direction{\theAlexis sperrt die Tür auf, hereinstürzt \theSalwin}

\Salwin
Herr Generaldirektor, ich wusste es ja,ich sah ja draußen Ihren Wagen. Ich komme eben von einer Rekognoszierungsfahrt. Aufdem Motorrad \direction{kann nicht weiter vor Atemlosigkeit und Aufregung}

\Generaldirektor
Und, und? Was ist?

\Salwin
Der Wind --- der Nebel --- der Wind hat sich gewendet. Die Leute fliehen in Scharen --- der Wind hat sich gewendet --- nach Süden.

\Alexis
Wo flieht man, wo?

\Salwin
Sagen Sie mal, gibt es denn hier kein Telefon?

\Generaldirektor
Zum Teufel, woher kommen Sie denn?

\Salwin
Ich war allein, auf dem Motorrad ungefähr eine Stunde südlich von Dybern.

\Heilsarmeeschwester
\direction{ist in einer Ecke niedergekniet und betet}

\Generaldirektor
Eine Stunde von Dybern! Es istnicht abzusehen. Kommen Sie sofort Alexis. Wir müssen auf der Stelle --- \direction{ab mit Alexis}

\Salwin
Beten Sie nicht, so beten Sie dochnicht Schwester. Sagen Sie lieber, wo ist hier ein Telefon?

\Heilsarmeeschwester
Unten in der Kanzlei. Dort können Sie jetzt nicht hin. \direction{betet weiter}

\Salwin
Ja warum denn nicht?

\Heilsarmeeschwester
\direction{immer noch zwischen Beten} Dort schlafen ein Dutzend Leute.

\Salwin
\direction{will hinunter gehen} Die werden ohnehin bald aufwachen.

\Heilsarmeeschwester
\direction{hält ihn zurück} Nein, das dürfen Sie nicht, man hat mir verbotenmein Gott, wer sind Sie denn?

\Salwin
Salwin, erster Redakteur der Pressekorrespondenz.

\Heilsarmeeschwester
Sie können jetzt unmöglich hinunter. Ich beschwöre Sie im Namen Gottes und unseres Erlösers: bleiben Sie hier! Es war mir ohnehin kaum mehr möglich, die Leute in Ruhe zu halten. Erst ein Wunder hat mir geholfen.

\Salwin
Interessant. Wie war denn dieses Wunder?

\Heilsarmeeschwester
Während sie sangen und beteten, kam ein kühlender, sanfter Wind.

\Salwin
Interessant. Und Sie sind allein hier, Schwester --- wie ist Ihr Name Schwester?

\Heilsarmeeschwester
Mein Name bleibt ungenannt.

\Salwin
Und Sie haben allein hier eine ganze Horde zu bewachen?

\Heilsarmeeschwester
Die armen Leute sind keineHorde.

\Salwin
Aber sie versuchen aus zubrechen.

\Heilsarmeeschwester
Sie sind eben auch nur irrend und schwach.

\Salwin
Und wenn sie erfahren, dass der Nebel näher rückt?

\Heilsarmeeschwester
Dann gebe Gott mir die Kraft, sie zu stützen.

\Salwin
Sie sind sehr mutig. Aber ich muss jetztgehen, Schwester, wollen Sie mir nicht doch Ihren Namen sagen?

\Heilsarmeeschwester
\direction{schüttelt den Kopf}

\Salwin
Na, denn nicht. Aber noch eins, Schwester \direction{schon in der Tür} Können Sie mir eines noch sagen: Sind Sie ganz sicher, dass dieser sonderbare und rätselhafte Nebel wirklich von Gott kommt?

\Heilsarmeeschwester
Es kommt doch alles von Gott.

\Salwin
Oder vom Teufel. \direction{da sie entsetzt zurückweicht} Entschuldigen Sie, ich habe es nicht ganz so gemeint. Beten Sie nurungestört weiter. \direction{ab}

\Heilsarmeeschwester
\direction{sinkt betend in die Knie}

\end{play}

% ---------


\scene{Hungrige Mäuler}
\label{scene:IV}
\characterlist{
	\theKinder,
	\theJosef,
	\theKathrine,
	\theBrix,
	\theMelchior,
	\theBarbara,
	\theAndreas,
	\theGeneraldirektor,
	\theAlexis,
	\theSoldaten
}
\setting{Wirtsstube wie im ersten Bild. Hinten auf der Ofenbank die alte \theKathrine. Vor ihr spielen vier kleine \theKinder Haschen. Gleich darauf kommt \theJosef auch schon zur Tür herein und treibt einen Haufen \theKinder, ungefähr sieben oder acht, vor sich her. Die \theKinder, Knaben und Mädchen sind fünf bis dreizehn Jahre alt. Etwas trübes, dämmeriges Licht.}

\begin{play}

\Josef
Herein mit euch! Und dass mir keiner von der Bande sich noch vor die Tür wagt.

\Junge{1}
\direction{groß}
Aber was sollen wir hier denn tun?

\Maedchen{1}
\direction{klein, mit sehr heller Stimme} Es ist so langweilig.

\Maedchen{2}
\direction{größer}
Wir können doch nicht den ganzen Tag auf der Ofenbank hocken.

\Josef
Es sind nun mal keine vergnüglichen Zeiten.

\Junge{2}
\direction{ganz klein, weinerlich} Ich will  zu meiner Mutter.

\Josef
\direction{nimmt ihn auf den Schoss} Na wart nur,das wird nicht mehr lange dauern.

\Junge{1}
Ja, das heißt es jetzt alle Tage und dann darf man nichtmal mehr vor die Tür.

\Josef
Jetzt schweigt schon still. In einer halben Stunde gibt's was zu essen. Und einstweilen kann Mutter Kathrine euch eine Geschichte erzählen. \direction{fängt an seine Pfeife zu putzen}

\Kinder
\direction{durcheinander} Fein, fein Kathrine soll uns was erzählen--- nicht wahr, Kathrine, du erzählst uns was.

\Kathrine
Ich weiß nicht viel Lustiges zu erzählen.

\Maedchen{1}
\direction{groß}
Dann erzähl eben was Trauriges.

\Maedchen{2}
\direction{groß}
Ja, ja, das ist uns gerade recht. So was zum weinen.

\Junge{3}
Aber sieh zu, dass auch ordentlich was passiert.

\Kathrine
Ich weiß wirklich nichts zu erzählen, Kinder.

\Kinder
Oh bitte, bitte!

\Maedchen{2}
Vom Weihnachtsmann.

\Junge{3}
Ach was, den gibt's doch heuer nicht.

\Maedchen{3}
Von was denn sonst?

\Maedchen{1}
\direction{mit der auffallend hellen Stimme} Von Feen und Elfen.

\Junge{1}
Die gibts doch erstrecht nicht. Die sind längst schon kaputtgegangen im Wald.

\Maedchen{1}
Können die auch den Nebel nicht vertragen?

\Junge{3}
Ich glaube überhaupt nicht an Feen und Elfen.

\Kinder
Also was anderes, Kathrine.

\Maedchen{1}
Weißt du was. Erzähl' uns vom Tod.

\Junge{1}
Bist wohl verrückt.

\Maedchen{3}
Was hat die für Ideen!

\Maedchen{1}
Aber den Tod, den gibt's. Der ist wirklich im Wald.

\Junge{3}
Mein Vater hat gesagt, es gibt keinen Tod.

\Maedchen{4}
Natürlich gibt es keinen Tod, man stirbt einfach.

\Maedchen{1}
Aber warum denn gerade am Weidenweg?

\Junge{1}
Weil dort der dickste Nebel ist, dummes Ding.

\Maedchen{1}
Ich bin gar kein dummes Ding. Und wenn man im Nebel stirbt, dann ist dort der Tod.

\Josef
\direction{schlägt mit der Hand auf den Tisch} Jetzt hört schon einmal auf mit dem Quatsch.

\Maedchen{3}
Zu meiner Großmutter ist aber doch der Tod gekommen. Er war lang und hager und aus lauter Knochen, mit einer Sense.

\Junge{1}
Red keinen Unsinn.

\Junge{4}
Das ist doch nur so ein Gespenst, das gibt es doch nicht.

\Maedchen{4}
Nicht wahr, Kathrine, das gibt es nicht?

\Junge{1}
So sag doch Kathrine. Zweites kleines Mädchen. Nicht wahr, Kathrine, es gibt einen Tod.

\Maedchen{3}
Geht er wirklich herum und klappert mit allen Knochen?

\Maedchen{2}
Und hat er wirklich einen Mund, der immer nur lacht?

\Maedchen{3}
Und die Augen wie Löcher tief drinnen im Kopf?

\Josef
\direction{indem er wieder auf den Tisch schlägt} Wollt Ihr nicht endlich ein Ende machen!

\direction{Der kleine Junge, der auf seinen Schoß gesessen ist, springt erschrocken herunter und läuft zu den \theKinder{}n. Einen Augenblick Stille}

\Maedchen{1}
So sag doch Kathrine.

\Kathrine
Ihr seid wirklich ganz, ganz dumme Kinder. Und überhaupt soll man vom Tod nicht sprechen. Habt ihr gehört, niemals. Weil er sonst wirklich kommt.

\Maedchen{3}
Also, da seht ihr, es gibt ihn doch!

\Kathrine
Schweig still, Mädel, was weißt denn du. So einen Tod wie du meinst, den gibts schon lange nicht. Der war einmal, als die Menschen noch besser waren, als sie halbwegs Frieden hielten auf Erden. Da hatte auch der Tod noch ein Gesicht. Wenn auch kein schönes, aber doch wie ein Mensch, ein gestorbener Mensch.

\Junge{1}
Und jetzt?

\Maedchen{4}
Wie schaut er jetzt denn aus?

\Kathrine
\direction{vorgebeugt und heiser} Jetzt hat der Tod einen Rüssel. Und glatte Glotzaugen rechts und links. Jetzt hat der Tod gar kein Gesicht, Sieht aus wie ein böses und dummes Tier.

\direction{Die Tür wird nach kurzen Klopfen aufgestoßen und herein kommt ein großer schlanker Mann in feldgrauem Mantel und mit Gasmaske vor dem Gesicht von \theBrix. Ihm folgt ein kleiner dicker Mann mit plattem und gemeinem Gesicht, ebenfalls feldgrau, aber mit der Gasmaske in der Hand. Die Kinder stürzen kreischend vor Angst zur Tür hinaus, die in die Küche führt. \theJosef springt auf, lehnt entsetzt an der Wand}

\Melchior
Guten Tag.

\Brix
Guten Tag.

\Kathrine
Herr Jesus, steh uns bei.

\Josef
\direction{etwas mühsam} Guten Tag.

\Brix
Der Generaldirektor schon hier?

\Melchior
Könnt Ihr denn keine Antwort geben?

\Brix
Die Sauerstoffpumpe schon angekommen?---

\Melchior
\direction{auf \theJosef zutretend} Mensch, dir hat es wohl die Red' verschlagen.

\Brix
Genügend Gasmasken im Haus?

\Josef
Ich --- ich weiß --- von nichts.

\Melchior
Was starrst du denn so? Der Herr ist kein Gespenst.

\Kathrine
Vater unser, der du bist in dem Himmel --- es riecht nach Krieg.

\Melchior
Was schwatzt die Alte dort? Die wird noch mehr Gasmasken zu sehen bekommen.

\Kathrine
Ich werde im Leben keine Gasmaskenmehr zu sehen bekommen. Gott ist mir gnädig.

\Brix
Was soll das heißen?

\Kathrine
Ich brauche nichts mehr zu sehen. Gott sei gepriesen.

\Melchior
Das Weib ist närrisch.

\Brix
Was ist denn mit ihr? Was starrt sie mich so an? \direction{weicht einen Schritt zurück}

\Josef
Herr, sie ist blind.

\Brix
Melchior, wir haben keine Zeit hier zuwarten. Rasch, komm. Leg unseren Plan auf den Tisch, der Generaldirektor soll ihn hier finden.

\direction{Melchior legt ein Stück Papier auf den Tisch}

Hier ist alles verzeichnet, hier steht genau, wie der Nebel weiterrückt. Ihr bekommt Sauerstoff und Gasmasken, die Kinder bringt in den Keller. Alles Übrige werdet ihr noch erfahren. Melchior, setz deine Gasmaske auf.

\direction{Da \theMelchior zögert} Was hast du? Gehorche! Wir gehen. Kehrt euch!

\Melchior
\direction{stülpt die Gasmaske auf, sehr militärisch} Zu Befehl, Herr Oberst.

\direction{beide ab}

\Josef
Verflucht nochmal, das ist mir jetzt in die Glieder gefahren. Der Lange war ja das reine Gespenst. Sauerstoff und Gasmasken.

\Kathrine
Man soll nie vom Tod sprechen, sonst ist er dann plötzlich da.

\Josef
Ach halt die Schnauze, alte Hexe. Das war ein Offizier und sein Soldat. Aber was ist mit Barbara. Wenn nur Barbara ---

\direction{Barbara kommt mit einem Sack Kartoffeln herein}

\Barbara
Was war denn hier, was ist denn geschehen? Die Kinder sind wie außer Rand und Band. Ich war eben im Keller, da kam ein Mädchen heruntergestürzt, aber es war nichts aus ihr herauszukriegen.

\Josef
Es kommen böse Zeiten, Barbara. Der Nebel rückt näher. Man hat uns gewarnt. Sauerstoff und Gasmasken---

\direction{Kathrine ist aufgestanden und humpelt auf die Tür zu}

\Barbara
Wo willst du denn hin, Kathrine?

\Josef
Du hörst doch, dass der Nebel näherrückt.

\Barbara
Geh nicht hinaus.

\Kathrine
Mir kann kein Nebel mehr was anhaben.Ich will nach Haus, ich mag euren Sauerstoff nicht.

\Barbara
Du bist ja wahnsinnig. Wirst doch jetzt nicht fortgehen wollen. \direction{will sie zurückhalten}

\Kathrine
Lass los, ich mag nicht. Mich soll keiner mehr retten. Erst haben sie mir mein Mädelchen verdursten lassen, dann haben sie mir meinen Buben vergiftet. Mir aber wollen sie noch eine Gasmaske aufstülpen. Ich tu nimmer mit. Lass los. \direction{ab}

\Barbara
\direction{starrt ihr nach} Vielleicht hat sie recht.

\Josef
Sprich nicht so, Barbara, wir müssen jetzt handeln.

\Barbara
Was redest du von Handeln. Hier können wir uns höchstens noch wehren.

\direction{schüttelt die Kartoffeln in einen Eimer und setzt sich hin, um sie zu schälen}

\Josef
Es gilt nicht uns allein, Barbara. Wir haben das Haus voll Kinder. Fremde Kinder, das ist eine große Verantwortung.

\Barbara
Komm, nimm ein Messer und hilf mir Kartoffel schälen.

\Josef
Ich kann jetzt nicht Kartoffel schälen. \direction{rennt aufgeregt hin und her} Jeden Augenblick soll der Generaldirektor kommen. Man verspricht uns eine Sauerstoffpumpe und Gasmasken.

\Barbara
\direction{sieht auf} Gasmasken? Ja für wen denn Gasmasken?

\Josef
Es könnte doch sein, weißt du Barbara \direction{hantiert nervös an einem Fenster herum} dass der Nebel auch in die Stuben dringt, durch die Türritzen und den Fensterspalt,und dann ---

\Barbara
Und dann?

\Josef
Dann brauchen wir die Gasmasken auch. Dass du immer noch Kartoffel schälen kannst, Barbara.

\Barbara
Die Kinder müssen doch zu essen haben. Sag mal, bekommen die Kinder vielleicht auch solche Gasmasken?

\Josef
Ich denke schon.

\Barbara
\direction{hört auf zu schälen} Gibt es denn so kleine Gasmasken, so ganz kleine auch, ganz winzig kleine.

\Josef
Es wird wohl große und kleine geben. Aber zuerst werden wir die Kinder einmal in den Keller hinunterbringen. Du hast recht, sie müssen zu essen bekommen.

\direction{setzt sich neben sie und beginnt ebenfalls Kartoffeln zu schälen}

Möglichst bald. Hörst du sie in der Dachkammer oben. Das tobt und lärmt. Wir wollen gut sein gegen die armen Würmer. Stelle dir vor, wenn einmal unser eigenes

\Barbara
\direction{sehr schroff} Schweig still.

\Josef
Nun man kann nicht wissen. Gott wird es uns sicher vergelten. Und wenn einmal unser eigenes

\Barbara
\direction{rasend} so schweig doch schon, ich will das nicht hören.

\Josef
Aber Barbara

\Barbara
\direction{packt \theJosef{}s Arm} Unser eigenes Kind soll auf einer Wiese spielen, in einer wunderbaren, klaren, leuchtenden Luft, Josef, wir wandern aus in ein Land, wo es keinen solchen Nebel gibt, wo es keinen solchen Nebel geben kann. Josef, ich will hier kein Kind aufziehen. Nicht wahr, Josef, wir wandern aus. Auf eine Insel, mitten im Meer. Es muss doch noch einen Ort geben auf dieser Erde, wo solch ein Nebel nicht möglich ist.

\Josef
Aber Barbara, um Gotteswillen!

\Barbara
\direction{plötzlich zusammensinkend} Josef,Josef, ich fürcht' mich so. Mein Kind soll nicht in einem Keller zur Welt kommen. Man darf ihm keine Gasmaske auf das Köpfchen ---

\Josef
Barbara, es ist eine harte Zeit, Du wirst doch jetzt nicht den Mut verlieren, du warst ja so tapfer die letzten Tage. Schau, der Nebel ist wie eine Krankheit. Gott hat sie uns in das Land geschickt, aber auch dieser Nebel wird vergehen, jede Krankheit hört einmal auf zu wüten. Es hilft nichts, wenn man dem Unglück trotzt, das Schicksal ist ja doch stärker als wir.

\Maedchen{1}
\direction{mit der besonders hellen Stimme steckt den Kopf zur Tür herein} Ist der Tod noch da?

\marginnote{\emph{unartiges Kind}.\par(läufiges Mutterschwein)}
\Josef
Willst du wohl, du kleine Range.

\Maedchen{1}
\direction{kommt etwas näher} Gibt es bald was zu essen?

\Barbara
Sehr bald, mein Kind. Ruf doch einmal die andern in die Küche hinunter. Sie sollen Ordnung machen und mir helfen \direction{greift nach dem Messer und schält weiter Kartoffeln} Rasch Josef, die Kinder sollen nicht hungern.

\direction{\theJosef nimmt sein Messer, das kleine Mädchen ab}

\Josef
Hast recht Barbara, das ist jetzt das wichtigste. Kocht das Wasser schon? So mach doch kein so finsteres Gesicht, immer Kopf hoch, du warst doch sonst keine von denen, die leicht verzagen. Musst an unser Kleines denken Barbara. \direction{\theBarbara zuckt zusammen} Ja, Barbara, musst daran denken, was für so ein Kind gut ist ungesund. Du Barbara, ich hab dich schon lang nicht singen gehört.

\Barbara \direction{nimmt den Trog mit den geschälten Kartoffeln und trägt ihn in die Küche. Die Türe bleibt offen. Aus der Kücheheraus fragt \theBarbara sehr heiser} Was soll ich denn singen?

\Josef
Na, halt so ein kleines Lied, so wie früher, du weißt schon.

\Barbara
\direction{singt mit einer rauen, gebrochenen Stimme} Eia popeia, was raschelt im \direction{bricht ab}

\direction{Die Tür wird aufgerissen, \theAndreas stürzt herein. Er ist verstört und erhitzt}

\Andreas
Barbara, wo ist Barbara \direction{sinkt auf eine Bank}

\Barbara
\direction{in der Tür} Was ist geschehen, Andreas?

\Andreas
Jan ist verhaftet. Militär. Man will auch mich. Könnt Ihr mich verstecken?

\Barbara
\direction{ist ins Zimmer getreten} Was hast du getan, Andreas?

\Andreas
Das Maul aufgemacht.

\Josef
\direction{schließt die Tür zur Küche, in der ein paar \theKinder auftauchen} Pst, sprich nicht so laut, daneben sind Kinder.

\Barbara
\direction{bringt Andreas ein Glas Wasser} Da hast du Andreas, trink erst einmal. Und mach das Hemd zu. Sei ganz ruhig, dann kannst du erzählen. Hast ja nicht einmal einen Mantel an und rennst so durch den Nebel.

\Andreas
Das ist kein Nebel, das ist doch kein Nebel. Glaubt das nicht länger. Sie wollen uns ja nur was vorlügen. Schwindel, Schwindel, nichts als Schwindel.

\Josef
Mensch, nimm dich in acht. Was redest du da.

\Barbara
Kein Nebel?

\Andreas
In der Stadt weiå es beinah schon einjeder. Der Jan hat es gesagt, der Jan hat seine Nase in mehr hineingesteckt, als ihr ahnt, der kann auch was erzählen von Sauerstoffpumpen, die heimlich in den Mauern stecken. Deshalb haben sie ihn auch hopp genommen. Und die Luise hat geschrien und gebrüllt wie verrückt. Die hat ihn gern. Ob meine Agnes auch so geschrien hätte? Was meinst du Barbara?

\Barbara
Sprich nicht davon.

\Andreas
Ich wollte ihm jedenfalls helfen, dem Jan, und auch der Doktor war dabei, der junge, der Doktor Jonas. Vor dem Kino war es, ihr wisst schon, dem neuen. Jetzt kann man sich denken, weshalb es unter der Erde ist. Na, und da wollten sie auch mir an den Kragen. Ausnahmezustand heißt man das. Und da bin ich noch rasch auf und davon.

\Josef
Ich verstehe aber noch immer kein Wort.

\Barbara
Du hast gesagt, dass es kein Nebel ist?

\Andreas
Es ist kein Nebel, und wenn sie sich noch tausend gelehrte Kommissionen kommen lassen.

\Barbara
Was ist es denn Andreas? Um Gotteswillen, so sprich doch.

\Andreas
Ja wisst ihr es denn wirklich noch nicht. \direction{stößt die Worte mühsam hervor} Gift ist es --- Gas --- Giftgas.

\Josef
Das ist nicht wahr.

\Barbara
Herrgott im Himmel.

\Andreas
In unserer eigenen Fabrik erzeugt. Den Herrschaften ist was ausgekommen. Die wissen selber nimmer aus noch ein. Aber vorbereitet waren sie darauf, könnt ihr mich verstecken, wenn man mich suchen kommt?

\Josef
Du kannst im Schuppen bleiben oder in der Dachkammer. In den Keller kommen die Kinder.

\Barbara
Du Josef, hör zu. Wenn das nicht der Nebel ist, dann ist es ja gar kein Unglück und kein Schicksal und keine Krankheit.

\Josef
Gott hat uns schwer geprüft.

\Andreas
Gott! Gott hat keine Giftgasfabrik.

\Barbara
Du Josef, hör zu, mir fällt noch was ein. Wenn es nicht der Nebel ist, unsere Agnes, die ist uns ja dann gar nicht bloß gestorben.

\Josef
Was denn Barbara, was meinst du damit?

\Barbara
Josef, unsere Agnes, das Kind, das Mädel ist uns ermordet worden.

\Andreas
Ich spreng die Fabrik in die Luft, ich vertilge diese Bestien.

\Josef
Schweigt still, um Gotteswillen, die Kinder nebenan---

\Barbara
Sie haben unsere Agnes vergiftet.

\Andreas
Ihr bei lebendem Leib die Lungen verbrannt.

\Josef
Barmherziger Himmel, was redet ihr da, das kann doch nicht sein.

\Andreas
Geh raus in den Wald und sieh dir das an. Dort ist jeder Grashalm verreckt.

\Barbara
An unseren Weiden wird es heuer keine Kätzchen mehr geben.

\Andreas
Agnes hat diese Kätzchen so gern gehabt.

\Josef
Ich glaub es nicht, ich kann es nicht glauben.

\Andreas
\direction{springt auf} Hört ihr, ein Auto! Sie kommen schon, um mich zu holen.

\Barbara \direction{schiebt ihn zu einer Seitentüre hinaus} Rasch hinaus in den Schuppen und wenn sie herein sind, läufst du schon auf den Dachboden rauf.

\direction{Gleich darauf kommen der \theGeneraldirektor, \theAlexis, die \theHeilsarmeeschwester und drei \theSoldaten mit einer riesigen Kiste}

\Generaldirektor
Wir sind doch hier im Wirtshaus am Rand?

\Josef
Jawohl.

\Alexis
Sie sind der Wirt?

\Josef
Jawohl.

\Alexis
Das ist Ihre Frau?

\Josef
Jawohl. \direction{\theBarbara tritt mit verschränkten Armen in den Hintergrund}

\Generaldirektor
\direction{auf sie zutretend} Liebe Frau, wir bewundern Sie alle. Sie haben in diesen schlimmen Zeiten und unter den schwierigsten Umständen den schönsten und edelsten Mut bewiesen. Ich danke Ihnen.

\Barbara
Ich brauch keinen Dank.

\Generaldirektor
Sie sollen nicht denken, dass wir nicht wissen, was ein Haus voll fremder Kinder für Sie jetzt bedeutet. Und deshalb haben wir Ihnen die brave Schwester hier als Hilfe mitgebracht.

\Barbara
Ich brauch keine Hilfe.

\Heilsarmeeschwester
Liebe Freundin, Sie werden schon noch manche Hilfe brauchen. Weisen Sie mich nicht ab. Ich habe schwerere Arbeit geleistet.

\Alexis
Die Schwester wird die Kinder herrlich versorgen. Sie wird mit ihnen beten und mit ihnen singen.

\Barbara
In meinem Haus wird nicht mehr gebetet.

\Heilsarmeeschwester
Wie?

\Barbara
Und überhaupt nicht mehr gesungen.

\Generaldirektor
Aber liebe Frau---

\Josef
Barbara, was fällt dir denn ein?

\Barbara
In diesem Haus ist kein Platz mehr für Kinder. Ich weiß nicht, wer die Herren sind, aber falls Sie es noch nicht wissen sollten! In diesem Haus ist wer gestorben, den man vergiftet hat, ein junges Mädel, selber noch ein Kind.

\Generaldirektor
Aber Beste, was denken Sie denn. Wir wissen doch alle von Ihrem Unglück.

\Barbara
Das war kein Unglück, Herr.

\Generaldirektor
Wie meinen Sie?

\Barbara
Das war ein Verbrechen.

\Josef
Herr, entschuldigen Sie, sie hat den Tod der Schwester noch nicht überwunden.

\Barbara
Was sprichst du von Tod. Du weißt so gut wie ich%;B es war ein Mord.

\Alexis
So ist das wahnsinnige Gerücht auch schon bis hierher vorgedrungen.

\Generaldirektor
Um Gotteswillen, Sie werden das doch nicht glauben. Hören Sie! Drei Kommissionen von Chemikern und Ärzten haben entschieden, dass es der Nebel ist, nichts anderes als der Nebel. Wir bringen Ihnen die besten Schutzmaßnahmen, wir tun ja, was in unseren Kräften liegt. Vertrauen Sie uns doch. Sehen Sie, hier stehe ich vor Ihnen, der Generaldirektor der ungeheuren Werke---

\Josef
Der Generaldirektor?

\Generaldirektor
Ja, das bin ich.

\Josef
Dort auf dem Tisch liegt ein Zettel für den Generaldirektor. Ein Herr --- ein Mann --- ein Offizier hat ihn hingelegt. Für den Generaldirektor, hat er gesagt.

\direction{\theGeneraldirektor stürzt auf den Zettel zu}.

\Alexis
Wie? Was? Ein Offizier? Das war Brix.Das kann nur der Oberst gewesen sein. \direction{siehtdem Generaldirektor über die Schulter}

\Generaldirektor
Alexis, entsetzlich! Es wardie höchste Zeit. Der Nebel muss ja ineiner halben Stunde schon -

\Alexis
Unmöglich sein, das gibt es nicht.

\Generaldirektor
Brix sagt niemals, was ernicht weiss. Wenn er doch auf uns gewartethätte.

\Alexis
Dann ist jetzt keine Zeit mehr zu verlieren. \direction{zu den Soldaten} Packt die Gasmasken aus. Dann muss die Pumpe gleichaufgestellt werden. Alle Kinder sollensofort in den Keller.\direction{Die Soldaten werfen einen Haufen Gasmasken auf den Tisch}
\Generaldirektor
Ihr müsst alle Türen und Fenster verstopfen. Wo sind denn die Kinder?

\Josef
\direction{macht die Tür zur Kücke auf} Hier. Kommtmal herein.\direction{Ein paar Kinder kommen neugierig undverlegen aus der Küche}

\Junge{5}
Gibts was zu essen?\direction{Da erblicken die Kinder die Gasmaskenauf dem Tisch und stürzen kreischendhinaus}

\Alexis
Das ist ja eine nette Wirtschaft.

\Heilsarmeeschwester
Ich werde mit den Kleinensprechen. \direction{ihnen nach}

\Alexis
\direction{zu den Soldaten} Kommt gleich mit mir, damit wir die Pumpe aufstellen. \direction{zu \theJosef} Führen Sie uns sofort in den Keller. \direction{\theAlexis, \theJosef und die beiden \theSoldaten ab}

\Barbara
Ich brauch keine Gasmasken und keine Pumpe in meinem Haus. Nehmen Sie die Kinder fort von hier.

\Generaldirektor
Aber liebste, beste Frau, Sie werden uns doch jetzt nicht im Stich lassen wollen. Und alles nur wegen eines Gerüchts.

\Barbara
Wenn ein Gericht nicht wahr ist, sperrt man die Leute nicht ein. Wenn ein Gerücht nicht wahr ist, braucht man kein Militär.

\Generaldirektor
Aber das muss doch jetzt ganz gleichgültig sein. Jetzt gilt es Menschenleben zu retten. In Zeiten der Not, da halten doch alle Menschen zusammen.---

\Barbara
Das ist jetzt aber keine Not.

\Generaldirektor
Herr des Himmels, wenn das keine Not sein soll!

\Barbara
Not ist, wenn der Schnee das Dach eindrückt, wenn das Wasser die Mauern wegreißt, wenn die Hitze das Getreide verdorrt, wenn die Krankheit den Menschen frisst. Wenn aber der Mensch selber den Menschen frisst, ihm nichts zum leben lässt,nicht einmal mehr die Luft, das ist nicht Not, einerlei, wen es trifft.

\Generaldirektor
\direction{zurückweichend} Ja, was denn sonst als Not?

\Barbara
Krieg.

\Generaldirektor
Sie werden doch nicht behaupten wollen ---

\Barbara
Dort auf der Ofenbank ist mir das Mädel gelegen, es hat sie verbrannt, inwendig verbrannt, sie hat nach Wasser geschrien,und auch das Wasser hat sie verbrannt. Nehmen Sie die Kinder weg, gleich, sofort, hier ist kein Haus mehr für Kinder.

\Generaldirektor
Aber liebe Frau, Sie, die Sie doch selbst ein Kind erwarten ---

\Barbara
Sie verstehen das nicht \direction{geht zum Fenster und öffnet die innere Scheibe}

\Generaldirektor
Was machen Sie da?

\Barbara
\direction{mit der Hand am Griff des äußeren Fensters} Wenn dieses Kind einmal zur Welt kommt, dann soll keine gute Frau da sein, die ihm Kartoffel kocht und es in einen Keller sperrt, weil draußen überall so ein Nebel ist. Dann soll eine Frau kommen und ein Mann oder vielleicht auch viele Frauen und viele Männer, die ganz was anderes tun, wenn man ihnen die Luft verpesten und die Kinder vergiften will. Nehmen Sie die Kinder fort, oder ich reiße das Fenster auf.
\end{play}

% ---------


\scene{Nachbohren}
\label{scene:V}

\act{Zeiter Akt}
\scene{Irrungen}
\label{scene:VI}
\scene{Zusammenbruch}
\label{scene:VII}

\end{document}
